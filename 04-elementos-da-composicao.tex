\section{Elementos básicos da composição do texto}

\subsection{Palavras de controle e texto}\label{sec:palavras-de-controle}

Para construir um texto usamos aqui nada mais que palavras e comandos,
simplesmente. Nesta seção veremos como eles se coordenam.

Durante o processamento de seu texto, a maior parte do tempo o
\LaTeX\ apenas encontra letras comuns, que prepara para colocar em um
parágrafo. Algumas vezes, no entanto, ele encontra uma barra --- o que
significa que uma sequência de controle foi encontrada. Se o caractere
seguinte não for uma letra, trata-se de um 
\emph{caractere de controle} (vide seção \ref{subsec:seq-controle}), e
o \LaTeX\ continua processando o texto, levando em conta, claro, o
significado do comando que encontrou. Já se após a barra há uma
letra, o sistema se prepara para ler uma palavra de controle: continua
a ler caracteres do texto até encontrar o primeiro caractere que não
seja uma letra\footnote{Lembre-se: letras são os caracteres de
  \texttt{a} a \texttt{z} e de \texttt{A} a \texttt{Z}.}. Se a palavra
de controle é seguida de espaços em branco, \emph{eles são
  ignorados}; e se ela é seguida de \emph{uma} quebra de linha, ela é
ignorada também. O que acontece com mais quebras de linha?
Experimente! Exercício: como você faria para escrever \TeX emplo?

Se os espaços em branco são ignorados, como faço para que uma palavra
de controle como \LaTeX\ seja seguida por um espaço (como foi aqui)?
Os espaços são necessários após uma palavra de controle para definir
seu fim --- caso contrário, o \LaTeX\ consideraria que \verb'\TeXemplo' é
uma palavra de controle só. Mas qualquer coisa que permita ao sistema
identificar que a palavra de controle terminou serve para o mesmo
propósito. Assim, se você colocar um grupo vazio seguindo o comando
(`\verb'\TeX{} emplo''), ou colocar um grupo envolvendo o comando
(`\verb'{\TeX} emplo''), o espaço que segue o fim do grupo será
preservado. Há ainda um outro modo, mais simples, de colocar um espaço
logo depois de uma palavra de controle: basta usar o comando
`\verb'\ '', que é uma barra seguida de um espaço. Esse comando
simplesmente produz um espaço em branco, e podemos escrever
`\verb'\TeX\ emplo'' para obter o \TeX\ emplo.

\subsection{Títulos, autor e data de documentos}

Em muitas classes de documentos, estão disponíveis os comandos para
definir o título, o(s) autor(es) e a data do documento. Cada classe
exibe essa informação de um modo, mas em boa parte delas você define o
título com um comando \verb'\title{Minhas Férias}', o autor usando o
comando \verb'\author{YoMoiIchEu}'. A data é composta automaticamente
com a data em que o documento for processado (no idioma do
documento). Você pode escolher a data usando o comando
\verb'\date{Muito, muito tempo atrás}'.

Depois de especificados o título e o autor (mais de um autor pode ser
declarado, separando-se seus nomes por \verb'\and'), você escolhe o
ponto do texto no qual quer que apareçam, e usa o comando
\verb'\maketitle'. Voilà!

\subsection{Sobre espaçamento horizontal}\label{sec:espacos}

Nem todos os espaços são iguais. Não só variam em tamanho, mas possuem
comportamentos distintos. Falaremos a seguir dos 
\emph{espaços duros}\index{espacos duros@espaços duros}
e de espaços um pouco mais largos, embora isso esteja longe de esgotar
o assunto.\footnote{Espaço preenchido e espaço em branco estão em
  constante interação em qualquer peça de composição visual.} Falaremos dos espaços mais comuns no texto, como os que separam palavras. Algumas vezes (principalmente quando abordarmos a escrita de expressões matemáticas), outros tipos de espaçamento serão necessários. 

Todo parágrafo justificado, isto é, que tem as margens direita e
esquerda alinhadas verticalmente, exige que o espaçamento entre palavras seja ``elástico'',
aumentando ou diminuindo conforme a necessidade. O \LaTeX\ possui um
mecanismo interno elaborado para o gerenciamento desses espaços (que não descreveremos aqui). Ainda
assim, é importante saber que alguns espaços são mais elásticos do que
outros, e que os espaços comuns possuem limites de compressão e
expansão.

Por exemplo, o espaço que segue o ponto final (ou a interrogação, ou a exclamação) em uma frase é mais
elástico que o espaço que une as demais palavras. Mas como o fim de
uma frase é identificado?

Por padrão, o \LaTeX\ assume que um ponto final --- ou outra pontuação
como `?{}', ou `!{}', ou `\ldots'
(reticências\index{reticencias@reticências} são produzidas usando o
comando \verb'\ldots') --- marca o fim de uma
frase sempre que, e somente quando, for precedida por uma letra
minúscula. Na maior parte dos casos esse comportamento é exatamente o
que precisamos, mas nem sempre.

Títulos como Dr.\ não terminam uma frase, na maior parte dos
casos. Isso é resolvido usando `\verb*'Dr.\ ''.\footnote{Como vimos na seção \ref{sec:palavras-de-controle}.}
Por outro lado, existem casos em que uma letra maiúscula seguida de
pontuação \emph{termina} uma frase: URSS\@. Para indicar que o ponto
final após uma letra maiúscula termina a frase, existe o comando `\verb'\@'\thinspace', correto?
\begin{center}\footnotesize
\verb'Entendi, OK\@. A frase terminou no último ponto. E nesse também. E nesses.'
\end{center}

Falaremos agora dos espaços \emph{duros}. Existem palavras que estão naturalmente ligadas, e não toleram quebras de linha entre si. Isto acontece em expressões como ``seção~\ref{sec:espacos}'', ``Dr.\ House''. Frequentemente é preciso prestar atenção a expressões como ``Teorema de~Kuratowski'', ``Associação Contra os~Maus-tratos a~Espécies'', em que nem todos os espaços são duros, mas alguns são.
Para produzir um espaço duro em \LaTeX, usa-se o til `\verb'~''. Por exemplo, ``\verb'5~cm'''. Com um pouco de prática se torna natural o a introdução desses espaços quando apropriado.

