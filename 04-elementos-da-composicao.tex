\section{Elementos básicos da composição do texto}

\subsection{Palavras de controle e texto}

Para construir um texto usamos aqui nada mais que palavras e comandos,
simplesmente. Nesta seção veremos como eles se coordenam.

Durante o processamento de seu texto, a maior parte do tempo o
\LaTeX\ apenas encontra letras comuns, que prepara para colocar em um
parágrafo. Algumas vezes, no entanto, ele encontra uma barra --- o que
significa que uma sequência de controle foi encontrada. Se o caractere
seguinte não for uma letra, trata-se de um 
\emph{caractere de controle} (vide seção \ref{subsec:seq-controle}), e
o \LaTeX continua processando o texto, levando em conta, claro, o
significado do comando que encontrou. Já se após a barra encontra uma
letra, o sistema se prepara para ler uma palavra de controle: continua
a ler caracteres do texto até encontrar o primeiro caractere que não
seja uma letra\footnote{Lembre-se: letras são os caracteres de
  \texttt{a} a \texttt{z} e de \texttt{A} a \texttt{Z}.}. Se a palavra
de controle é seguida de espaços em branco, \emph{eles são
  ignorados}, se ela é seguida de \emph{uma} quebra de linha, ela é
ignorada também. O que acontece com mais quebras de linha?
Experimente! Exercício: como você faria para escrever \TeX emplo?

Se os espaços em branco são ignorados, como faço que uma palavra
de controle como \LaTeX\ seja seguida por um espaço (como foi aqui)?
Os espaços são necessários após uma palavra de controle para definir
seu fim --- caso contrário, o \LaTeX\ consideraria que \verb'\TeXemplo' é
uma palavra de controle só. Mas qualquer coisa que permita ao sistema
identificar que a palavra de controle terminou serve para o mesmo
propósito. Assim, se você colocar um grupo vazio seguindo o comando
(`\verb'\TeX{} emplo''), ou colocar um grupo envolvendo o comando
(`\verb'{\TeX} emplo''), o espaço que segue o fim do grupo será
preservado. Há ainda um outro modo de colocar um espaço logo depois de
uma palavra de controle, que é bem simples: basta usar o comando
`\verb'\ '', que é uma barra seguida de um espaço. Esse comando
simplesmente produz um espaço em branco, e podemos escrever
`\verb'\TeX\ emplo'' para obter o \TeX\ emplo.

\subsection{Títulos, autor e data de documentos}

Em muitas classes de documentos, estão disponíveis os comandos para
definir o título, o(s) autor(es) e a data do documento. Cada classe
exibe essa informação de um modo, mas em boa parte delas você define o
título com um comando \verb'\title{Minhas Férias}', o autor usando o
comando \verb'\author{YoMoiIchEu}'. A data é composta automaticamente
com a data em que o documento for processado (no idioma do
documento). Você pode escolher a data usando o comando
\verb'\date{Muito, muito tempo atrás}'.

Depois de especificados o título e o autor (mais de um autor pode ser
declarado, separando-se seus nomes por \verb'\and'), você escolhe o
ponto do texto em que quer que apareçam, e usa o comando
\verb'maketitle'. Assim:

\maketitle

Et voilà!

\subsection{Sobre espaçamento}
\verb!\@.! para fim de frase

