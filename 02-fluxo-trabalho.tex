\section{Rotina de trabalho}

Escrever um documento usando \LaTeX, não é muito diferente de escrever
um documento numa máquina de escrever, embora o resultado seja
bastante diverso. Em geral, você irá abrir um programa para edição
de texto%
\footnote{%
  Existem mesmo alguns programas sofisticadíssimos
  para a edição de documentos \LaTeX, mas este não é nosso foco
  aqui.}% todo: citar exemplos de TeXmakers da vida etc..
, escreverá o texto, e pedirá ao \LaTeX\ que gere o
documento \extensao{pdf} (ou~\extensao{ps}, ou~\extensao{dvi})que
desejar. Simples assim.

Não abordaremos aqui o processo de instalação do \LaTeX, ou como
preparar o seu computador para processar os arquivos \extensao{tex}. A
boa notícia é que essa é uma tarefa simples. Há várias páginas na
internet que explicam detalhadamente como instalar o programa,
independentemente de qual seja o sistema do seu computador. Abaixo seguem
alguns links de páginas que vale a pena visitar.

Certamente, vale a pena ler 
\begin{itemize}
\item \emph{\LaTeX, A Document Preparation System}, de Leslie Lamport
  (criador do \LaTeX),
  e
\item \emph{The \TeX book}, de Donald E.~Knuth (criador do \TeX, que é
  a base sobre o qual se assenta o \LaTeX).
\item Wiki brasileiro de \TeX: \url{www.tex-br.org}
\item Getting to Grips with \LaTeX (tutoriais por Andrew Roberts): \url{http://www.andy-roberts.net/misc/latex/}
\item Apostila de \LaTeX\ da Universidade Federal Fluminense: \url{www.telecom.uff.br/pet/petws/downloads/apostilas/LaTeX.pdf}
\item \TeX\ Users Group: \url{www.tug.org}
\item Comprehensive \TeX\ Archive Network: \url{www.ctan.org}
\end{itemize}
