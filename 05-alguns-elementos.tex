\section{Alguns elementos do texto}

Aqui veremos exemplos de três ambientes qeu delimitam construções
comuns em textos: listas (de vários tipos), trechos com lateral
rasgada (não-justificados), e elementos ``flutuantes''.

\subsection[Listas]{Listas {\it \&} Cia.}

Listas encarnam uma função que é misto de destaque e segregação. A um
só tempo, o conteúdo de uma lista é apartado da corrente do texto,
enquanto que cada um de seus itens tem sua unidade e individualidade
reforçadas, como você pode ver experimentando com os exemplos abaixo.
Eles mostram, respectivamente, listas\index{listas}
não-numeradas\index{listas!nao numeradas@não numeradas} e
numeradas\index{listas!numeradas} e descrições.

\begin{center}\footnotesize\hrule\smallskip
\begin{tabular}{cc}
\begin{minipage}{.465\textwidth}
\begin{verbatim}
\begin{itemize}
\item cebola,
\item açafrão, e
\item alho.
\end{itemize}
\end{verbatim}
\end{minipage} &
\begin{minipage}{.465\textwidth}
\begin{itemize}
\item cebola,
\item açafrão, e
\item alho.
\end{itemize}
\end{minipage}
\end{tabular}
\smallskip
\end{center}

\begin{center}\footnotesize\smallskip
\begin{tabular}{cc}
\begin{minipage}{.465\textwidth}
\begin{verbatim}
\begin{enumerate}
\item cebola,
\item açafrão, e
\item alho.
\end{enumerate}
\end{verbatim}
\end{minipage} &
\begin{minipage}{.465\textwidth}
\begin{enumerate}
\item cebola,
\item açafrão, e
\item alho.
\end{enumerate}
\end{minipage}
\end{tabular}
\smallskip
\end{center}

\begin{center}\footnotesize\smallskip
\begin{tabular}{cc}
\begin{minipage}{.465\textwidth}
\begin{verbatim}
\begin{description}
\item[cebola] Muito empregada p/ temperar.
\item[açafrão] Também.
\item[alho] Idem.
\end{description}
\end{verbatim}
\end{minipage} &
\begin{minipage}{.465\textwidth}
\begin{description}
\item[cebola] Muito empregada p/ temperar.
\item[açafrão] Também.
\item[alho] Idem.
\end{description}
\end{minipage}
\end{tabular}
\smallskip\hrule
\end{center}

\subsection{Citando textualmente}

Existem dois ambientes comumente usados para incluir excertos de
outros textos no seu. Os ambientes \ambiente{quote} para trechos
curtos, e \ambiente{quotation} para trechos mais longos.

\subsection{Alinhamento}\label{sec:alinhamento}

Boa parte dos textos possui alinhamento justificado, i.e., possui
ambas as margens retas e paralelas. Nem sempre isso é
desejado. Existem muitas maneiras de definir o alinhamento do texto:
falamos de duas delas aqui.

\subsubsection{Texto não-justificado}

\begin{flushleft}
No ambiente
\ambiente{flushleft}%
\index{flushleft@\ambiente{flushleft}}%
\index{alinhamento!alinhado a esquerda@alinhado à esquerda}, o texto é
``empurrado'' para a esquerda. Os espaços não são nem esticados nem
comprimidos. O efeito resultante são linhas de comprimento variável, 
 por vezes uma opção interessante de diagramação.
\end{flushleft}

\begin{flushright}
Simetricamente,
\ambiente{flushright}%
\index{flushright@\ambiente{flushright}}%
\index{alinhamento!alinhado a direita@alinhado à direita} tem o
comportamento que seria de se esperar, fazendo  parágrafos  ficarem
com a esquerda~\emph{rasgada}, isto é, sejam empurrados para a
direita. 
\end{flushright}

\begin{center}
Mais não é preciso dizer: \ambiente{center}\index{center@center (ambiente)}\index{centralizado}.
\end{center}

\subsection{Elementos
  flutuantes}\label{sec:floats}\index{flutuante@flutuante (elemento)}
% TODO: falar de listoffigures e listtof

Tipógrafos atentam para uma série de características na disposição do
texto que frequentemente passam despercebidas ao nosso consciente. Uma
delas é o equilíbrio entre o texto que se espalha pelas páginas e os
demais elementos, como figuras e tabelas, que pontuam a paisagem aqui
e ali. 

O \LaTeX\ toma várias precauções na disposição desses elementos,
ditos~\emph{flutuantes} (porque sua posição não é fixa no texto como a
de uma palavra em uma frase). É como se os elementos fossem troncos de
árvore à deriva sobre a correnteza de palavras que compõe o texto.
Figuras e tabelas são somente alguns exemplos de elementos
flutuantes. Eles são ambientes (respectivamente~\ambiente{figure}
e~\ambiente{table}) que encapsulam o conteúdo que irá
flutuar.\footnote{Falamos de \emph{o que} acontece com um elemento
  flutuante: ele deriva. \emph{Como} ele o faz, isto é, como o
  \LaTeX\ tenta determinar um posicionamento adequado, é um assunto um
  pouco mais delicado, que exige uma bagagem técnica (tanto
  tipográfica quanto \TeX nica mesmo) que extrapola nosso escopo. O leitor
  interessado não encontrará dificuldade para acessar uma miríade de
  discussões a sobre \emph{algoritmos de posicionamento de
    \emph{floats} do \LaTeX} na internet, ou em
  livros: (respectivamente) no
  \emph{TUGboat}~\cite{float-positioning-tug-proc-2000}, ou em livros
  como~\cite{companion}.} Trocando em miúdos, ele (conteúdo) nem
sempre será posto no exato ponto do texto (fonte) em que aparece.

É frequente que elementos flutuantes apareçam ainda em um índice
próprio (listas de figuras ou tabelas, por exemplo), e que possuam uma
legenda (para que, mesmo extraídos de sua vizinhança textual,
remetam ao assunto de que tratam).

% colocar exemplo usando caption
% enfatizar que o latex sabe onde colocar as imagens
% opções de posicionamento
% todo: como se insere no mecanismo de composição de página

\subsubsection{Figuras}\index{figuras}

Figuras são uma ferramenta poderosa na composição de textos, quando
usadas com parcimônia. É possível colocar imagens no documento dizendo
ao \LaTeX\ sua localização (ou apenas seu nome, se estiverem na mesma
pasta que o documento). Também é possível desenhar usando o próprio
\LaTeX, por exemplo, com o pacote \pacote{Tikz}.

Para colocar figuras em um documento \LaTeX, basta usar o comando
\begin{center}
\verb!\includegraphics{nome-do-arquivo}!
\end{center}
em que a extensão do tipo de arquivo não precisa ser incluída. Mas
atenção: nem toda extensão de imagem é conhecida pelo
\LaTeX\ nativamente, embora baste usar um pacote para superar o
problema, na maior parte dos casos. Acrescente
\verb!\usepackage{graphicx}! no preâmbulo de seu documento e você
poderá incluir imagens \extensao{png}, \extensao{jpg} e
\extensao{pdf}, para citar algumas.

O trecho a seguir inclui a figura~\ref{fig:tikz:piramide-cortada} em
um ambiente flutuador.

\begin{footnotesize}
\begin{verbatim}
\begin{figure}
  \begin{center}
    \input{exemplos-externos/tikz-01}
    \caption{Uma figura gerada com o pacote \pacote{Tikz}.}\label{fig:tronco-piramide}
  \end{center}
\end{figure}
\end{verbatim}
\end{footnotesize}

\begin{figure}
  \begin{center}
    \input{exemplos-externos/tikz-01}
    \caption{Uma figura gerada com o pacote \pacote{Tikz}.}\label{fig:tikz:piramide-cortada}
  \end{center}
\end{figure}

Note que o comando
\verb'\caption'\index{caption@\texttt{\char`\\caption}} é usado
\emph{dentro} do ambiente \ambiente{figure} para acrescentar
uma~\emph{legenda}\index{figura!legenda}\index{figure!legenda}. Repare
ainda que o \verb'label' que segue a legenda permite fazer referência
(seção~\ref{sec:ref-e-label}) 'a figura. Assim, por exemplo ``a
figura~\ref{fig:tikz:layers} foi deslocada
pelo \LaTeX'' é escrito: 
\begin{center}\footnotesize
\texttt{a
  figura\char`\~\char`\\ref\char`\{fig:layers\char`\}\ foi deslocada
  pelo \char`\\LaTeX}
\end{center}
em que ``\verb'fig:layers''' é o rótulo da figura.

\begin{figure}
  \begin{center}
    \input{exemplos-externos/tikz-02}
    \caption{Outra figura gerada com o pacote \pacote{Tikz}.}\label{fig:tikz:layers}
  \end{center}
\end{figure}

\subsubsection{Tabelas}\label{sec:tabelas}\index{tabelas}

Tabelas são um dos maiores desafios de diagramação enfrentados
corriqueiramente por tipógrafos. Cada tabela é única, e uma leve
alteração de espaçamento tem o poder de alterar bastante o destaque de
uma e outra informação.

Comecemos pelo ambiente~\ambiente{tabular}, que é o que permite
escrever as tabelas. Ele é um ambiente que recebe um parâmetro,
indicando algumas propriedades das colunas da tabela, como podemos ver abaixo.

O argumento do ambiente tabular é uma sequência de letras, entre as
quais estão `{\tt c}', `{\tt l}' e `{\tt r}'. Elas indicam que as
o alinhamento do texto nas colunas da tabela. As letras, da esquerda
para a direita, referem-se, respectivamente às colunas, da esquerda
para a direita. A seguir, cada linha da tabela contém o texto de uma
``célula'' da tabela, separada da seguinte por um `\char`\&'. A célula
mais à direita deve ser sucedida por \verb'\\', que indica o fim da
linha da tabela.

Existem outras letras que podem aparecer como parte do parâmetro
de~\ambiente{tabular}. Uma delas é a barra vertical `{\tt |}', que
traça uma barra entre colunas. 

Há ainda comandos que permitem traçar linhas que separam apenas algumas das
células, que fixam a largura de uma célula (para que seja possível
escrever parágrafos de mais de uma linha na tabela). 

Não chegamos a mencionar todas as opções à disposição para a
composição de tabelas. Além desses, é possível estender o conjunto de
comandos disponíveis, incluindo pacotes no preâmbulo do documento. A
tabela~\ref{tab:notas}, a seguir, foi composta usando o
pacote~\pacote{booktabs}. Outro pacote comumente usado é o
\pacote{longtable}, quando estamos tratando de tabelas que se espalham
por mais de uma página.

\medskip
\begin{center}\hrule\smallskip
\begin{tabular}{c|c}
\begin{minipage}{.405\textwidth}\footnotesize
\verbatiminput{exemplos/05-tabela-01}
\end{minipage} &
\begin{minipage}{.535\textwidth}
\begin{center}
\begin{tabular}{clcr|r|}
  a & b & c & d & e\\
  fg & hi & jk & lm & no\\
  \hline
  p & q & r & s & t
\end{tabular}
\end{center}

\end{minipage}
\end{tabular}
\smallskip\hrule
\end{center}
\medskip



%% \medskip
%% \begin{center}\hrule\smallskip
%% \begin{tabular}{c|c}
%% \begin{minipage}{.405\textwidth}\footnotesize
%% \verbatiminput{exemplos/05-02-table-booktabs}
%% \end{minipage} &
%% \begin{minipage}{.535\textwidth}\setlength{\parindent}{1pc}
\begin{table}
\begin{center}
\begin{tabular}{llr}
\toprule
\multicolumn{2}{c}{Nome} \\
\cmidrule(r){1-2}
Nome & Sobrenome & nota \\
\midrule
Pablo & Guerra & $7.5$ \\
César & Bento & $5$ \\
Elias & Ribeiro & $12$ \\
\bottomrule
\end{tabular}
\caption{Nota não reflete o aprendizado.}%
\label{tab:notas}
\end{center}
\end{table}

%% \end{minipage}
%% \end{tabular}
%% \smallskip\hrule
%% \end{center}
%% \medskip

%% \begin{table}\centering
%%   \caption{Gastos {\it \&} despesas / 1º semestre}

%%   \begin{tabular}{crrrrrrr}
%%     mês & jan & fev & mar & abr & mai & jun & total\\
%%     \hline
%%   receita & $10$ & $0$ & $5$ & $20$ & $12$ & $13$ & $60$\\
%%   gastos  & $-3$ & $-4$ & $-3$ & $-3$ & $-5$ & $-3$ & $-21$ \\
%%   \hline
%%   balanço & $7$ & $-4$ & $2$ & $17$ & $7$ & $10$ & $39$\\
%%   \end{tabular}
%% \end{table}
