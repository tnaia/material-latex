\section{Alguns elementos do texto}

\subsection[Listas]{Listas {\it \&} Cia.}

itemise, enumerate, description, quotation

\begin{center}\footnotesize\hrule\smallskip
\begin{tabular}{c|c}
\begin{minipage}{.465\textwidth}
\begin{verbatim}
\begin{itemize}
\item cebola,
\item açafrão, e
\item alho.
\end{itemize}
\end{verbatim}
\end{minipage} &
\begin{minipage}{.465\textwidth}
\begin{itemize}
\item cebola,
\item açafrão, e
\item alho.
\end{itemize}
\end{minipage}
\end{tabular}
\smallskip\hrule
\end{center}


\begin{center}\footnotesize\hrule\smallskip
\begin{tabular}{c|c}
\begin{minipage}{.465\textwidth}
\begin{verbatim}
\begin{description}
\item[cebola] Muito empregada p/ temperar.
\item[açafrão] Também.
\item[alho] Idem.
\end{description}
\end{verbatim}
\end{minipage} &
\begin{minipage}{.465\textwidth}
\begin{description}
\item[cebola] Muito empregada p/ temperar.
\item[açafrão] Também.
\item[alho] Idem.
\end{description}
\end{minipage}
\end{tabular}
\smallskip\hrule
\end{center}

\subsection{Alinhamento}\label{sec:alinhamento}

Boa parte dos textos possui alinhamento justificado, i.e., possui
ambas as margens retas e paralelas. Nem sempre isso é
desejado. Existem muitas maneiras de definir o alinhamento do texto:
falamos de duas delas aqui.

\subsection{Texto não-justificado}

\begin{flushleft}
No ambiente
\ambiente{flushleft}\index{flushleft@\ambiente{flushleft}}, o texto é
``empurrado'' para a esquerda. Os espaços não são nem esticados nem
comprimidos. O efeito resultante são linhas de comprimento variável, o
que por vezes é uma opção interessante de diagramação.
\end{flushleft}

\begin{flushright}
Simetricamente,
\ambiente{flushright}\index{flushright@\ambiente{flushright}} tem o
comportamento esperado, fazendo com que o texto no ambiente em
questão, a partir do parágrafo em que aparece, fique com a
esquerda~\emph{rasgada} --- ou seja, para a direita.
\end{flushright}

\subsection{Elementos flutuantes}\label{sec:floats}

Tipógrafos atentam para uma série de características na disposição do
texto que frequentemente passam despercebidas ao nosso consciente. Uma
delas é o equilíbrio entre o texto que se espalha pelas páginas e os
demais elementos, como figuras e tabelas, que pontuam a paisagem aqui
e ali. 

O \LaTeX\ toma várias precauções na disposição desses elementos,
ditos~\emph{flutuantes} (porque sua posição não é fixa no texto como a
de uma palavra em uma frase). É como se os elementos fossem troncos de
árvore à deriva sobre a correnteza de palavras que compõe o texto.

Figuras e tabelas são somente alguns exemplos de elementos
flutuantes. Eles são ambientes (respectivamente~\ambiente{figure}
e~\ambiente{table}) que encapsulam o conteúdo que irá flutuar.

% colocar exemplo usando caption
% enfatizar que o latex sabe onde colocar as imagens
% opções de posicionamento
% todo: como se insere no mecanismo de composição de página

\subsubsection{Figuras}

Figuras são uma ferramenta poderosa na composição de textos, quando
usadas com parcimônia. É possível colocar imagens no documento dizendo
ao \LaTeX\ sua localização (ou apenas seu nome, se estiverem na mesma
pasta que o documento). Também é possível desenhar usando o próprio
\LaTeX, por exemplo, com o pacote \pacote{Tikz}.

Para colocar figuras em um documento \LaTeX, basta usar o comando
\begin{center}
\verb!\includegraphics{nome-do-arquivo}!
\end{center}
em que a extensão do tipo de arquivo não precisa ser incluída. Mas
atenção: nem toda extensão de imagem é conhecida pelo
\LaTeX\ nativamente, embora basta usar um pacote para superar o
problema, na maior parte dos casos. Acrescente
\verb!\usepackage{graphicx}! no preâmbulo de seu documento e você
poderá incluir imagens \extensao{png}, \extensao{jpg} e
\extensao{pdf}, para citar algumas.

\begin{figure}
  \begin{center}
    \input{exemplos-externos/tikz-01}
    \caption{Uma figura gerada com o pacote \pacote{Tikz}.}\label{fig:tikz:piramide-cortada}
  \end{center}
\end{figure}


\begin{figure}
  \begin{center}
    \input{exemplos-externos/tikz-02}
    \caption{Outra figura gerada com o pacote \pacote{Tikz}.}\label{fig:tikz:layers}
  \end{center}
\end{figure}

\subsubsection{Tabelas}\label{sec:tabelas}

\begin{center}
\begin{tabular}{clcr|r|}
  a & b & c & d & e\\
  f & g & h & i & j\\
  \hline
  k & l & m & n & o
\end{tabular}
\end{center}


\begin{table}\centering
  \caption{Gastos {\it \&} despesas / 1º semestre}

  \begin{tabular}{crrrrrrr}
    mês & jan & fev & mar & abr & mai & jun & total\\
    \hline
  receita & $10$ & $0$ & $5$ & $20$ & $12$ & $13$ & $60$\\
  gastos  & $-3$ & $-4$ & $-3$ & $-3$ & $-5$ & $-3$ & $-21$ \\
  \hline
  balanço & $7$ & $-4$ & $2$ & $17$ & $7$ & $10$ & $39$\\
  \end{tabular}
\end{table}
