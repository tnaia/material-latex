\section{Bibliografia {\it \&} Cia.: Bib\TeX}\label{sec:biblio}

Veremos nesta seção duas abordagens para a composição de
bibliografias. Em uma delas, escrevemos a bibliografia linha por
linha, assim como escrevemos o texto. O \LaTeX\ automaticamente 
numera as entradas da bibliografia, e você pode referenciá-los com o
comando \macroCallWithParameter{cite}{rotulo}.

Outro jeito, muito popular a propósito, de trabalhar com bibliografias,
é usando o programa Bib\TeX. Nessa abordagem, as entradas
bibliográficas são escritas em um arquivo de extensão~\extensao{bib},
seus campos (autor, edição, editora, etc.) são marcados
semanticamente, e a formataçao é deixada a encargo do programa
Bib\TeX\footnote{E pacotes que você porventura acrescente para
  configurar esse conportamento.}.

\subsection{Fazendo no muque}

O mecanismo original de composição de bibliografias pressupõe que elas
estejam postas em um ambiente próprio,
o~\ambiente{thebibliography}. Cada entrada possui opcionalmente um
rótulo público, que aparecerá entre colchetes quando for citada, e
ainda um rótulo interno, que funciona como os rótulos definidos com
\macroCall{label}, podendo ser referenciado usando o
comando~\macroCallWithParameter{cite}{rotulo}. Se nenhum rótulo
público é fornecido, o \LaTeX\ numera as entradas, e coloca ali o
respectivo número.

Outra característica importante deste método é que as referências
aparecem exatamente na ordem em que foram declaradas, como seria de se
esperar. Isso não acontece, veremos, quando se usa o Bib\TeX, que
automatiza a ordenação dos itens da referência.

O processo de compilação do documento se altera quando se acrescenta
bibliografias em um documento, do mesmo modo como acontece quando
usam-se referências internas: o processamento do arquivo gera alguns
arquivos auxiliares, que são usados para escrever as citações.

\begin{verbatim}
\begin{thebibliography}{longuissimo}
\bibitem[Tahan83]{malba-tahan} TAHAN, Malba. \emph{O Homem que
Calculava}. Ed. Círculo do Livro. Edição integral. 1983.

\bibitem[Calvino03]{se-um-viajante} CALVINO, Ítalo. 
\emph{Se um Viajante numa Noite de Inverno}. Ed. Schwarcz. 2003.
\end{thebibliography}
\end{verbatim}

\begin{thebibliography}{longuissimo}
\bibitem[Tahan83]{malba-tahan} TAHAN, Malba. \emph{O Homem que
Calculava}. Ed. Círculo do Livro. Edição integral. 1983.

\bibitem[Calvino03]{se-um-viajante} CALVINO, Ítalo. 
\emph{Se um Viajante numa Noite de Inverno}. Ed. Schwarcz. 2003.
\end{thebibliography}

Vejamos o papel de cada um dos elementos no exemplo. {\tt
  longuissimo} é qualquer texto que tenha tamanho maior (ou igual) ao
rótulo mais longa entrada da bibliografia. Ele é usado pelo
\LaTeX\ para reservar espaço para os rótulos quando ele compõe os
itens da bibliografia.

Tanto {\tt Tahan83} quanto {\tt Calvino03} são rótulos que aparecerão,
por exemplo, quando usar o comando~\macroCallWithParameter{cite}{malba-tahan}.

\subsection{Bib\TeX}

O programa Bib\TeX\ foi criado por Oren Patashnik há mais de vinte
anos. A ideia é preparar um arquivo (geralmente com a extensão
\extensao{bib}), em que estão listadas diversas entradas. Cada entrada
tem um tipo, e uma série de campos com seus respectivos valores. Os
tipos de entradas padrão incluem livros, revistas, propostas de
palestras em conferências (\emph{inproceedings}), artigos
(\emph{article}), parte de livros (\emph{inbook} --- capítulo, seção
etc.), manuais (\emph{manual}), dissertações de mestrado
(\emph{masterthesis}), teses de doutorado (\emph{phdthesis}),
relatórios (\emph{report}), textos que não foram publicados
(\emph{unpublished}) e alguns outros (entre eles miscelâneo ---
\emph{misc} --- quando nada mais servir). Para uma lista de todos os
tipos, nada como olhar a documentação do Bib\TeX\ em seu
sistema.\negthinspace\footnote{Geralmente o arquivo se chama
  \arquivo{btxdoc}.}$^\textrm,$\thinspace\footnote{Também recomendamos
  a páginas~\url{http://www.bibtex.org}
  e~\url{http://data.bibbase.org/}. O formato Bib\TeX\ é bastante
  difundido, e diversos programas possuem expansões para facilitar a
  busca e conversão de entradas de bibliografia usando esse formato.}

Para acrescentar a bibliografia ao documento, podemos usar 
\macroCallWithParameter{bibliography}{arquivo} (\extensao{.bib} não é
necessário), onde se quer que ela seja incluída. 

O comum é que a bibliografia cresça juntamente com o texto. Assim,
eventualmente você escreverá um texto em que citará um determinado
texto. Você acrescenta o \macroCall{cite} correspondente, e depois
acrescenta ao arquivo com as entradas bibliográficas a nova
entrada. Para que ela apareça no documento, será preciso que o
Bib\TeX\ processe o documento, após o que o documento ainda deverá ser
processado como de costume, para que as referências sejam
atualizadas. Para dar conta do recado, processe o documento antes e
ao menos duas vezes depois de processá-lo com o Bib\TeX:
\begin{enumerate}
\item Processe o documento uma vez (\LaTeX e o documento),
\item Bib\TeX e o documento, e
\item processe o documento mais duas vezes (a primeira produz as
  referências corretas e a seguinda as coloca no documento produzido).
\end{enumerate}

O pacote \pacote{babelbib}, usado em conjunto com~\pacote{babel}
facilita a composição de entradas bibliográficas quando as referências
contém itens em diversos idiomas. Problemas com acentuação podem ser
solucionados amiúde com o uso de comandos para acentuação como os
vistos na seção~\ref{sec:ascii}. 

Terminamos a seção com um exemplo simplório de um arquivo com as
entradas bibliográficas para o Bib\TeX. (O comando~\macroCall{LaTeXe}
produz \LaTeXe.)

\begin{verbatim}
@BOOK{malba-tahan,
 author={Malba Tahan},
 title={O Homem que Calculava},
 publusher={Círculo do Livro}
 year={1983},
 note={edição integral}
}

@MANUAL{{lshort,
  author = "Tobias Oetiker and others",
  title = "The Not So Short Introduction to {\LaTeXe}:
           Or {\LaTeXe} in 157 minutes",
  month = june,
  note = "on ctan \url{/info/lshort}",
  year = 2010
}
\end{verbatim}

