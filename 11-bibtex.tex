\section{Bibliografia {\it \&} Cia.: Bib\TeX}\label{sec:biblio}

Veremos nesta seção duas abordagens para a composição de
bibliografias. Em uma delas, escrevemos a bibliografia linha por
linha, assim como escrevemos o texto. O \LaTeX\ automaticamente 
numera as entradas da bibliografia, e você pode referenciá-los com o
comando \macroCall{cite}.

Outro jeito, muito popular a propósito, de trabalhar com bibliografias
é usando o programa Bib\TeX. Nessa abordagem, as entradas
bibliográficas são escritas em um arquivo \extensao{bib}, seus campos
(autor, edição, editora, etc.) são marcados semanticamente, e a
formataçao é deixada a encargo do programa Bib\TeX\footnote{E pacotes
  que você porventura acrescente para configurar esse conportamento.}.

\subsection{Fazendo no muque}

O mecanismo original de composição de bibliografias pressupõe que elas
estejam postas em um ambiente próprio,
o~\ambiente{thebibliography}. Cada entrada possui opcionalmente um
rótulo público, que aparecerá entre colchetes quando for citada, e
ainda um rótulo interno, que funciona como os rótulos definidos com
\macroCall{ref}. Se nenhum rótulo público é fornecido, o
\LaTeX\ numera as entradas, e coloca ali o respectivo número.

Outra característica importante deste método é que as referências
aparecem exatamente na ordem em que foram declaradas, como seria de se
esperar. Isso não acontece, veremos, quando se usa o Bib\TeX, que
automatiza a ordenação dos itens da referência.

O processo de compilação do documento se altera quando se acrescenta
bibliografias em um documento, do mesmo modo como acontece quando
usam-se referências internas: o processamento do arquivo gera alguns
arquivos auxiliares, que são usados para escrever as citações.

\begin{verbatim}
\begin{thebibliography}{longuissimo}
\bibitem[Tahan83]{malba-tahan} TAHAN, Malba. \emph{O Homem que
Calculava}. Ed. Círculo do Livro. Edição integral. 1983.

\bibitem[Calvino03]{se-um-viajante} CALVINO, Ítalo. 
\emph{Se um Viajante numa Noite de Inverno}. Ed. Schwarcz. 2003.
\end{thebibliography}
\end{verbatim}

\begin{thebibliography}{longuissimo}
\bibitem[Tahan83]{malba-tahan} TAHAN, Malba. \emph{O Homem que
Calculava}. Ed. Círculo do Livro. Edição integral. 1983.

\bibitem[Calvino03]{se-um-viajante} CALVINO, Ítalo. 
\emph{Se um Viajante numa Noite de Inverno}. Ed. Schwarcz. 2003.
\end{thebibliography}

Vejamos o papel de cada um dos elementos no exemplo. {\tt
  longuissimo} é qualquer texto que tenha tamanho maior (ou igual) ao
rótulo mais longa entrada da bibliografia. Ele é usado pelo
\LaTeX\ para reservar espaço para os rótulos quando ele compõe os
itens da bibliografia.

Tanto {\tt Tahan83} quanto {\tt Calvino03} são rótulos que aparecerão,
por exemplo, quando usar o comando~\macroCallWithParameter{cite}{malba-tahan}.

\subsection{Bib\TeX}


bibliografia

citando

bibtex
