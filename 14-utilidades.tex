\section{Utilidades}

A partir de agora estás outorgado o título de \LaTeX nico! O que vem
adiante são apenas adendos ao teu cinto de utilidades, mas havendo
dominado o material até aqui, deves estar apto a enfrentar a maior
parte das quiméras tipográficas que o aventureiro compositor
encontrará em uma jornada habitual. No mais, não hesite em convocar a
comunidade, que não se fará surda a qualquer pedido de auxílio.

A lista abaixo contém algumas (poucas!) sugestões de pagotes que você pode achar
interessante investigar. Existem vários pacotes que possuem
finalidades parecidas, quando não idênticas --- fica à sua escolha. A
ideia é que vocÊ conheça um pouco do que dá para fazer com o \LaTeX, a
nova ferramenta no seu cinto de utilidades. Sem mais delongas, a
lista.

\begin{description}
  \item[hyperref] Cria hiperlinks dentro do
    próprio documento, além de controlar seu aspecto. Tem forte
    integração com a estrutura de documentos \extensao{pdf},
    permitindo controlar propriedades como \emph{autor},
    \emph{língua}, etc.
  \item[url] Cria o comando \verb'url', que encapsula páginas na
    internete faz uma quebra ``inteligente''.
  \item[xcolor] Deixa seu texto mais colorido!
  \item[fancyhdr] Personaliza o cabeçalho e rodapé de páginas,
    exibindo, por exemplo, a seção atual, nome do autor ou qualquer
    texto.
  \item[tikz] Desenhe figuras com texto! Visite
    \url{http://www.texample.net/} para ver do que ele é capaz.
  \item[beamer] uma classe de documento para compor slides.
  \item[amsmath] Pacote da \emph{American Mathematical Society} com
    vários comandos para facilitar a composição de expressões
    matemáticas.
  \item[a0poster] Pôsteres em a0!
  \item[microtype] Microtipografia.
  \item[memoir] Uma classe de documento que estende e aprimora
    grandemente as classes documento tradicionais, acrescentando uma
    série de outras categorias.
  \item[multicolumn] Permite usar um número variável de colunas no
    texto.
  \item[indentfirst] Recua a primeira linha do primeiro parágrafo de
    seções.
  \itema[belbib] Traduz palavras da bibliografia, como ``edição'',
  ``ano'', etc.
  \item[helvet] Permite usar a fonte Helvética no texto. 
\end{description}
