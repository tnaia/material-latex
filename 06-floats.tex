\section{Objetos flutuantes}\label{sec:floats}

Tipógrafos atentam para uma série de características na disposição do
texto que frequentemente passam despercebidas ao nosso consciente. Uma
delas é o equilíbrio entre o texto que se espalha pelas páginas e os
demais elementos, como figuras e tabelas, que pontuam a paisagem aqui
e ali. 

O \LaTeX\ toma várias precauções na disposição desses elementos,
ditos~\emph{flutuantes} (porque sua posição não é fixa no texto como a
de uma palavra em uma frase). É como se os elementos fossem troncos de
árvore à deriva sobre a correnteza de palavras que compõe o texto.

Figuras e tabelas são somente alguns exemplos de elementos
flutuantes. Eles são ambientes (respectivamente~\ambiente{figure}
e~\ambiente{table}) que encapsulam o conteúdo que irá flutuar.

% colocar exemplo usando caption
% enfatizar que o latex sabe onde colocar as imagens
% opções de posicionamento
% todo: como se insere no mecanismo de composição de página
