\section{Símbolos}

Diagramação é a disposição de símbolos. E há uma infinidade
deles. Citamos nesta apostila alguns deles, mas certamente não o
suficiente para atender à sua necessidade. Recomendamos fortemente que
mantenha uma cópia do excelente trabalho de Scott Pakin, \emph{The
  Comprehensive  \LaTeX\ Symbol
  List}~\cite{Pakin2008}, que muito provavelmente
já está em alguma parte de sua instalação do sistema\footnote{Em
  algumas instalações o arquivo é chamado \arquivo{symbols-a4.pdf}.},
e que exibe uma lista organizada de aproximadamente cinco mil símbolos
que estão a sua disposição. 

Entre os símbolos disponíveis, estão elementos decorativos, símbolos
fonéticos, matemáticos, de linguagens arcaicas, musicais,
genealógicos, enxadrísticos, químicos, diacríticos incomuns ou
compostos, de diagramas de Feynman, de segurança, de legenda em mapas,
etcétera. Nada inesperado para um sistema que permite escrever em
élfico\ldots

%todo escrever em élfico
