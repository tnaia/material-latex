\section{Mais figuras}

Figuras são uma ferramenta poderosa na composição de textos, quando
usadas com parcimônia. É possível colocar imagens no documento dizendo
ao \LaTeX\ sua localização (ou apenas seu nome, se estiverem na mesma
pasta que o documento). Também é possível desenhar usando o próprio
\LaTeX, por exemplo, com o pacote \pacote{Tikz}.

Para colocar figuras em um documento \LaTeX, basta usar o comando
\begin{center}
\verb!\includegraphics{nome-do-arquivo}!
\end{center}
em que a extensão do tipo de arquivo não precisa ser incluída. Mas
atenção: nem toda extensão de imagem é conhecida pelo
\LaTeX\ nativamente, embora basta usar um pacote para superar o
problema, na maior parte dos casos. Acrescente
\verb!\usepackage{graphicx}! no preâmbulo de seu documento e você
poderá incluir imagens \extensao{png}, \extensao{jpg} e
\extensao{pdf}, para citar algumas.

\begin{figure}
  \begin{center}
    \input{exemplos-externos/tikz-01}
    \caption{Uma figura gerada com o pacote \pacote{Tikz}.}
  \end{center}
\end{figure}


\begin{figure}
  \begin{center}
    \input{exemplos-externos/tikz-02}
    \caption{Outra figura gerada com o pacote \pacote{Tikz}.}
  \end{center}
\end{figure}

