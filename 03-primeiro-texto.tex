\section{Primeiro documento}


\subsection{Texto e sequências de controle}\label{subsec:seq-controle}

Quando você escreve um texto~\LaTeX, a maior parte do tempo você está
escrevendo como se usasse uma máquina de escrever
comum. Eventualmente, no entanto, você desejará acrescentar algo ao
texto além de palavras. Pode ser que queira \emph{enfatizar alguma
  passagem}, ou

\begin{quote}
  ``\ldots \textsl{citar algo que, alguma vez, muito apropriadamente, foi
    dito ou escrito, e que ilustra bem o que quer que seja.''}

  \hfill\textsl{Autor Conhecido}
\end{quote}

Em situações como essas, empregam-se \emph{sequências de controle},
que especificam o papel de alguma palavra, região ou ponto do texto.

Por exemplo, empreguei uma palavra de controle (\emph{control word\/})
pouco acima, para dizer ao \LaTeX\ que ``Texto e sequências de
controle'' é um título de seção. Sabendo disso,  o sistema pode fazer
várias coisas, como
\begin{enumerate}
\item descobrir o número da seção,
\item alterar o tamanho e peso da fonte empregada para escrever as
  palavras do título (com o número da seção ao lado), e
\item acrescentar uma linha ao sumário do texto com o número da página
  em que a seção começa.
\end{enumerate}

Sequências de controle iniciam por uma barra `\verb|\|'. A maior parte
delas, que chamamos \emph{palavras de controle}, são formadas pela
barra seguida por letras\index{letras} (consideramos aqui letras os
caracteres `\texttt{A}' a `\texttt{Z}', e `\texttt{a}' a
`\texttt{z}'). Há um outro tipo
de sequência de controle, que chamaremos aqui de \emph{caractere de controle}
(control character), que consiste de uma barra seguida de um caractere
não-letra, por exemplo `\verb|\-|', e `\verb|\{|' (a função dessas
sequências será explicada nas seções \ref{subsec:hifenacao} e
\ref{sec:matematica}).

Naturalmente, surge a pergunta: mas e se eu quiser usar uma
\textbackslash\ no meu texto? De fato, se você digitar
``\verb|amigo\inimigo|'' para obter amigo\textbackslash inimigo, terá
uma surpresa: muito provavelmente o \LaTeX\ reclamará de uma
\verb!undefined control sequence \inimigo!. Veremos mais adiante que
alguns caracteres são ``reservados'' pelo \LaTeX\ para algumas funções
especiais. Alguns exemplos são os caracteres `\%', `\$' e `\_', além,
claro, do nosso amigo `\verb|\|'. Se você deseja usá-los no seu texto,
será preciso usar alguma sequência de controle que os coloque lá. A
propósito, as sequências de controle necessárias para esses caracteres
em particular são

\begin{center}
  `\verb|\%|' \ para \ `\%'%
  \qquad`\verb|\$|' \ para  \ `\$'%
  \qquad`\verb|\_|\negthinspace' \ para \ `\_'%
  \qquad`\verb|\textbackslash|' \ para \ `\textbackslash'\qquad
\end{center}

\subsection{Um documento simples}

Um texto preparado para o \LaTeX\ em geral é precedido por um
\emph{preâmbulo}, em que geralmente são descritas características do
texto (por exemplo, se ele é uma carta, um livro, um relatório; quem é
o seu autor; se o documento será impresso frente e verso, ou se apenas
uma página por folha.

O trecho abaixo tem três sequências de controle. Vejamos o que
significam. 

\begin{footnotesize}
\begin{verbatim}
\documentclass{article}
\begin{document}

Olá mundo! % Colocar um conteúdo de verdade.

\end{document}
\end{verbatim}
\end{footnotesize}

Primeiro definimos a \emph{classe} do documento, com a
sequência de controle \verb|\documentclass|. Essa sequência requer um
parâmetro, (qual a classe do documento, no caso \verb!article!) que é
posto entre chaves. Teremos mais a falar sobre parâmetros, ou
\emph{argumentos} daqui a pouco.

A classe \verb!article!, define uma série
de coisas, como o tamanho das margens e a formatação de muitos
elementos do texto, p.~ex., a formatação dos números das páginas. Outras
classes comumente usadas incluem \verb!letter!, para cartas,
\verb!beamer! para apresentações de slides, \verb!report! para
relatórios, \verb!book! para livros, \verb!a0poster! para pôsteres em
A0, etc. Há vários outros, como p.~ex., modelos para teses
disponibilizados por universidades, muitos dos quais se pode obter
gratuitamente na internet.

A seguir, demarca-se o início do documento propriamente dito. O par de
sequências de controle \verb!\begin! e \verb!\end! delimita uma
\emph{região} (falaremos mais delas em breve). Aqui, a região é o
próprio documento, seu conteúdo visível. Assim,
\verb!\begin{document}! delimita o início de uma região do tipo
\emph{document}, que é encerrada por \verb!\end{document}!.

Finalmente, o conteúdo do documento: a frase ``Olá mundo!'', seguida
de um \emph{comentário}. Se você é um programador, a noção de
comentário (como aliás muitas outras que abordaremos aqui)  deve
ser-lhe bem familiar. Em nosso exemplo, o comentário é 

\begin{center}
  \it Colocar um conteúdo de verdade.
\end{center}

Comentários iniciam-se por um caractere `\verb.%.', e vão até o fim da
linha. Comentários são anotações no texto que o autor pode fazer para
lembrar-se de alguma coisa, embora possam ter outros usos.

\subsection{Parâmetros}

As sequências de controle (também chamadas aqui de
\emph{comandos}\index{comando}) encontradas,  até agora foram
sempre seguidas de algum texto entre chaves. Em \LaTeX, as chaves
servem para agrupar coisas, para que sejam vistas como uma unidade
só. 

De modo geral, sequências de controle operam de acordo com os
parâmetros, ou argumentos, que passamos para elas. Se uma sequência
emprega um certo número de parâmetros (digamos, 2), ela considera que
eles são os (dois) agrupamentos imediatamente depois dela. Mas
atenção: o \LaTeX\ sempre considera \emph{agrupamento} a menor unidade indivisível
que encontra ao ler um texto! Letras que não estejam  envolvidas em
chaves são, cada uma, um elemento diferente, assim como sequências de
controle o são. Por outro lado, um texto envolvido entre chaves conta
como um único agrupamento, um único elemento.

Por exemplo, suponhamos que haja um comando \verb!\importante! para
destacar texto, que opere sobre um único parâmetro (o texto
importante). O que cada uma das linhas a seguir destaca?

\begin{footnotesize}
\begin{verbatim}
\importante Lembre-se de usar chaves!
\importante{fazer as compras}
\importante{Destacar textos {importantes}}
\end{verbatim}
\end{footnotesize}

Respostas: (Você tentou os exercícios? Vá lá, mais uma chance!)
Respectivamente: ``L''; ``fazer as compras'', e ``Destacar textos {importantes}''.

Comandos nem sempre precisam de argumentos. Por exemplo,
\verb!\newpage! termina a página atual e continua o texto na página
seguinte, e \verb!\maketitle! mostra o título, autor e data do texto.

\subsection{Regiões}

Você já deve ter reparado que há uma certa ``anatomia'' no
texto. Por exemplo, há imagens, citações, tabelas, poemas, listas,
enumerações, e descrições, só para citar alguns. Todos são 
elementos de natureza diferente do texto, tanto visual como
conceitualmente.

Essas regiões, também chamadas de \emph{ambientes}, são trechos do
texto que têm um papel diferente, e, assim, provavelmente demandam um
tratamento diferente.

Já usamos regiões uma vez nesta apostila: o corpo do texto, o
\emph{document}, onde vivem seus elementos visíveis. Neste ponto, você
já deve imaginar como fazer para delimitar um ambiente. Digamos, que
uma parte de nosso relatório seja pura magia. Para que isso seja de
fato incorporado ao texto, basta fazer:

\begin{footnotesize}
\begin{verbatim}
\begin{pura-magia}
Chirrin-chirrion!
\end{pura-magia}
\end{verbatim}
\end{footnotesize}


\subsection{Acentuação: para além do ASCII}\label{subsec:ascii}

Ao experimentar os exemplos dados até agora (se não fez, esta é uma
boa oportunidade! Tente gerar documentos a partir dos exemplos, eu
fico aqui esperando) você deve ter reparado que os caracteres
acentuados não aparecem no documento final. Mas experimente o seguinte
\begin{footnotesize}
\begin{verbatim}
\documentclass{article}
\begin{document}
Ol\'a mundo! Voc\^e come\c cou a notar algo?
\end{document}
\end{verbatim}
\end{footnotesize}

Não desespere. Acentuar é muito mais fácil do que isso. Tentemos outra
coisa
\begin{footnotesize}
\begin{verbatim}
\documentclass{article}
\usepackage[utf8]{inputenc}
\begin{document}
Olá mundo! Você começou a notar algo?
\end{document}
\end{verbatim}
\end{footnotesize}

Qual o resultado? E se você tentar o seguinte?
\begin{footnotesize}
\begin{verbatim}
\documentclass{article}
\usepackage[T1]{fontenc}
\begin{document}
Olá mundo! Você começou a notar algo?
\end{document}
\end{verbatim}
\end{footnotesize}

Uma das alternativas acima deve solucionar a questão dos acentos em
seu computador, a depender de como estão armazendas as letras no
seu texto. Mais precisamente, cada uma das linhas novas, que começam
por \verb'\usepackage', tenta dizer ao \LaTeX\ como interpretar a
\emph{codificação} do arquivo que ele irá processar.

\begin{detalhe}
O leitor atento poderá se perguntar: mas o texto que salvei é
\emph{puro}\footnote{Usamos aqui \emph{texto puro} como tradução da
  expressão em inglês \emph{plain text}: texto sem formatação.}, sem
formatação alguma --- como ele pode ser armazenado de mais de um modo?
quem determina que codificação o arquivo tem? 
A resposta direta a essa pergunta é a seguinte: arquivos são
armazenados como sequências de zeros e uns no computador (ao menos até
este momento, em 2010). A \emph{codificação} de um arquivo é o
conjunto de regras que associa a determinadas sequências de zeros e
uns a cada uma das letras de um texto.
\end{detalhe}

Apesar de os comandos para acentuação serem dispensáveis na maioria
dos casos, há situações em que pode ser útil saber um truque ou
outro. Principalmente quando o que se deseja é escrever algum nome
estrangeiro em algum ponto particular do texto, e não se sabe como
obter o caractere a partir do seu teclado.

O trecho a seguir é um excerto do \TeX book.

\medskip
\begin{center}\hrule\smallskip
\begin{tabular}{c|c}
\begin{minipage}{.405\textwidth}\footnotesize
\verbatiminput{exemplos/03-verbatim-example-03}
\end{minipage} &
\begin{minipage}{.535\textwidth}
Erd\"os, B\=askara, Gabor Szeg\"o.

`\`o' (grave accent)
`\'o' (acute accent)
`\^o' (circumflex or “hat”)
`\"o' (umlaut or dieresis)
`\~o' (tilde or “squiggle”)
`\=o' (macron or “bar”)
`\.o' (dot accent)
`\u o' (breve accent)
`\v o' (há\v cek or “check”)
`\H o' (long Hungarian umlaut)
`\t oo' (tie-after accent)
`\c o' (cedilla)
`\d o' (dot-under accent)
`\b o' (bar-under accent)
`\oe',`\OE' (French ligature OE)
`\ae',`\AE' (Latin and 
             Scandinavian ligature AE)
`\aa,\AA' (Scandinavian A-with-circle)
`\o',`\O' (Scandinavian O-with-slash)
`\l',`\L' (Polish suppressed-L)
`\ss' (German “es-zet” or sharp S)

\end{minipage}
\end{tabular}
\smallskip\hrule
\end{center}
\medskip


Mas o que faz o comando `\verb'\usepackage''? Veremos a seguir.

\subsection{Pacotes}

Uma característica importantíssima do \LaTeX\  é sua
expansibilidade, que permite que ele se adapte às necessidades
dos mais variados usuários. Assim como é possível estender as
capacidades de um programa acrescentando-lhe `plugins', `add-ons', ou,
em mais baixo-nível, bibliotecas, é possível dotar o \LaTeX\ de mais
comandos, pela inclusão de \emph{pacotes}.

Pacotes são documentos de texto (como os que você escreve ao seguir
esta apostila). Certo, eles não são \emph{exatamente} documentos de
texto como os que você escreve agora: os pacotes possuem diversas
definições de comandos, macros e ambientes, que agregam funcionalidade
ao \LaTeX. Pacotes têm muitas vezes a extensão \extensao{sty}, embora
você não precise se preocupar com esse detalhe (ao menos enquanto você
não estiver escrevendo seus próprios pacotes, ou investigando as
fascinantes entranhas do sistema).

Para usar um pacote, basta usar o comando \verb'\usepackage'. Esse
comando faz com que o \LaTeX\ procure pelo arquivo do pacote e torne
sua funcionalidade disponível para que você dela disponha como quiser.
O argumento do comando é o nome do pacote. Pouco atrás usamos um
comando para poder usar acentos em arquivos codificados em \extensao{utf8}.
\begin{center}\footnotesize
\begin{verbatim}
\usepackage[utf8]{inputenc}
\end{verbatim}
\end{center}

Este comando tem um \emph{parâmetro opcional},
\texttt{utf8}. Parâmetros opcionais estão presentes em vários
comandos. Um parâmetro opcional pode ser omitido; ele geralmente
representa alguma configuração ou pequena alteração no modo de
funcionamento do comando.

Assim, é comum que pacotes possam ser configurados por meio de
parâmetros opcionais passados a eles.

Da mesma maneira, classes de documento também podem ser configuradas
por meio da passagem de parâmetros opcionais. Alguns exemplos: pode-se
passar os parâmetros opcionais \texttt{11pt}, \texttt{twocolumn},
\texttt{twoside}, \texttt{draft} para a declaração da classe
\texttt{article}. Assim, para um documento a ser impresso
frente-e-verso, em duas colunas, podemos escrever
\begin{footnotesize}
\begin{verbatim}
\documentclass[twocolumn,twoside]{article}
\begin{document}
...
\end{document}
\end{verbatim}
\end{footnotesize}

É importante notar que separamos os parâmetros opcionais por
vírgulas. Isso acontece para comandos como \texttt{documentclass} e
\texttt{usepackage}, mas não é válido para outros comandos (vide
seção~\ref{sec:comandos}).

Há pacotes para as mais diversas coisas: acrescentar cor ao texto,
usar capitulares (letras grandes, muitas vezes cheias de adornos, no
início de parágrafos), para descrever palavras-cruzadas, jogos de
xadrez, para desenhar, para fazer tabelas grandes, colocar trechos de
texto em números variáveis de colunas, acrescentar marcas d'água,
personalizar cabeçalhos e rodapés, e mesmo
``meta-pacotes.''\footnote{Pacotes que auxiliam a escrita de outros
  pacotes. Esses pacotes geralmente são de um gênero mais técnico,
  parecendo às vezes ``coisa de programador.''}

%\savebox{\mybox}{\protect\verb!\verb!}
\subsection{%
  \texorpdfstring{%
  Verbatim: ambiente \ambiente{verbatim} e \texttt{\char`\\{}verb}}{%
  Verbatim: ambiente verbatim e \textbackslash verb}}

Por vezes o que queremos é que o texto digitado apareça exatamente
como o escrevemos. Veremos a seguir que o \LaTeX\ toma algumas
decisões por conta própria na hora de compor o texto, e os importantes
benefícios que esse comportamento traz consigo. Por hora, mencionemos
um importante exemplo: nem todo espaço no arquivo-fonte corresponderá
a um espaço na formatação final. Calma, as palavras não serão
coladas. Mas experimente usar dois espaços entre um par de palavras. O
que acontece\footnote{Não há resposta aqui \texttt{=)}.}?

Em algumas situações, como por exemplo em listagens de programas, pode
ser útil usar o \LaTeX\ como se ele não fosse mais do que uma máquina
de escrever digital. Queremos que o texto seja posto \emph{verbatim},
isto é, exatamente como foi escrito. Para isso, podemos usar (sic) o
ambiente \verb'verbatim'.

\medskip
\begin{center}\footnotesize\hrule\smallskip
\begin{tabular}{c|c}
\begin{minipage}{.465\textwidth}
\verbatiminput{exemplos/03-verbatim-example-01}
\end{minipage} &
\begin{minipage}{.465\textwidth}
\begin{verbatim}
int main(int argc, char argv) {
  int resposta = 42;
  /* TODO: calcular a pergunta */
  return 0;
}
\end{verbatim}

\end{minipage}
\end{tabular}
\smallskip\hrule
\end{center}
\medskip

Há um outro método para ``cancelar'' a interpretação de caracteres,
para trechos menores, destinados a viver dentro de uma frase
comum. Por exemplo, as várias vezes em que me referi a comandos
\verb'\LaTeX', precisei fazer com que a interpretação do comando fosse
abortada (caso contrário, teria obtido \LaTeX). O comando que faz isso
é o \verb'\verb', que possui uma sintaxe especial. O comando é seguido
por um caractere qualquer (espaço vale!). Esse caractere servirá para
delimitar o fim do argumento de \texttt{\char`\\{}verb}. A esse
caractere se segue o texto a ser ``verbatimizado,'' que é todo o texto
até a próxima ocorrência do delimitador. Exemplo:
`\texttt{\char`\\{}verb!\char`\\LaTeX!}' resulta em `\verb!\LaTeX!',
mas `\verb'\LaTeX'' resulta `\LaTeX'.

Um último comentário. Tanto o comando quanto o ambiente verbatim
possuem uma versão ``estrelada'', que exibe os espaços em branco
\verb*'deste jeito aqui'. O ambiente é chamado \verb'verbatim*' e o comando
\texttt{\char`\\{}verb*}. 

\subsection{Alguma tipografia}

Já dissemos que o \LaTeX\ tem um jeito particular de dispor o texto
que escrevemos. Veremos agora que história é essa.

\medskip
\noindent\begin{minipage}{\textwidth}
\begin{center}\footnotesize\hrule\smallskip
\begin{tabular}{c|c}
\begin{minipage}{.465\textwidth}
\verbatiminput{exemplos/03-verbatim-example-02}
\end{minipage} &
\begin{minipage}{.465\textwidth}
As grandiloquência exibicionista são 
pouco persuasiva para aqueles honestamente 
curioso.
   
Verdade                           isso. 
Para         quem         já tanto 
circunvaga o sentido, cheio de dedos no 
pântano dos significados, um pouco de tento 
com o que passa a ser floreio decorativo é 
no mínimo cortês.

E tudo \LaTeX ado apropriadamente. 
\emph{Muito} apropriadamente.
Usando alguns comandos \LaTeX\ que já foram 
vistos\dots.

\end{minipage}
\end{tabular}
\smallskip\hrule
\end{center}
\end{minipage}
\medskip

Algo que salta à vista de primeira é que as quebras de linha não são
respeitadas. Também parece que os espaços a mais são
desconsiderados\dots e a realidade não está mesmo longe disso: um
espaço ou vários espaços são a mesma coisa para o \LaTeX. Uma (única)
quebra também é equivalente a um espaço. Duas quebras de linha, por
outro lado, fazem com que um novo parágrafo seja iniciado.

Notável também é o fato de que o primeiro
parágrafo\index{paragrafo@parágrafo} não tem recuo, enquanto que os
demais o têm. Isto se deve ao fato de que para algumas culturas (em
particular na tipografia de língua inglesa), não é costumeiro marcar a
primeira linha de um parágrafo com recuo a menos que este seja
precedido imediatamente por outro parágrafo. Afinal, esse recuo tem
por objetivo facilitar a identificação visual do novo parágrafo, o que
não é necessário se o parágrafo é o primeiro de uma seção ou capítulo
do texto, por exemplo.

Encontramos pela primeira vez também os comandos \verb'\LaTeX', que
escreve \LaTeX, e \verb'\emph', que \emph{enfatiza} o texto que lhe é
passado como parâmetro. Note que o que o comando faz é enfatizar: o
jeito como ele faz isso não é a nossa preocupação nesse momento.

\begin{center}
\it O que importa aqui é que o trecho tem que ser destacado.  
\end{center}

E isso é diferente de dizer que o texto deve ser posto em negrito, ser
sublinhado, ser escrito em fúcsia, \reflectbox{ou} \reflectbox{de}
\reflectbox{algum} \reflectbox{jeito} \reflectbox{estranho}. Afinal, o
paradigma aqui é que a aparência do texto refletirá a função, o papel
semântico desempenhado por cada um de seus elementos. Assim,
descreve-se num primeiro momento o que cada um
\emph{significa}\index{marcacao semantica@marcação semântica},
deixando-se para outra etapa (quando pertinente) o ajuste do modo pelo
qual essa função é realçada visualmente.

\LaTeX\ lida com uma granularidade maior de conceitos do que comumente
nos é dado controlar em ambientes usuais de edição de texto; conceitos
que, a princípio, podem surpreender os não iniciados ao universo dos
cuidados tipográficos. A partir de agora, e à medida que adquire
experiência com um sistema tipográfico de alta qualidade como o
\LaTeX, você notará uma série de mudanças na percepção de um texto. Seu
vocabulário vai crescer, seus olhos e atenção serão exercitados em
novas direções, e muito provavelmente você se surpreenderá com a
influência que ``detalhes'' têm no ritmo e facilidade de leitura de um
texto. Mãos à obra!

\subsubsection{Hífens e hifenação}\label{subsec:hifenacao}

Muito embora haja apenas um tipo de hífen em seu teclado, existem
muito mais hífens na tipogravia. Há aquele usado em palavras
compostas, como ``guarda-chuva'' ou ainda ``resguardar-se'', e que também
servem para marcar a quebra de uma palavra no fim de uma linha
(sua \emph{hifenação}); há o traço usado para indicar um intervalo de números,
por exemplo 12--14; há o travessão --- o mais longo entre os hífens; e
há o sinal de menos, usado em equações, como em $20-3=17$. É fácil
produzir cada um desses símbolos em \LaTeX.

\begin{itemize}\footnotesize
\item \verb'guarda-chuva', \verb'resguardar-se'
\item \verb'exercícios das páginas 12--14' 
\item \verb'no dia de hoje --- véspera de amanhã'
\item \verb'diga-me também que $2-2=5$, Winston'
\end{itemize}

O último dos exemplos acima introduz o chamado \emph{modo
  matemático}\index{modo matematico@modo matemático}, assunto da seção
\ref{sec:matematica}.

Mas há ainda o que falar sobre hifenação. Na maior parte dos casos, o
\LaTeX sabe hifenar corretamente as palavras de diversos idiomas (o
portugês entre eles). Para isso basta usar o pacote \pacote{babel},
passando como parâmetro \parametro{brazil}. Algumas vezes, porém,
usamos termos que possuem uma hifenação pouco comum, ou usamos
palavras que o \LaTeX não consegue hifenar a contento. Quando isso
ocorre, podemos dizer explícitamente em que pontos uma palavra pode
ser hifenada. Há dois modos de fazê-lo: pode-se, no preâmbulo,
adicionar um comando \verb'\hyphenation', que leva como parâmetro uma
lista de hifenações, separadas por espaços, como abaixo. Note que não
se podem usar comandos ou caracteres especiais no argumento do comando. 

\begin{center}
  \verb'\hyphenation{FNAC A-bra-cur-six}'
\end{center}

No exemplo acima, FNAC, fnac e Fnac não serão jamais hifenadas, ao
passo que Abracursix e abracursix o serão, segundo os hífens
especificados.

Outro modo é explicar onde uma determinada ocorrência de uma palavra
pode ser hifenada, quando ela ocorre no texto. Nesse caso, a sugestão
de hifenação vale naquele ponto somente. O \LaTeX\ não se lembrará
dela se a palavra for usada novamente.

\medskip
\begin{center}\footnotesize\hrule\smallskip
\begin{tabular}{c|c}
\begin{minipage}{.465\textwidth}
\verbatiminput{exemplos/03-verbatim-example-04}
\end{minipage} &
\begin{minipage}{.465\textwidth}
É algo assim, como direi?
su\-per\-ca\-li\-frag\-i\-lis%
\-tic\-ex\-pi\-a\-li\-do\-cious

\end{minipage}
\end{tabular}
\smallskip\hrule
\end{center}
\medskip

\subsubsection{Caracteres reservados}

São dez os caracteres reservados pelo \LaTeX\ para funções especiais
(ou seja, é preciso alguma ginástica para obtê-los). Eles são os seguintes.

\begin{center}
\verb'\   _   ^   ~   &   #   {   }   %   $'
\end{center}

A barra marca o início de um comando; o ``underscore'' e o circumflexo
são usados no modo matemático (seção~\ref{sec:matematica}); o ``e
comercial'' é usado em tabelas (seção~\ref{sec:tabelas}); o ``jogo da
velha'' é usado na definição de comandos (seção~\ref{sec:comandos});
as chaves agrupam texto; o caractere de porcentagem marca o início de
comentários; e o cifrão delimita o modo matemático.

Esses caracteres podem ser usados em um documento prefixando-os por
uma barra.

\medskip
\begin{center}\footnotesize\hrule\smallskip
\begin{tabular}{c|c}
\begin{minipage}{.465\textwidth}
\verbatiminput{exemplos/03-verbatim-example-05}
\end{minipage} &
\begin{minipage}{.465\textwidth}
\centering \_  \^{}  \~{}   \&  \#   \{   \}   %   $

\end{minipage}
\end{tabular}
\smallskip\hrule
\end{center}
\medskip

A exceção é a barra, que pode ser obtida  por meio do
comando \verb'\textbackslash'.
 
\subsubsection[Ligaduras e Kerning]{Apurando os sentidos: ligaduras, kerning}% e história}

As letras por vezes requerem pequenas modificações no espaçamento
entre si, ou mesmo em sua forma, a depender dos símbolos que estão
próximos de si. Por exemplo, alguns pares de letras são aproximados,
enquanto outras vezes, partes de letras se fundem. Observe os exemplos
abaixo.

\medskip
\noindent\begin{center}%
\scalebox{3}[3]{f{i}}\hfil%
\scalebox{3}[3]{fi}\hfil%
\scalebox{3}[3]{T{a}}\hfil%
\scalebox{3}[3]{Ta}
\\[.9cm] 
\scalebox{3}[3]{s{t}}\hfil%
\scalebox{3}[3]{st}\hfil%
\scalebox{3}[3]{f{l}}\hfil%
\scalebox{3}[3]{fl}
\end{center}
\medskip

Ligaduras (do inglês, \emph{ligatures}), ocorrem quando um agrupamento
de letras é substituído por um outro símbolos, quer para melhorar sua
legibilidade, quer para tornar o texto mais belo.

Já o \emph{kerning} é um aumento ou diminuição do espaço entre letras,
que varia de acordo com o entorno de cada caractere.

\medskip
\noindent\begin{center}%
\scalebox{2}[2]{Uma {T}orta {P}ara {J}aiminho}

\scalebox{2}[2]{Uma Torta Para Jaiminho}

\medskip

%\hfil\scalebox{2}[2]{\textsc{{V}á}}%
%\scalebox{2}[2]{\textsc{Vá}}\hfil%
\scalebox{2}[2]{\textsc{{A}v{a}r{o}}}\hfil%
\scalebox{2}[2]{\textsc{Avaro}}%
\hfil\scalebox{2}[2]{\textsc{{P}a{r}a}}
\hfil\scalebox{2}[2]{\textsc{Para}}\hfil
\end{center}

\begin{comment}
  (breve) história do TeX, do LaTeX e irmãos
  Resumo de como o sistema ``monta'' as páginas
\end{comment}



