\section{Primeiro documento}


\subsection{Texto e sequências de controle}

Quando você escreve um texto~\LaTeX, a maior parte do tempo você está
escvrevendo como se usasse uma máquina de escrever
comum. Eventualmente, no entanto, você desejará acrescentar algo ao
texto além de palavras. Pode ser que queira \emph{enfatizar alguma
  passagem}, ou

\begin{quote}
  ``\ldots \textsl{citar algo que, alguma vez, muito apropriadamente, foi
    dito ou escrito, e que ilustra bem o que quer que seja.''}

  \hfill\textsl{Autor Conhecido}
\end{quote}

Em situações como essas, empregam-se \emph{sequências de controle},
que especificam o papel que alguma palavra, região ou ponto do texto
possui.

Por exemplo, empreguei uma palavra de controle (\emph{control word})
pouco acima, para dizer ao \LaTeX\ que ``Texto e sequências de
controle'' é um título de seção. Sabendo disso,  o sistema pode fazer
várias coisas, como
\begin{enumerate}
\item descobrir o número da seção,
\item alterar o tamanho e peso da fonte empregada para escrever as
  palavras do título (com o número da seção ao lado), e
\item acrescentar uma linha ao sumário do texto com o número da página
  em que a seção começa.
\end{enumerate}

Usualmente, sequências de controle iniciam por uma barra `\verb|\|',
e são seguidas por letras\index{letras} (considermos aqui letras os
caracteres {\tt A} a {\tt Z}, e {\tt a} a {\tt z}). Há um outro tipo
de sequência de controle, que chamaremos aqui de caractere de controle
(control character), que consiste de uma barra seguida de um caractere
não-letra, por exemplo `\verb|\-|', e `\verb|\"|' (a função dessas
sequências será explicada nas seções \ref{indefinida} e
\ref{indefinida}).

Naturalmente, surge a pergunta: mas e se eu quiser usar uma
\textbackslash\ no meu texto? De fato, se você digitar
``\verb|amigo\inimigo|'' para obter amigo\textbackslash inimigo, terá
uma surpresa: muito provavelmente o \LaTeX\ reclamará de uma
\verb!undefined control sequence \inimigo!. Veremos mais adiante que
alguns caracteres são ``reservados'' pelo \LaTeX\ para algumas funções
especiais. Alguns exemplos são os caracteres `\%', `\$' e `\_', além,
claro, do nosso amigo `\verb|\|'. Se você deseja usá-los no seu texto,
será preciso usar alguma sequência de controle que os coloque lá. A
propósito, as sequências de controle necessárias para esses caracteres
em particular são

\begin{center}
  `\verb|\%|' \ para \ `\%'%
  \qquad`\verb|\$|' \ para  \ `\$'%
  \qquad`\verb|\_|\negthinspace' \ para \ `\_'%
  \qquad`\verb|\textbackslash|' \ para \ `\textbackslash'\qquad
\end{center}

\subsection{Um documento simples}

Um texto preparado para o \LaTeX\ em geral é precedido por um
\emph{preâmbulo}, em que geralmente são descritas características do
texto (por exemplo, se ele é uma carta, um livro, um relatório; quem é
o seu autor; se o documento será impresso frente e verso, ou se apenas
uma página por folha.

O trecho abaixo tem três sequências de controle. Vejamos o que
significam. 

\begin{verbatim}
\documentclass{article}
\begin{document}

Olá mundo! % Colocar um conteúdo de verdade.

\end{document}
\end{verbatim}


Primeiro definimos a \emph{classe} do documento, com a
sequência de controle \verb|\documentclass|. Essa sequência requer um
parâmetro, (qual a classe do documento, no caso \verb!article!) que é passado entre
chaves. Teremos mais a falar sobre parâmetros, ou \emph{argumentos}
daqui a pouco.

A classe \verb!article!, define uma série
de coisas, como o tamanho das margens e a formatação de muitos
elementos do texto, p.~ex., a formatação dos números das páginas. Outras
classes comumente usadas incluem \verb!letter!, para cartas,
\verb!beamer! para apresentações de slides, \verb!report! para
relatórios, \verb!book! para livros, \verb!a0poster! para pôsteres em
A0, etc. Há vários outros, como p.~ex., modelos para teses
disponibilizados por universidades, muitos dos quais se pode obter
gratuitamente na internet.

A seguir, demarca-se o início do documento propriamente dito. O par de
sequências de controle \verb!\begin! e \verb!\end! delimita uma
\emph{região} (falaremos mais delas brevemente). Aqui, a região é o
próprio documento, seu conteúdo visível. Assim,
\verb!\begin{document}! delimita o início de uma região do tipo
\emph{document}, que é encerrada por \verb!\end{document}!.

Finalmente, o conteúdo do documento: a frase ``Olá mundo!'', seguida
de um \emph{comentário}. se você é um programador, a noção de
comentário (como aliás muitas outras que abordaremos aqui)  deve
ser-lhe bem familiar. Em nosso exemplo, o comentário é 

\begin{center}
  \it Colocar um conteúdo de verdade.
\end{center}

Comentários iniciam-se por um caractere `\verb.%.', e vão até o fim da
linha. Comentários são anotações no texto que o autor pode fazer para
lembrar-se de alguma coisa, embora possam ter outros usos.

\subsection{Parâmetros}

As sequências de controle (também chamadas aqui de
\emph{comandos}\index{comando}) encontradas, ou  até agora foram
sempre seguidas de algum texto entre chaves. Em \LaTeX, as chaves
servem para agrupar coisas, para que sejam vistas como uma entidade
só. 

De modo geral, sequências de controle operam de acordo com os
parâmetros, ou argumentos, que passamos para elas. Se uma sequência
emprega um certo número de parâmetros (digamos, 2), ela considera que
eles são os (dois) agrupamentos imediatamente depois dela. Mas
atenção: o \LaTeX\ sempre considera agrupamento a menor unidade indivisível
que encontra ao ler um texto! Letras que não estejam  envolvidas em
chaves são, cada uma, um elemento diferente, assim como sequências de
controle o são. Por outro lado, um texto envolvido entre chaves conta
como um único agrupamento, um único elemento.

Por exemplo, suponhamos que haja um comando \verb!\importante! para
destacar texto, que opere sobre um único parâmetro (o texto
importante). O que cada uma das linhas a seguir destaca?

\begin{verbatim}
\importante Lembre-se de usar chaves!
\importante{fazer as compras}
\importante{Destacar textos {importantes}}
\end{verbatim}

Respostas: (Você tentou os exercícios? Vá lá, mais uma chance!)
Respectivamente: ``L''; ``fazer as compras'', e ``Destacar textos {importantes}''.

Comandos em sempre precisam de argumentos. Por exemplo,
\verb!\newpage! termina a página atual e continua o texto na página
seguinte, e \verb!\maketitle! mostra o título, autor e data do texto.

\subsection{Regiões}

Você já deve ter reparado que há uma certa ``anatomia'' no
texto. Por exemplo, há imagens, citações, tabelas, poemas, listas,
enumerações, e descrições, só para citar alguns. Todos são 
elementos de natureza diferente do texto, tanto visual como
conceitualmente.

Essas regiões, também chamadas de \emph{ambientes}, são trechos do
texto que têm um papel diferente, e, assim, provavelmente demandam um
tratamento diferente.

Já usamos regiões uma vez nesta apostila: o corpo do texto, o
\emph{document}, onde vivem seus elementos visíveis. Neste ponto, você
já deve imaginar como fazer para delimitar um ambiente. Digamos, que
uma parte de nosso relatório seja pura magia. Para que isso seja de
fato incorporado ao texto, basta fazer.
\begin{verbatim}
\begin{pura-magia}
Chirrin-chirrion!
\end{pura-magia}
\end{verbatim}

\subsection{Acentuação: para além do ASCII}

\subsection{Verbatim: {\tt \\verb}}
Sequências de controle(\verb!\'! \verb!\"!).

\begin{verbatim}
\documentclass{article}
    
\begin{document}
    
Ol\'a mundo! % Êpa?! Que é que é isso?
    
\end{document}
\end{verbatim}

Pacotes (e parâmetros opcionais).

\begin{verbatim}
\documentclass{article}
    
\usepackage[utf8]{inputenc}
\usepackage[brazil]{babel}
\begin{document}

As grandiloquência exibicionista são pouco persuasiva para aqueles
honestamente curioso.
   
Verdade                           isso. 
Para         quem         ja tanto circunvaga o sentido, cheio de
dedos no pântano dos significados, um pouco de tento com o que
passa a ser floreio decorativo é no mínimo cortês.
\end{document}
\end{verbatim}

Quebra de linha

Hifenando (nova regra, correção pontual --- nomes estrangeiros,
talvez?)

Espaços

Caracteres especiais \verb!\ _ ^ & # { }!

Tipos de hífen

Ligaduras

Kerning

Marcação de conteúdo

(breve) história do TeX, do LaTeX e irmãos

Resumo de como o sistema ``monta'' as páginas

