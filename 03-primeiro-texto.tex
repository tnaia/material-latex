\section{Primeiro documento}


\subsection{Texto e sequências de controle}

Quando você escreve um texto~\LaTeX, a maior parte do tempo você está
escvrevendo como se usasse uma máquina de escrever
comum. Eventualmente, no entanto, você desejará acrescentar algo ao
texto além de palavras. Pode ser que queira \emph{enfatizar alguma
  passagem}, ou

\begin{quote}
  ``\ldots \textsl{citar algo que, alguma vez, muito apropriadamente, foi
    dito ou escrito, e que ilustra bem o que quer que seja.''}

  \hfill\textsl{Autor Conhecido}
\end{quote}

Em situações como essas, empregam-se \emph{sequências de controle},
que especificam o papel que alguma palavra, região ou ponto do texto
possui.

Por exemplo, empreguei uma palavra de controle (\emph{control word})
pouco acima, para dizer ao \LaTeX\ que ``Texto e sequências de
controle'' é um título de seção. Sabendo disso,  o sistema pode fazer
várias coisas, como
\begin{enumerate}
\item descobrir o número da seção,
\item alterar o tamanho e peso da fonte empregada para escrever as
  palavras do título (com o número da seção ao lado), e
\item acrescentar uma linha ao sumário do texto com o número da página
  em que a seção começa.
\end{enumerate}

Usualmente, sequências de controle iniciam por uma barra `\verb|\|',
e são seguidas por letras\index{letras} (considermos aqui letras os
caracteres {\tt A} a {\tt Z}, e {\tt a} a {\tt z}). Há um outro tipo
de sequência de controle, que chamaremos aqui de caractere de controle
(control character), que consiste de uma barra seguida de um caractere
não-letra, por exemplo `\verb|\-|', e `\verb|\"|' (a função dessas
sequências será explicada nas seções \ref{indefinida} e
\ref{indefinida}).

Naturalmente, surge a pergunta: mas e se eu quiser usar uma
\textbackslash\ no meu texto? De fato, se você digitar
``\verb|amigo\inimigo|'' para obter amigo\textbackslash inimigo, terá
uma surpresa: muito provavelmente o \LaTeX\ reclamará de uma
\verb!undefined control sequence \inimigo!. Veremos mais adiante que
alguns caracteres são ``reservados'' pelo \LaTeX\ para algumas funções
especiais. Alguns exemplos são os caracteres `\%', `\$' e `\_', além,
claro, do nosso amigo `\verb|\|'. Se você deseja usá-los no seu texto,
será preciso usar alguma sequência de controle que os coloque lá. A
propósito, as sequências de controle necessárias para esses caracteres
em particular são

\begin{center}
  `\verb|\%|' \ para \ `\%'%
  \qquad`\verb|\$|' \ para  \ `\$'%
  \qquad`\verb|\_|\negthinspace' \ para \ `\_'%
  \qquad`\verb|\textbackslash|' \ para \ `\textbackslash'\qquad
\end{center}

\subsection{Um documento simples}

Um texto preparado para o \LaTeX\ em geral é precedido por um
``preâmbulo'', em que geralmente são descritas características do
texto (por exemplo, se ele é uma carta, um livro, um relatório; quem é
o seu autor; se o documento será impresso frente e verso, ou se apenas
uma página por folha.

\begin{verbatim}
\documentclass{article}
\begin{document}

Olá mundo! % Êpa?! Que é isso?

\end{document}
\end{verbatim}

Vejamos o que cada parte significa. A primeira linha declara a
\emph{classe} do documento. A classe \verb!article!, define uma série
de coisas, como o tamanho das margens e a formatação de muitos
elementos do texto, p.~ex., a posição dos números das páginas. Outras
classes comumente usadas incluem \verb!letter!, para cartas,
\verb!beamer! para apresentações de slides, \verb!report! para
relatórios e \verb!book! para livros, \verb!a0poster! para pôsteres em
A0, etc. Há vários outros: modelos para teses disponibilizados por
universidades, muitos dos quais se pode obter gratuitamente na internet.


\subsection{Parâmetros}
O que são comandos\index{comando} (e parâmetros).

\subsection{Regiões}
Regiões.

\subsection{Acentuação: para além do ASCII}

\subsection{Verbatim: {\tt \\verb}}
Sequências de controle(\verb!\'! \verb!\"!).

\begin{verbatim}
\documentclass{article}
    
\begin{document}
    
Ol\'a mundo! % Êpa?! Que é que é isso?
    
\end{document}
\end{verbatim}

Pacotes (e parâmetros opcionais).

\begin{verbatim}
\documentclass{article}
    
\usepackage[utf8]{inputenc}
\usepackage[brazil]{babel}
\begin{document}

As grandiloquência exibicionista são pouco persuasiva para aqueles
honestamente curioso.
   
Verdade                           isso. 
Para         quem         ja tanto circunvaga o sentido, cheio de
dedos no pântano dos significados, um pouco de tento com o que
passa a ser floreio decorativo é no mínimo cortês.
\end{document}
\end{verbatim}

Quebra de linha

Hifenando (nova regra, correção pontual --- nomes estrangeiros,
talvez?)

Espaços

Caracteres especiais \verb!\ _ ^ & # { }!

Tipos de hífen

Ligaduras

Kerning

Marcação de conteúdo

(breve) história do TeX, do LaTeX e irmãos

Resumo de como o sistema ``monta'' as páginas

