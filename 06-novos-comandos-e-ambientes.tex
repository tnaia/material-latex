\section{Expandindo o \LaTeX}

Há uma grande quantidade de comandos disponíveis ao usuário de
\LaTeX\ (e ainda mais são criados em pacotes novos
continuamente). Embora a maior parte das coisas que se pode querer
fazer em \LaTeX\ já exista na forma de algum comando, não raro podemos
nos valer, com proveito, do poder de \emph{estender o \LaTeX} ---
como elástico que é.
 
Há uma série de razões por que usar comandos é uma boa. Falaremos
delas\ldots\ ao fim desta seção.

\subsection{Criando comandos}\label{sec:comandos}

O mecanismo primário de programação do \TeX\ (e logo o \LaTeX) é a
definição de macros. Se isso soa grego, então ignore. Criar, ou
\emph{definir}, comandos é instruir o sistema sobre o que fazer quando
ele encontra determinada sequência de controle.

Definimos comandos usando o comando \verb'\newcommand'. O tipo mais
simples de comando que há é a mera substituição. É como se
definíssemos um apelido para algo que escrevemos com frequência.

Por exemplo, se estou a compor um estudo sobre a genealogia da família
de feiticeiros do castelo \mbox{Rá-tim-bum}, faz sentido definir os
comandos \verb'\strad', e \verb'\nino'
\newcommand{\stradv}{Stradivarius Victorius}
\newcommand{\nino}{Antonino}
\begin{ttsample}
\index{newcommand@{\tt\char`\\newcommand}}
  \defNewCommand{stradv}{Stradivarius Victorius}\\
  \defNewCommand{nino}{Antonino}
\end{ttsample}
que define \verb'\stradv' como sinônimo para ``{\tt Stradivarius
  Victorius}'', e \verb'\nino' analogamente. Então para escrever
``\nino\ \stradv\ II é o mais novo \stradv'', basta escrever
\begin{ttsample}
\macroCall{nino}\barra\ \macroCall{stradv}\barra\ II
 é o mais
  novo \macroCall{stradv}.
\end{ttsample}


Esse tipo de comando é particularmente útil (para definir siglas
grandes, por exemplo, ou nomes comuns). Mas por vezes não nos
basta. Queremos mais. Queremos que o comando seja parametrizável, ou
seja, que parte dele seja alterável. Fazemos isso dizendo ao
\LaTeX\ que o comando leva parâmetros, passando para
\verb'\newcommand' um parêmetro opcional: o número de argumentos
(parâmetros) que o comando usa. Por exemplo,
\begin{ttsample}
  \defNewCommandWithParameter{estrangeiro}{1}{``\#1''}
\end{ttsample}

Este comando coloca aspas em torno de termos estrangeiros. Note o uso
do caractere reservado `{\tt\#}'. Seguido de um número $n$ entre $1$ e
$9$, ele referencia o $n$-ésimo argumento que foi passado para o
comando. Além disso, observe que o número de parâmetros é posto entre
colchetes, \emph{entre} o nome da macro\footnote{Macro e comando são
    sinônimos aqui.}\index{macro@macro} e a sua definição.

Mais um último exemplo. O que o comando \verb'\formal' abaixo faz?
(\emph{Dica: é possível passar argumentos vazios para um comando.})

\begin{ttsample}
  \defNewCommandWithParameter{formal}{2}{Sr\#1 \#2}\\
\end{ttsample}

\subsection{Modificando comandos}

Só é possível definir um comando uma vez. Se após isso queremos mudar
seu significado, o que fazemos é \emph{redefiní-lo}.
\begin{ttsample}
  \redefCommand{emph}{PRESTA ATENÇÃO AQUI!}
\end{ttsample}

A sintaxe é a mesma da criação usual de comandos. Ao redefinir um
comando, não é preciso continuar usando o mesmo número de parâmetros
que originalmente ele usava. É como definir pela primeira vez um
comando, só que você acrescenta ``{\tt re}'' antes de {\tt
  newcommand}.\footnote{A vantagem de ocorrer um erro se você tenta
  definir um comando que já existe é que desse modo o \LaTeX\ garante
  que, se você está redefinindo um comando, está fazendo-o consciente
  de que está sobrescrevendo um comando existente. Isso é importante
  quando você está manipulando macros que serão usadas por outras
  pessoas, ou por você mesmo daqui a algum tempo.}

\subsection{Criando e modificando ambientes}

Você cria um ambiente usando \verb'\newenvironment'. Um ambiente é um
comando com dois argumentos: o texto que você quer que seja inserido
\emph{antes} do início da região,e o que quer que seja inserido
\emph{depois}. É possível (e simples) definir ambientes que recebam
parâmetros. Deixamos por conta do leitor descobrir como fazê-lo.

\bigskip
\begin{ttsampleflushleft}%
\macroCallWithThreeParameters{newenvironment}{com fofoca}{Você não
    acredita o que\ldots}{\ldots e essa agora!}\\
  \ttbegin{com fofoca}\\
    Casaram!\\
  \ttend{com fofoca}
\end{ttsampleflushleft}

\subsection[Estender pra quê?]%
{Não sou mandão. Por que ficar criando comandos?}
\label{sec:porque-estender}

Terminamos esta seção com alguma discussão sobre o que vimos
aqui. Definir comandos é uma parte importante do uso do \LaTeX. É com
eles que você conseguirá marcar a estrutura e o significado de cada um
dos elementos do seu texto. Comandos não precisam fazer nada (por um
bom tempo) para que sejam úteis. Por exemplo

\begin{ttsample}
  \defNewCommandWithParameter{palavraChave}{1}{\#1}
\end{ttsample}

O importante aqui, a princípio, é que o \emph{significado} seja
explicitado. Futuramente, à medida que você ganhar fluência em
\LaTeX, verá que um texto semanticamente marcado transforma-se
facilmente em uma obra de encher os olhos. Por exemplo, se
pudermos confiar que todas as palavras-chave do texto estão marcadas
com o comando \verb'\palavraChave', e toda palavra estrangeira com
\verb'\estrangeira', será fácil colocar em um parágrafo todas as
palavras-chave do texto automaticamente,\negthinspace\footnote{Isto é,
  de modo que novas anotações sejam agregadas a ele toda vez que o
texto for processado, sem necessidade de interferência manual no
processo} ou fazer um glossário dos termos em outro idioma. A
anotação do significado de um elemento é um ponto de referência ao
qual podemos acrescentar formatação e ações.


Comandos estáticos agilizam a digitação, e são ferramentas importantes
para garantir a uniformidade de nomenclatura no texto. 
Conceitos que ainda estamos concebendo ou batizando, termos de cuja
tradução não estamos certos, siglas e  elementos que desejamos
referenciar em glossários ou índices, todos esses são fortes
candidatos a serem parcial ou totalmente encapsulados por um comando
ou ambiente personalisado. Esse é um dos modos pelos quais o
\LaTeX\ apoia o paradigma \acronimowysiwym.\footnote{Além disso, o
  ponto do texto em que ocorre a
  definição do comando ``encapsulador'' é o lugar mais seguro
  (de fato, é \emph{o}~lugar) para se colocar marcações visuais, como
  \textbf{negrito}, \textsf{sem serifa} ou \textsc{versalete}, já que
  assim as mesmas transformações visuais são assim aplicadas
  consistentemente a todos os elementos com igual marcação semântica.}


% situações em que comandos são importantes: abreviações, semântica,
% dry


% encapsulando formatação em semântica (negrito, small capitals,
% centralizando, sans  typewritter)
