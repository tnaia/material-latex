\section{Aspectos estruturais}

\subsection{A construção da página}
\subsubsection{Anatomia da página}

notas de rodapé, notas marginais, cabeçalho, número de página

\subsection{Parágrafos marginais}
\newcommand{\amounttorotate}{0}\newlength{\recuo}
Usar \marginpar{\ifthenelse{\isodd{\thepage}}{\raggedright}{\raggedleft}\footnotesize notas marginais} notas
marginais no texto pode ser uma maneira interessante de destacar algum
conceito. O comando \verb'\marginpar{parágrafo}' acrescenta um
parágrafo à margem do parágrafo atual. É possível mudar drasticamente
a aparência de um parágrafo lateral (assim como de qualquer outro tipo
de parágrafo): diminuir a fonte em que é escrito, deixá-lo rasgado à
direita ou esquerda (seção~\ref{sec:alinhamento}),
envolvê-lo em uma caixa,
rotacioná-lo%
\marginpar{%
  \ifthenelse{\isodd{\thepage}}%
             {\raggedright\renewcommand{\amounttorotate}{-90}\setlength{\recuo}{-1em}}%
             {\raggedleft\renewcommand{\amounttorotate}{90}\setlength{\recuo}{-1em}}%
  \rotatebox{\amounttorotate}{\hspace{\recuo}\footnotesize\it $\mathcal{A}$ssim.}},
etc. --- em suma, qualquer transformação. Por exemplo, parágrafos de
páginas pares e ímpares são por padrão colocados de modo a que estejam
na lateral da folha que ficaria ``para fora'' caso o texto seja
encadernado. Esse comportamento, para ser mais preciso, depende de
algumas definições na classe do documento.\footnote{Por exemplo, se
  você está usando alguma  classe de documento padrão,
  como~\classedoc{article} ou~\classedoc{book}, a opção
  \parametro{twoside} implica que o documento será impresso
  frente-e-verso, o que geralmente implica que parágrafos marginais
  serão colocados à direita ou esquerda dependendo de a página a que
  pertencem ser par ou ímpar (a opção~\parametro{oneside} faz todo
  paragrafo marginal aparecer no mesmo lado da página).}

\begin{center}\footnotesize\hrule\smallskip
\begin{tabular}{c|c}
\begin{minipage}{.47\textwidth}
\begin{verbatim}
Houve um tempo\footnote{Por volta de 1920.} 
em que as pessoas viviam como se estivessem 
na década de 20.
\end{verbatim}
\vfill
\end{minipage} &
\begin{minipage}{.47\textwidth}
Houve um tempo\footnote{Por volta de 1920.} em
que as pessoas viviam como se estivessem na 
década de 20.
\vspace*{1cm}
\end{minipage}
\end{tabular}
\smallskip\hrule
\end{center}

clearpage, newpage

\subsubsection{Caixas}\label{sec:caixas}

\subsubsection{Medidas}\label{sec:medidas}

comprimentos  (medidas, medindo coisas, criando seus próprios
comprimentos)

espaços

pacote \verb!geometry!, \verb!multicolumn!

\subsubsection{Encaixotando}

\subsection{Arquivos auxiliares}

\subsection{Contadores}\label{sec:contadores}

\subsection{Comandos frágeis}
