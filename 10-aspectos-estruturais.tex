\section{Aspectos estruturais}

Chegou a hora de aproximar-mo-nos um pouco mais do \LaTeX,
conhecer-lhe melhor os trejeitos e manhas, falar mais a sua
língua. Esse conhecimento é valioso quando quisermos convencê-lo a
fazer algo diferente para nós.

%% \subsection{A construção da página}

%% Knuth descreve o processo pelo qual o \TeX\ compõe o texto fazendo uma
%% analogia entre as etapas do algoritmo e a digestão. O sistema teria
%% olhos, boca, estômago, e intestinos (que soltam ao final a nossa obra).
%% Não nos cabe aqui abordar os detalhes sórdidos do que se passa nas
%% entranhas do sistema. Trataremos do grande projeto da página, e de
%% como o \LaTeX\ move informação para cá e para lá. Mas é bom ter em
%% mente de onde viemos, nem que seja para levar o sistema ao médico
%% certo quando alguma coisa entalar.

%% clearpage, newpage

%% \subsubsection{Anatomia da página}

%% notas de rodapé, notas marginais, cabeçalho, número de página

\subsection{Parágrafos marginais}
\newcommand{\amounttorotate}{0}\newlength{\recuo}
Usar \marginpar{\ifthenelse{\isodd{\thepage}}{\raggedright}{\raggedleft}\footnotesize notas marginais} notas
marginais no texto pode ser uma maneira interessante de destacar algum
conceito. O comando \verb'\marginpar{parágrafo}' acrescenta um
parágrafo à margem do parágrafo atual. 

É possível mudar drasticamente
a aparência de um parágrafo lateral (assim como de qualquer outro tipo
de parágrafo): diminuir a fonte em que é escrito, deixá-lo rasgado à
direita ou esquerda (seção~\ref{sec:alinhamento}),
envolvê-lo em uma caixa,
rotacioná-lo%
\marginpar{%
  \ifthenelse{\isodd{\thepage}}%
             {\raggedright\renewcommand{\amounttorotate}{-90}\setlength{\recuo}{-1em}}%
             {\raggedleft\renewcommand{\amounttorotate}{90}\setlength{\recuo}{-1em}}%
  \rotatebox{\amounttorotate}{\hspace{\recuo}\footnotesize\it $\mathcal{A}$ssim.}},
etc. --- em suma, qualquer transformação. Por exemplo, parágrafos de
páginas pares e ímpares são por padrão colocados de modo a que estejam
na lateral da folha que ficaria ``para fora'' caso o texto seja
encadernado. Esse comportamento, para ser mais preciso, depende de
algumas definições na classe do documento.\footnote{Por exemplo, se
  você está usando alguma  classe de documento padrão,
  como~\classedoc{article} ou~\classedoc{book}, a opção
  \parametro{twoside} implica que o documento será impresso
  frente-e-verso, o que geralmente implica que parágrafos marginais
  serão colocados à direita ou esquerda dependendo de a página a que
  pertencem ser par ou ímpar (a opção~\parametro{oneside} faz todo
  paragrafo marginal aparecer no mesmo lado da página).}


\begin{center}\footnotesize\hrule\nopagebreak\smallskip
\begin{tabular}{c|c}
\begin{minipage}{.47\textwidth}
\begin{verbatim}
Houve um tempo\footnote{Por volta de 1920.} 
em que as pessoas viviam como se estivessem 
na década de 20.
\end{verbatim}
\vfill
\end{minipage} &
\begin{minipage}{.47\textwidth}
Houve um tempo\footnote{Por volta de 1920.} em
que as pessoas viviam como se estivessem na 
década de 20.
\vspace*{1cm}
\end{minipage}\nobreak
\end{tabular}%
\nobreak\smallskip\nobreak\hrule
\end{center}


%% \subsubsection{Caixas}\label{sec:caixas}

%% \subsubsection{Medidas}\label{sec:medidas}

%% comprimentos  (medidas, medindo coisas, criando seus próprios
%% comprimentos)

%% espaços

%% \subsubsection{Encaixotando}

\subsection{Arquivos auxiliares}

O \LaTeX\ se vale de um bom número de arquivos auxiliares para
realizar seu trabalho. Tomemos um tempo para observar como funciona o
processo de uso de um arquivo auxiliar.

Algus desses arquivos são produzidos pelo próprio \LaTeX, durante a
compilação do documento. O índice do documento, suas listas de tabelas
e figuras, assim como vários outros, são criados para armazenar as
linhas de índices de elementos de certos tipos. Ao processar um
documento com índice, por exemplo, os números das páginas em que as
seções se iniciam são armazenados para posterior uso. Cada vez que o
texto passa pelo \LaTeX, os números de página mais recentemente
armazenados nos arquivos auxiliares são colocados nos índices.

Um processo parecido acontece com arquivos externos que são gerados
por programas como Bib\TeX\ (seção~\ref{sec:biblio}),
\programa{makeindex} e~\programa{makeglossaries}
(seção~\ref{sec:indice-glossario}). A diferença, então, é que
\emph{após} ser processado pelo \LaTeX, o
texto-fonte\index{texto-fonte} passa ainda por um dos demais, ou
ambos. E, depois, ainda deve ser \LaTeX ado mais duas vezes ao menos,
para que as referências sejam atualizadas.

%% \subsection{Contadores}\label{sec:contadores}

\subsection{Comandos frágeis}

Alguns comandos, como todos nós, precisam por vezes de atenção
especial, é preciso protegê-los. Você pode fazê-lo com
\macroCall{\protect}, que tem como argumento algum texto que precise
ser protegido.

Essa necessidade advém do fato que alguns fazem sentido apenas
se presentes em determinada parte do texto, e, se não protegidos,
podem ser inadvertidamente removidos de seu contexto-natal por outros
comandos.

Exemplos de comandos ``transportadores'' são \macroCall{section}
e~\macroCall{caption}, por exemplo. O texto que é passado como
parâmetro para essas sequências de controle não aparece apenas \emph{in loco},
mas são carregados para índices, listas de figuras, ou mesmo para o
cabeçalho da página.

Mas não entremos em detalhes ainda. O importante por agora é que,
havendo perigo à vista, pode ser necessária proteção.
