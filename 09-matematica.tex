\section{Matemática}\label{sec:matematica}

Ah, a matemática\ldots Ela é em grande parte a razão pela qual temos o
\LaTeX\ (e os computadores!). Aqui, em particular, o \LaTeX\ brilha.

Existem dois ``modos'' principais nos quais o \LaTeX\ pode operar
quando escreve expressões matemáticas: o \emph{modo matemático inline}
e o \emph{modo matemático ``display''}. Ele está no primeiro, em
geral, quando está escrevendo uma fórmula que deverá ocupar um espaço
limitado. No meio de um parágrafo, por exemplo; mas também em índices
ou em frações, como veremos adiante.

% como o texto normal não é mais texto normal em modo matemático

amsmath


$$x^2$$

$$x_2$$

$$2x^2x$$

$$2x^{2x}$$

$$\sqrt{a}$$

$$\sqrt[a]b$$

Com versão inline.

$$\frac ab$$

$$\sum \sum_a^b$$

$$\int\!\!\!\int_{-\infty}^{+\infty}$$

Frações contínuas e displaystyle
$$(\frac ab) \hbox{ versus } \left(\frac ab\right)$$

$$\partial\delta\alpha\beta\gamma\Gamma\epsilon\varepsilon$$

$$a \choose b$$

$$=\sim\neq\leq\geq\in\notin\cap\cup\oplus\cdot\times\div/\equiv\forall\exists\mapsto\Rightarrow\longleftrightarrow
\lim_{x\to b}$$


$$\vec{\nu}+f(\overbrace{a_1a_2\ldots a_n}^{\hbox{$n$ primo}})$$
