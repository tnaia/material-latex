\section{Matemática}\label{sec:matematica}\index{modo matematico@modo matemático}

Ah, a matemática\ldots Ela é em grande parte a razão pela qual temos o
\LaTeX\ (e os computadores!). Aqui, em particular, o \LaTeX\ brilha.


Existem dois ``modos'' principais nos quais o \LaTeX\ pode operar
quando escreve expressões matemáticas: o \emph{modo matemático inline}
e o \emph{modo matemático ``display''}. Ele está no primeiro, em
geral, quando está escrevendo uma fórmula que deverá ocupar um espaço
limitado (no meio de um parágrafo, por exemplo), mas também em índices
ou em frações, como veremos adiante.

É importante perceber que as regras de espaçamento entre letras são
diferentes quando se está trabalhando no modo matemático. As letras
são postas em um tipo itálico, e os espaços são desconsiderados entre
letras; o espaço entre caracteres como $=$, $+$ e $-$ mudam, e
parágrafos (duas quebras de linha consecutivas) não são
permitidos.

Nesta seção faremos uma pequena incursão na composição de fórmulas
usando \LaTeX. Tenha em mente que há uma série de parâmetros que
afetam a legibilidade de uma expressão matemática --- e
mencionaremos apenas algumas delas. Em todos os casos, lance mão de
seu bom-senso, pergunte a opinião de seus amigos, e você não deve ter
problemas. Mãos à obra!

\subsection{Entrando no modo matemático}

Já mencionamos a existência de dois modos matemáticos. Para escrever
uma fórmula em meio a um parágrafo, basta escrevê-la entre um par de
cifrões \verb'$'. A fórmula será então tratada peloa \LaTeX\ como
qualquer outra palavra no parágrafo: \verb'$a + b$' resulta em
$a+b$. A presença de espaços é indiferente no modo
matemático. `\verb'a+b'', `\verb'a +b'', e `\verb'a + b'' são todas formas
equivalentes. Uma grande vantagem disso é que o autor pode formatar a
expressão como melhor lhe convier em termos de legibilidade quando
está a escrever o texto, e o resultado não será ``estragado'' por
isso. Esse é o modo de escrever matemática no meio da linha
(\emph{inline}\index{modo matematico!inline@inline}) . Outro modo é
colocar a fórmula em um banner, com destaque: é o modo de exibição
(\emph{display})\index{modo matematico!display@display}. Para usá-lo,
coloque a fórmula entre `\verb'\['' e `\verb'\]'': a expressão
\verb'\[a\times b = c.\]' faz o \LaTeX\ produzir
\[
a\times b = c.
\]

(O ponto final foi colcado só para terminar a frase, poderia ser
omitido sem maiores consequências que uma frase interminada.)

\subsection[Índices e expoentes]{Para cima e para baixo}

É simples: para gerar uma expressão ``superscrita'' a outra, ($a^b$)
usa-se o comando `\verb'^''. Para ``subscritos'', usamos
'\verb'_''. Observe os exemplos, e note que a necessidade de agrupar
alguns conjuntos de símbolos para obter certos resultados.

\medskip
\begin{center}\hrule\smallskip\footnotesize
\begin{tabular}{c|c}
\begin{minipage}{.405\textwidth}\footnotesize
\verbatiminput{exemplos/09-sub-e-superscripts-math-01}
\end{minipage} &
\begin{minipage}{.535\textwidth}\setlength{\parindent}{1pc}
\[  a^b = c^de  \]

\[  a^b = c^{de}  \]

\[  a_b = \log c \approx f(b)  \]

\[  a^{(c + d)} 
    = \lim_{a \to 0} \gamma^{a\tau}  
\]
\[  \sum_{i=1}^n i =  n(n+1)/2 \]

\end{minipage}
\end{tabular}
\smallskip\hrule
\end{center}
\medskip

Repare que para obter as expressões ``lim'' e ``log'' têm texto
escrito de modo diferente. Para  escrever texto na fonte romana
(i.e., fonte to texto corrente), é preciso sair temporariamente do
modo matemático, para que os espaços voltem a valer. Isso pode ser
feito usando-se uma~\emph{caixa}%%
%%(sobre caixas, vide a seção~\ref{sec:caixas})
. (Preste atenção aos espaços!)

\medskip
\begin{center}\hrule\smallskip
\begin{tabular}{c|c}
\begin{minipage}{.405\textwidth}\footnotesize
\verbatiminput*{exemplos/09-sub-e-superscripts-math-02}
\end{minipage} &
\begin{minipage}{.535\textwidth}\setlength{\parindent}{1pc}
\[ 
\mbox{para todo } a \in A 
\mbox{ há um único } b \in B
\]

\end{minipage}
\end{tabular}
\smallskip\hrule
\end{center}
\medskip

Alguns comandos para digitar expressões matemáticas, como o
\verb'\frac' levam mais de um parâmetro. Outros, como o \verb'\choose'
operam sobre todo o conteúdo do grupo que o contém, usando como
operandos o texto que se encontra à sua esquerda e à sua direita no
grupo.

\medskip
\begin{center}\hrule\smallskip
\begin{tabular}{c|c}
\begin{minipage}{.405\textwidth}\footnotesize
\verbatiminput{exemplos/09-sub-e-superscripts-math-03}
\end{minipage} &
\begin{minipage}{.535\textwidth}\setlength{\parindent}{1pc}
\[ \frac{a}{b} = \sqrt cD = \sqrt{cD}
  = \sqrt[n] p \]

\[ {a \choose b} 
   = {a \choose {b +c \choose c +d}} \]

\end{minipage}
\end{tabular}
\smallskip\hrule
\end{center}
\medskip

Nos exemplos anteriores vimos situações em que parênteses esticaram ou
encolheram, junto com o conteúdo que envolvem. Para obter esse efeito,
usamos os comandos \verb'\left' e~\verb'\right' (sempre aos pares),
seguidos do caractere que se deseja expandir (colchetes, chaves, barra
vertical ou parênteses). O exemplo a seguir já exibe um ou outro
requinte a mais. 

\medskip
\begin{center}\hrule\smallskip
\begin{tabular}{c|c}
\begin{minipage}{.405\textwidth}\footnotesize
\verbatiminput{exemplos/09-sub-e-superscripts-math-04}
\end{minipage} &
\begin{minipage}{.535\textwidth}\setlength{\parindent}{1pc}
\[ 
\left(\int f\right)
\stackrel{\textrm{def}}{=}
\left( \int_0^{+\infty} 
\!\!\!f(x)\,\textrm{d}x \right)
\]


\end{minipage}
\end{tabular}
\smallskip\hrule
\end{center}
\medskip

Primeiro, usamos \verb'\textrm' para que o `d' em d$x$ (e o ``def''
em~$\stackrel{\textrm{def}}=$) seja escrito 
em texto romano. O comando \macroCall{stackrel} \emph{empilha}
pequenos textos, e trata o símbolo resultante como um operador
relacional (maior, menor, menor ou igual e etc.). Usamos ainda os
comandos `\verb'\!' e~`\verb'\,'', correspondentes aos comandos
'\verb'\negthinspace'' e '\verb'\thinspace'' em modo texto,
respectivamente, usados (também respectivamente) para aproximar ou
afastar elementos do texto horizontalmente\footnote{Comumente dizemos
  que esses comandos   \emph{produzem espaço horizontal}:
  \macroCall{negthinspace} produz   espaço fino negativo,
  e~\macroCall{thinspace} produz um espaço fino.} %%
%%(mais sobre espaços na seção~\ref{sec:medidas})
.

Você pode obter letras gregas no modo matemátio facilmente. Vários
outros símbolos estão disponíveis, e a internet é sua amiga para
encontrá-los.

\medskip
\begin{center}\hrule\smallskip
\begin{tabular}{c|c}
\begin{minipage}{.405\textwidth}\footnotesize
\verbatiminput{exemplos/09-sub-e-superscripts-math-05}
\end{minipage} &
\begin{minipage}{.535\textwidth}\setlength{\parindent}{1pc}
\[
\partial\delta\alpha\beta\gamma
 > \Gamma
 < \epsilon
 \geq\varepsilon
\]

\[ \leq\psi=\sim\neq\leq\geq\in\notin
\cap\cup\oplus\cdot\times\div/\equiv \]

\[ \forall\exists\mapsto\Rightarrow
\longleftrightarrow\nu \]

\end{minipage}
\end{tabular}
\smallskip\hrule
\end{center}
\medskip

E isto é apenas parte do que se pode fazer com o \LaTeX, apenas
tocamos a superfície. Uma boa
referência é~\cite{graetzer00}.

% $$\vec{\nu}+f\overbrace{(a_1a_2\ldots a_n)}^{\hbox{$n$ primo}}$$
% ensuremath
