\section*{Sobre}

Este documento nasce como um material de apoio a oficinas de
\LaTeX. As oficinas são oferecidas pelo Grupo de Estudos de Software
Livre da Escola Politécnica da Universidade de São Paulo, o PoliGNU.
Contamos com o apoio do Instituto de Matemática e Estatística da
Universidade de São Paulo (IME).

\subsection*{Como usar esta apostila}

Como você bem entender. Este texto está licenciado sob a \emph{GNU Free
Documentation License} --- uma cópia está anexa ao fim deste documento).
Resumidamente, tens o direito de distribuir cópias deste documento,
modificado ou não, com a condição de mantê-lo sob a mesma licença.

\begin{detalhe}
Parágrafos que estejam com esta marcação contêm detalhes que talvez
sejam prescindíveis em uma primeira leitura. Falam de assuntos
marginais ao uso do \LaTeX, ou de tópicos que requerem alguma
\TeX nica (i.e., podem empregar conceitos que não são abordados até
um ponto mais adiantado do texto).
\end{detalhe}

\clearpage
