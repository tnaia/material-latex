\section{Estruturando o texto}

Textos, assim como animais, possuem uma anatomia. Essa anatomia é o que permite ao leitor se localizar em sua leitura, identificar algo que procura. A estrutura do texto, além disso, carrega uma mensagem em si, ao menos em potencial, ao refletir o encadeamento do texto.

A depender da classe do documento, há uma certa variedade de tipos de segmentações à nossa disposição para organizar o texto.
Artigos podem ser particionados em seções, subseções, subsubseções, apêndices.
Livros possuem, adicionalmente ao que está disponível em artigos, capítulos (contendo um certo número de seções).
Relatórios possuem adicionalmente (a livros) \emph{partes}, (que contém capítulos).
E por aí vai.

Você pode mesmo criar seu próprio nível hierárquico, como parágrafos, como veremos na seção~\ref{sec:contadores}.

Neste capítulo, abordaremos, a título de exemplo, secionamento (segmentação) de um texto em artigos (documentos da classe~\pacote{article}). O comportamento apresentado em livros, relatórios etcétera é análogo, e em caso de dúvida basta recorrer aos manuais da respectiva classe (que, por padrão, vêm juntamente com o pacote quando a sua distribuição \LaTeX\ é instalada).

O exemplo a seguir ilustra o uso de seções, subseções, subsubseções, seções não numeradas\index{secao@seção} e apêndices.


\medskip
{\footnotesize\verbatiminput{exemplos/05-01-sectioning}}

\medskip

Dizemos que uma seção inicia a partir do comando \verb'\section'. O
argumento que este comando leva é o título da seção. O mesmo acontece
para sub-seções, e as demais divisões do texto.

Seções são numeradas por padrão. Para obter uma seção, sub-seção, etc. não numerada,
use o respectivo comando em sua versão com asterisco, por
exemplo \verb'\section*{Prefácio}'.

A classe de documento e os pacotes que você usa definem quais os
comandos de secionamento disponíveis. Livros, por exemplo, têm
\verb'\chapter', relatórios têm \verb'\part', e por aí vai.

Falemos agora do acompanhamento natural de um texto secionado:
sumários (ou índices). Fazer um sumário\index{sumario@sumário}, com o
\LaTeX\ é muito simples. Marque os títulos das partes usando os comandos de
secionamento que acabamos de ver, e, no ponto do texto em que deseja
acrescentar o índice, coloque o comando \verb'\tableofcontents'.

Uma vez marcadas as seções do texto e solicitado o índice, o
\LaTeX\ anota (em um arquivo auxiliar) as páginas em que começam as
seções do texto, à medida que o processa. Essas informações
são usadas para escrever o sumário. A depender da parte do
texto na qual sumário foi posto, pode ser necessário processar o texto duas
vezes (na primeira as páginas em que ocorrem as seções são anotadas, e
na segunda as entradas no sumário são atualizadas com os valores
corretos).

Além das informações escritas no arquivo auxiliar (que tem a
extensão~\extensao{aux}), o comando \verb'\tableofcontents' faz ser
gerado um outro arquivo, com a extensão \extensao{toc} (\emph{table of
  contents}), que contém o sumário em si. 

É possível --- e igualmente fácil --- gerar listas de figuras, tabelas
ou quaisquer outros elementos usando \LaTeX. Veremos como fazê-lo nas
seções~\ref{sec:floats} e~\ref{sec:contadores}.\footnote{Para os
  curiosos, a receita: basta usar  \texttt{\char`\\{}listoffigures}
  e~\texttt{\char`\\{}listoftables} em conjunto com os ambientes
  \ambiente{figure} e \ambiente{table}.}


Os comandos de secionamento possuem em geral um parâmetro opcional,
que é uma versão ``mais compacta'' do título, para ser usada em no
sumário (ou, por vezes, no cabeçalho ou rodapé de páginas).
\begin{footnotesize}
\begin{verbatim}
\section[Prova Documental]{%
  Documento provando a corretude do argumento %
  que tive em uma longa insônia alcoolizada}
\end{verbatim}
\end{footnotesize}


