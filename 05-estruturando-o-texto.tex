\section{Estruturando o texto}

Textos, assim como animais, possuem uma anatomia. Essa anatomia é o que permite ao leitor se localizar em sua leitura, identificar algo que procura. A estrutura do texto, além disso, carrega uma mensagem em si, ao menos em potencial, ao refletir o encadeamento do texto.

A depender da classe do documento, há uma certa variedade de tipos de segmentações à nossa disposição para organizar o texto.
Artigos podem ser particionados em seções, subseções, subsubseções, apêndices.
Livros possuem, adicionalmente ao que está disponível em artigos, capítulos (contendo um certo número de seções).
Relatórios possuem adicionalmente (a livros) \emph{partes}, (que contém capítulos).
E por aí vai.

Você pode mesmo criar seu próprio nível hierárquico, como parágrafos, como veremos na seção~\ref{sec:contadores}.

Neste capítulo, abordaremos, a título de exemplo, secionamento (segmentação) de um texto em artigos (documentos da classe~\pacote{article}). O comportamento apresentado em livros, relatórios etcétera é análogo, e em caso de dúvida basta recorrer aos manuais da respectiva classe (que, por padrão, vêm juntamente com o pacote quando a sua distribuição \LaTeX\ é instalada).

O exemplo a seguir ilustra o uso de seções, subseções, subsubseções, seções não numeradas\index{secao@seção} e apêndices.

%\clearpage
\medskip
%\begin{center}\hrule\smallskip
%\begin{tabular}{c|c}
%\begin{minipage}{.405\textwidth}
{\footnotesize\verbatiminput{exemplos/05-01-sectioning}}
%\end{minipage}% &
%\begin{minipage}{.535\textwidth}
%\section*{Preface}
% texto...
\section{Introdução}
% texto...
\subsection{Contexto histórico}
% texto...
\subsection{Proposta investigativa}
% texto...
\section{Argumentação fantástica}
% texto...
\section{Conclusão bombástica}
% texto...
\appendix
% A partir daqui os capítulos são ``numerados''
% com letras em vez de números
\section{Prova incontestável}

%\end{minipage}
%\end{tabular}
%\smallskip\hrule
%\end{center}
\medskip

parte, capítulo, seção, subseção subsubseção, tableofcontents, seções
são nmeradas, parâ\-metro opcional de seção


\section{Document proving the insight I had on a
drunk night}
