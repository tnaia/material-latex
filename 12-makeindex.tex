\section{Índices remissivos {\it \&} Cia.}\label{sec:indice-glossario}

Glossários e índices remissivos são elementos delicados, por sua
natureza distribuída. São como que amarrações de palavras, uma teia
que se espalha por toda a tessitura do documento, associando
conceitos. Veremos agora como usar esses elementos no \LaTeX.

Aqui arquivos auxiliares continuam desempenhando um papel
importante. Mantemos um fluxo de trabalho análogo ao da
preparação de bibliografias (seção~\ref{sec:biblio}). Acrescentamos ao
documento diretrizes declarando a existência do elemento em questão, e
marcamos ainda a as suas manifestações (expressões, indexadas ou
definidas num glossário). Glossários demandam uma preparação a mais: é
preciso escrever as definições dos termos, ou declarar o arquivo
externo em que estão contidas. Tendo feito isso, o procedimento é
padrão:
\begin{enumerate}
\item \LaTeX ar o documento,
\item processar o documento para gerar glossário ou índice remissivo,
\item \LaTeX ar o documento mais duas vezes, para atualizar os índices
  e referências de páginas.
\end{enumerate}

\subsection{Índices remissivos}

O primeiro passo é acrescentar um pacote que permita gerar índices,
como por exemplo~\pacote{makeidx} (que usamos aqui). Ele funciona em
conjunto com o programa~\programa{makeindex}, que processa o documento
em busca dos índices, anotando as páginas em que ocorrem. Alem disso,
é preciso colocar o comando \macroCall{makeindex}, caso contrário o
índice não será produzido (é um jeito fácil de ``desligar'' o índice,
quando oportuno).

O segundo passo é criar as entradas do índice. O que é uma entrada do
índice? É uma referência para um ponto (em geral sua página) do texto,
associado a uma palavra. Como índices podem ser longos, é comum que
sejam ordenados alfabeticamente.
Para associar a um termo uma entrada no índice existe o comando
\macroCallWithParameter{index}{nome no
  índice}\index{index@\macroCall{index}}, que faz com que, no índice
remissivo, seja criada uma entrada apontando para a página corrente,
rotulado com ``{\tt nome no índice}''. 

Acontece que \programa{makeindex} nem sempre tem pela frente uma
tarefa fácil quando se propõe a organizar o índice em ordem
alfabética. Por exemplo, como você ordenaria ``néctar'', ``nervo'',
``\macroCall{newcommand}'', e ``$\pi$''? Para isso existe um modo
alternativo de declarar uma entrada no índice, usando a
seguinte sintaxe.

\begin{center}
\macroCallWithParameter{index}{versao ascii para ordenacao@versão
  \barra com\wrapinbraces{firulas}}\index{versao ascii para
  ordenacao@versão \textbackslash com\{firulas\}}
\end{center}

(Não, \macroCall{com} não é um comando.)

Uma última característica: algumas vezes queremos expressar uma
hierarquia de conceitos no índice. Isso também é possível (os
conceitos-subordinados aparecem todos organizados ``dentro'' da
entrada do elemento hierarquicamente superior, no índice). Como chegar
a esse efeito almejadíssimo é algo a respeito de que nos preservamos o
direito de calar. Tire a caneta da orelha e tome cá, esta é a tua
pulga. Poderá tirá-la daí com uma olhadela na documentação do
pacote~\pacote{makeidx}.

A terceira e última etapa antes que se passe ao processamento do texto
é definir em que ponto daquele estará o índice. É nesse exato lugar
que deve deixar o comando

\begin{center}
  \macroCall{printindex}
\end{center}

Agora é a dança de processamento de sempre, mudando os parezinhos: uma
com~\LaTeX, outra com~\programa, e mais duas com~\LaTeX\ para terminar
bem a farra.

\subsection{Glossários}

