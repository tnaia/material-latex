\documentclass[%draft, 
a4paper,11pt,twoside]{article}

\usepackage[brazil]{babel}
\usepackage[utf8]{inputenc}
\usepackage[pdfpagelabels,colorlinks]{hyperref}
\usepackage{a4wide}
\usepackage{indentfirst}
%\usepackage[xindy]{glossaries}
%\makeglossaries


%\usepackage{showidx}
\usepackage{makeidx}
\makeindex
\newcommand{\extensao}[1]{\texttt{#1}}

\title{Mais uma apostila de \LaTeX}
\author{Tássio Naia dos Santos} % por enquanto


\title{Mais uma apostila de \LaTeX}
\author{Tássio Naia dos Santos} % por enquanto

% pdf meta-informação
\hypersetup{baseurl={http://www.ime.usp.br/~tassio/apostila.pdf},
  pdftitle={Mais uma apostila de LaTeX},
  pdfauthor={Tássio Naia dos Santos},
  pdfkeywords={LaTeX,TeX,tipografia,apostila,CCSL,PoliGNU},
  pdflang={pt-BR (Portuguese)},
  unicode=true}




\begin{document}
\maketitle
\thispagestyle{empty}
\clearpage
\section*{Sobre}

Este documento nasce como um material de apoio a oficinas de
\LaTeX. As referidas oficinas são oferecidas pelo Grupo de Estudos de
Software Livre da Escola Politécnica da Universidade de São Paulo, o
PoliGNU. Contamos com o apoio do Centro de Competência em Software
Livre (CCSL) do Instituto de Matemática e Estatística da Universidade
de São Paulo (IME).

Este texto está licenciado sob a \emph{GNU Free
Documentation License} (uma cópia está anexa ao fim deste documento).
Resumidamente, tens o direito de distribuir cópias deste documento,
modificado ou não, com a condição de mantê-lo sob a mesma licença.
O texto é hoje um projeto hospedado no GitHub\copyright. Todos os
arquivos empregados para produzir o texto (exceto as referências)
podem ser baixados em \url{https://github.com/tnaia/material-latex/}.

\vspace*{\fill}

Viemos a saber que, apesar de todo o cuidado, os filhos de uma gafe e
um senhor descuido perderam-se por estas páginas, enquanto o café era
servido. Caso encontre algum desses erros, errando por aí afora,
pedimos o favor de identificá-lo, para que possamos apaziguar uma mãe
aflita. 
Nosso contato é \href{mailto:poli@gnu.org}{\nolinkurl{poli@gnu.org}}.


\begin{flushright}
Grato!\\
o autor
\end{flushright}
\vspace*{\fill}
\clearpage


\clearpage
\vspace*{.33\textheight}
\thispagestyle{empty}
\begin{flushright}
a José Augusto
\end{flushright}
\clearpage
%\begin{footnotesize}
\tableofcontents
%\end{footnotesize}

\clearpage
\section{Introdução}



\subsection{A metáfora}

\LaTeX\ se apoia fortemente numa certa relação entre apresentação e
conteúdo do texto, cujo
conhecimento pode poupar-nos (ou melhor, auxiliar-nos a lidar com)
algumas dores de cabeça: é o princípio de que o \emph{ritmo visual} de
um texto deve enfatizar sua estrutura. Por exemplo: a formatação
consistente de títulos de seções, destacando em que ponto se iniciam,
realça a coordenação entre os trechos que compõem o documento.

Esse pressuposto é válido para a vasta maioria dos escritos, em
particular livros convencionais, publicações de caráter técnico,
tais como relatórios, monografias,  relatórios, cartas, etc. A fatia
deixada de fora abarca produções caracterizadas por alguma
inconstância, defasagem intencional ou arritmia entre o conteúdo e a
formatação --- como trabalhos artísticos.

\subsection{Sinopse da Ópera}
% TeX e LaTeX (Knuth, Lamport, comunidade, uso)

Donald E.~Knuth criou \TeX, um sistema de tipografia digital muito~(!!)
poderoso, e extremamente flexível.

\begin{center}
\begin{minipage}{.75\textwidth}
  [\TeX\ is] \textit{a new typesetting system intended for the creation of
  beautiful books---and especially for books that contain a lot of
  mathematics.}

  \hfill Knuth---The \TeX book
\end{minipage}
\end{center}

  
Leslie Lamport criou o \LaTeX, que, a grosso modo, é uma interface
mais simplista para o uso do \TeX. Uma preocupação do \LaTeX\ é que,
ao usá-lo, tenhamos foco no conteúdo, na estrutura do que estamos a
compor. Busca separar as etapas de composição conceitual e visual do
texto.

Em contraposição ao modelo de edição de texto dos programas mais
populares hoje, em que 
\emph{o que você vê é o que você obtém}%
\footnote{Conhecido pela sigla em inglês \acronimo{wysiwyg}\index{wysiwyg@\acronimo{wysiwyg}},
  \emph{what you see is what you get}.} (ao menos deveria ser), ao usar
\LaTeX\ \emph{o que você vê é o que você quis dizer}%
\footnote{Do inglês: \emph{what you see is what you mean} (\acronimo{wyhiwym})\index{wysiwym@\acronimo{wysiwym}}.}.



\subsection{O que dá pra fazer}

Compor textos belíssimos. (E, por que não? Compor textos horrorosos.) Na
prática, veremos em breve, é simples produzir
documentos~\extensao{pdf}, \extensao{ps}, e~\extensao{dvi}; documentos
com diagramas (que podem ser desenhados usando o próprio sistema, ou
importando imagens~\extensao{jpg}, \extensao{eps}, \extensao{pdf},
etc.), tabelas, versos, referências bibliográficas, índices,
hiperlinks, e muitas outras coisas.

% colocar uma tabela com células (uma imagem importada, uma desenhada com o tikx, e texto)

\subsection{Antes de por a mão na massa\ldots}

\begin{detalhe}
Parágrafos que estejam com esta marcação contêm detalhes que talvez
sejam prescindíveis em uma primeira leitura. Falam de assuntos
marginais ao uso do \LaTeX, ou de tópicos que requerem alguma
\TeX nica (i.e., podem empregar conceitos que não são abordados até
um ponto mais adiantado do texto).
\end{detalhe}

\section{Rotina de trabalho}

Escrever um documento usando \LaTeX, não é muito diferente de escrever
um documento numa máquina de escrever, embora o resultado seja
bastante diferente. Em geral, você irá abrir um programa para edição
de texto%
\footnote{%
  Existem mesmo alguns programas sofisticadíssimos
  para a edição de documentos \LaTeX, mas este não é nosso foco
  aqui.}% todo: citar exemplos de TeXmakers da vida etc..
, escreverá o texto, e pedirá ao \LaTeX\ que gere o
documento \extensao{pdf} (ou~\extensao{ps}, ou~\extensao{dvi} )que
desejar. Simples assim.


\section{Primeiro documento}


\subsection{Texto e sequências de controle}\label{subsec:seq-controle}

Quando você escreve um texto~\LaTeX, a maior parte do tempo você está
escrevendo como se usasse uma máquina de escrever
comum. Eventualmente, no entanto, você desejará acrescentar algo ao
texto além de palavras. Pode ser que queira \emph{enfatizar alguma
  passagem}, ou

\begin{quote}
  ``\ldots \textsl{citar algo que, alguma vez, muito apropriadamente, foi
    dito ou escrito, e que ilustra bem o que quer que seja.''}

  \hfill\textsl{Autor Conhecido}
\end{quote}

Em situações como essas, empregam-se \emph{sequências de controle},
que especificam o papel de alguma palavra, região ou ponto do texto.

Por exemplo, empreguei uma palavra de controle (\emph{control word\/})
pouco acima, para dizer ao \LaTeX\ que ``Texto e sequências de
controle'' é um título de seção. Sabendo disso,  o sistema pode fazer
várias coisas, como
\begin{enumerate}
\item descobrir o número da seção,
\item alterar o tamanho e peso da fonte empregada para escrever as
  palavras do título (com o número da seção ao lado), e
\item acrescentar uma linha ao sumário do texto com o número da página
  em que a seção começa.
\end{enumerate}

Sequências de controle iniciam por uma barra `\verb|\|'. A maior parte
delas, que chamamos \emph{palavras de controle}, são formadas pela
barra seguida por letras\index{letras} (consideramos aqui letras os
caracteres `\texttt{A}' a `\texttt{Z}', e `\texttt{a}' a
`\texttt{z}'). Há um outro tipo
de sequência de controle, que chamaremos aqui de \emph{caractere de controle}
(control character), que consiste de uma barra seguida de um caractere
não-letra, por exemplo `\verb|\-|', e `\verb|\{|' (a função dessas
sequências será explicada nas seções \ref{subsec:hifenacao} e
\ref{sec:matematica}).

Naturalmente, surge a pergunta: mas e se eu quiser usar uma
\textbackslash\ no meu texto? De fato, se você digitar
``\verb|amigo\inimigo|'' para obter amigo\textbackslash inimigo, terá
uma surpresa: muito provavelmente o \LaTeX\ reclamará de uma
\verb!undefined control sequence \inimigo!. Veremos mais adiante que
alguns caracteres são ``reservados'' pelo \LaTeX\ para algumas funções
especiais. Alguns exemplos são os caracteres `\%', `\$' e `\_', além,
claro, do nosso amigo `\verb|\|'. Se você deseja usá-los no seu texto,
será preciso usar alguma sequência de controle que os coloque lá. A
propósito, as sequências de controle necessárias para esses caracteres
em particular são

\begin{center}
  `\verb|\%|' \ para \ `\%'%
  \qquad`\verb|\$|' \ para  \ `\$'%
  \qquad`\verb|\_|\negthinspace' \ para \ `\_'%
  \qquad`\verb|\textbackslash|' \ para \ `\textbackslash'\qquad
\end{center}

\subsection{Um documento simples}

Um texto preparado para o \LaTeX\ em geral é precedido por um
\emph{preâmbulo}, em que geralmente são descritas características do
texto (por exemplo, se ele é uma carta, um livro, um relatório; quem é
o seu autor; se o documento será impresso frente e verso, ou se apenas
uma página por folha.

O trecho abaixo tem três sequências de controle. Vejamos o que
significam. 

\begin{footnotesize}
\begin{verbatim}
\documentclass{article}
\begin{document}

Olá mundo! % Colocar um conteúdo de verdade.

\end{document}
\end{verbatim}
\end{footnotesize}

Primeiro definimos a \emph{classe} do documento, com a
sequência de controle \verb|\documentclass|. Essa sequência requer um
parâmetro, (qual a classe do documento, no caso \verb!article!) que é
posto entre chaves. Teremos mais a falar sobre parâmetros, ou
\emph{argumentos} daqui a pouco.

A classe \verb!article!, define uma série
de coisas, como o tamanho das margens e a formatação de muitos
elementos do texto, p.~ex., a formatação dos números das páginas. Outras
classes comumente usadas incluem \verb!letter!, para cartas,
\verb!beamer! para apresentações de slides, \verb!report! para
relatórios, \verb!book! para livros, \verb!a0poster! para pôsteres em
A0, etc. Há vários outros, como p.~ex., modelos para teses
disponibilizados por universidades, muitos dos quais se pode obter
gratuitamente na internet.

A seguir, demarca-se o início do documento propriamente dito. O par de
sequências de controle \verb!\begin! e \verb!\end! delimita uma
\emph{região} (falaremos mais delas em breve). Aqui, a região é o
próprio documento, seu conteúdo visível. Assim,
\verb!\begin{document}! delimita o início de uma região do tipo
\emph{document}, que é encerrada por \verb!\end{document}!.

Finalmente, o conteúdo do documento: a frase ``Olá mundo!'', seguida
de um \emph{comentário}. Se você é um programador, a noção de
comentário (como aliás muitas outras que abordaremos aqui)  deve
ser-lhe bem familiar. Em nosso exemplo, o comentário é 

\begin{center}
  \it Colocar um conteúdo de verdade.
\end{center}

Comentários iniciam-se por um caractere `\verb.%.', e vão até o fim da
linha. Comentários são anotações no texto que o autor pode fazer para
lembrar-se de alguma coisa, embora possam ter outros usos.

\subsection{Parâmetros}

As sequências de controle (também chamadas aqui de
\emph{comandos}\index{comando}) encontradas,  até agora foram
sempre seguidas de algum texto entre chaves. Em \LaTeX, as chaves
servem para agrupar coisas, para que sejam vistas como uma unidade
só. 

De modo geral, sequências de controle operam de acordo com os
parâmetros, ou argumentos, que passamos para elas. Se uma sequência
emprega um certo número de parâmetros (digamos, 2), ela considera que
eles são os (dois) agrupamentos imediatamente depois dela. Mas
atenção: o \LaTeX\ sempre considera \emph{agrupamento} a menor unidade indivisível
que encontra ao ler um texto! Letras que não estejam  envolvidas em
chaves são, cada uma, um elemento diferente, assim como sequências de
controle o são. Por outro lado, um texto envolvido entre chaves conta
como um único agrupamento, um único elemento.

Por exemplo, suponhamos que haja um comando \verb!\importante! para
destacar texto, que opere sobre um único parâmetro (o texto
importante). O que cada uma das linhas a seguir destaca?

\begin{footnotesize}
\begin{verbatim}
\importante Lembre-se de usar chaves!
\importante{fazer as compras}
\importante{Destacar textos {importantes}}
\end{verbatim}
\end{footnotesize}

Respostas: (Você tentou os exercícios? Vá lá, mais uma chance!)
Respectivamente: ``L''; ``fazer as compras'', e ``Destacar textos {importantes}''.

Comandos nem sempre precisam de argumentos. Por exemplo,
\verb!\newpage! termina a página atual e continua o texto na página
seguinte, e \verb!\maketitle! mostra o título, autor e data do texto.

\subsection{Regiões}

Você já deve ter reparado que há uma certa ``anatomia'' no
texto. Por exemplo, há imagens, citações, tabelas, poemas, listas,
enumerações, e descrições, só para citar alguns. Todos são 
elementos de natureza diferente do texto, tanto visual como
conceitualmente.

Essas regiões, também chamadas de \emph{ambientes}, são trechos do
texto que têm um papel diferente, e, assim, provavelmente demandam um
tratamento diferente.

Já usamos regiões uma vez nesta apostila: o corpo do texto, o
\emph{document}, onde vivem seus elementos visíveis. Neste ponto, você
já deve imaginar como fazer para delimitar um ambiente. Digamos, que
uma parte de nosso relatório seja pura magia. Para que isso seja de
fato incorporado ao texto, basta fazer:

\begin{footnotesize}
\begin{verbatim}
\begin{pura-magia}
Chirrin-chirrion!
\end{pura-magia}
\end{verbatim}
\end{footnotesize}


\subsection{Acentuação: para além do ASCII}\label{subsec:ascii}

Ao experimentar os exemplos dados até agora (se não fez, esta é uma
boa oportunidade! Tente gerar documentos a partir dos exemplos, eu
fico aqui esperando) você deve ter reparado que os caracteres
acentuados não aparecem no documento final. Mas experimente o seguinte
\begin{footnotesize}
\begin{verbatim}
\documentclass{article}
\begin{document}
Ol\'a mundo! Voc\^e come\c cou a notar algo?
\end{document}
\end{verbatim}
\end{footnotesize}

Não desespere. Acentuar é muito mais fácil do que isso. Tentemos outra
coisa
\begin{footnotesize}
\begin{verbatim}
\documentclass{article}
\usepackage[utf8]{inputenc}
\begin{document}
Olá mundo! Você começou a notar algo?
\end{document}
\end{verbatim}
\end{footnotesize}

Qual o resultado? E se você tentar o seguinte?
\begin{footnotesize}
\begin{verbatim}
\documentclass{article}
\usepackage[T1]{fontenc}
\begin{document}
Olá mundo! Você começou a notar algo?
\end{document}
\end{verbatim}
\end{footnotesize}

Uma das alternativas acima deve solucionar a questão dos acentos em
seu computador, a depender de como estão armazendas as letras no
seu texto. Mais precisamente, cada uma das linhas novas, que começam
por \verb'\usepackage', tenta dizer ao \LaTeX\ como interpretar a
\emph{codificação} do arquivo que ele irá processar.

\begin{detalhe}
O leitor atento poderá se perguntar: mas o texto que salvei é
\emph{puro}\footnote{Usamos aqui \emph{texto puro} como tradução da
  expressão em inglês \emph{plain text}: texto sem formatação.}, sem
formatação alguma --- como ele pode ser armazenado de mais de um modo?
quem determina que codificação o arquivo tem? 
A resposta direta a essa pergunta é a seguinte: arquivos são
armazenados como sequências de zeros e uns no computador (ao menos até
este momento, em 2010). A \emph{codificação} de um arquivo é o
conjunto de regras que associa a determinadas sequências de zeros e
uns a cada uma das letras de um texto.
\end{detalhe}

Apesar de os comandos para acentuação serem dispensáveis na maioria
dos casos, há situações em que pode ser útil saber um truque ou
outro. Principalmente quando o que se deseja é escrever algum nome
estrangeiro em algum ponto particular do texto, e não se sabe como
obter o caractere a partir do seu teclado.

O trecho a seguir é um excerto do \TeX book.

\medskip
\begin{center}\hrule\smallskip
\begin{tabular}{c|c}
\begin{minipage}{.405\textwidth}\footnotesize
\verbatiminput{03-verbatim-example-03}
\end{minipage} &
\begin{minipage}{.535\textwidth}
Erd\"os, B\=askara, Gabor Szeg\"o.

`\`o' (grave accent)
`\'o' (acute accent)
`\^o' (circumflex or “hat”)
`\"o' (umlaut or dieresis)
`\~o' (tilde or “squiggle”)
`\=o' (macron or “bar”)
`\.o' (dot accent)
`\u o' (breve accent)
`\v o' (há\v cek or “check”)
`\H o' (long Hungarian umlaut)
`\t oo' (tie-after accent)
`\c o' (cedilla)
`\d o' (dot-under accent)
`\b o' (bar-under accent)
`\oe',`\OE' (French ligature OE)
`\ae',`\AE' (Latin and 
             Scandinavian ligature AE)
`\aa,\AA' (Scandinavian A-with-circle)
`\o',`\O' (Scandinavian O-with-slash)
`\l',`\L' (Polish suppressed-L)
`\ss' (German “es-zet” or sharp S)

\end{minipage}
\end{tabular}
\smallskip\hrule
\end{center}
\medskip


Mas o que faz o comando `\verb'\usepackage''? Veremos a seguir.

\subsection{Pacotes}

Uma característica importantíssima do \LaTeX\  é sua
expansibilidade, que permite que ele se adapte às necessidades
dos mais variados usuários. Assim como é possível estender as
capacidades de um programa acrescentando-lhe `plugins', `add-ons', ou,
em mais baixo-nível, bibliotecas, é possível dotar o \LaTeX\ de mais
comandos, pela inclusão de \emph{pacotes}.

Pacotes são documentos de texto (como os que você escreve ao seguir
esta apostila). Certo, eles não são \emph{exatamente} documentos de
texto como os que você escreve agora: os pacotes possuem diversas
definições de comandos, macros e ambientes, que agregam funcionalidade
ao \LaTeX. Pacotes têm muitas vezes a extensão \extensao{sty}, embora
você não precise se preocupar com esse detalhe (ao menos enquanto você
não estiver escrevendo seus próprios pacotes, ou investigando as
fascinantes entranhas do sistema).

Para usar um pacote, basta usar o comando \verb'\usepackage'. Esse
comando faz com que o \LaTeX\ procure pelo arquivo do pacote e torne
sua funcionalidade disponível para que você dela disponha como quiser.
O argumento do comando é o nome do pacote. Pouco atrás usamos um
comando para poder usar acentos em arquivos codificados em \extensao{utf8}.
\begin{center}\footnotesize
\begin{verbatim}
\usepackage[utf8]{inputenc}
\end{verbatim}
\end{center}

Este comando tem um \emph{parâmetro opcional},
\texttt{utf8}. Parâmetros opcionais estão presentes em vários
comandos. Um parâmetro opcional pode ser omitido; ele geralmente
representa alguma configuração ou pequena alteração no modo de
funcionamento do comando.

Assim, é comum que pacotes possam ser configurados por meio de
parâmetros opcionais passados a eles.

Da mesma maneira, classes de documento também podem ser configuradas
por meio da passagem de parâmetros opcionais. Alguns exemplos: pode-se
passar os parâmetros opcionais \texttt{11pt}, \texttt{twocolumn},
\texttt{twoside}, \texttt{draft} para a declaração da classe
\texttt{article}. Assim, para um documento a ser impresso
frente-e-verso, em duas colunas, podemos escrever
\begin{footnotesize}
\begin{verbatim}
\documentclass[twocolumn,twoside]{article}
\begin{document}
...
\end{document}
\end{verbatim}
\end{footnotesize}

É importante notar que separamos os parâmetros opcionais por
vírgulas. Isso acontece para comandos como \texttt{documentclass} e
\texttt{usepackage}, mas não é válido para outros comandos (vide
seção~\ref{sec:comandos}).

Há pacotes para as mais diversas coisas: acrescentar cor ao texto,
usar capitulares (letras grandes, muitas vezes cheias de adornos, no
início de parágrafos), para descrever palavras-cruzadas, jogos de
xadrez, para desenhar, para fazer tabelas grandes, colocar trechos de
texto em números variáveis de colunas, acrescentar marcas d'água,
personalizar cabeçalhos e rodapés, e mesmo
``meta-pacotes.''\footnote{Pacotes que auxiliam a escrita de outros
  pacotes. Esses pacotes geralmente são de um gênero mais técnico,
  parecendo às vezes ``coisa de programador.''}

%\savebox{\mybox}{\protect\verb!\verb!}
\subsection{%
  \texorpdfstring{%
  Verbatim: ambiente \texttt{verbatim} e \texttt{\char`\\{}verb}}{%
  Verbatim: ambiente verbatim e \textbackslash verb}}

Por vezes o que queremos é que o texto digitado apareça exatamente
como o escrevemos. Veremos a seguir que o \LaTeX\ toma algumas
decisões por conta própria na hora de compor o texto, e os importantes
benefícios que esse comportamento traz consigo. Por hora, mencionemos
um importante exemplo: nem todo espaço no arquivo-fonte corresponderá
a um espaço na formatação final. Calma, as palavras não serão
coladas. Mas experimente usar dois espaços entre um par de palavras. O
que acontece\footnote{Não há resposta aqui \texttt{=)}.}?

Em algumas situações, como por exemplo em listagens de programas, pode
ser útil usar o \LaTeX\ como se ele não fosse mais do que uma máquina
de escrever digital. Queremos que o texto seja posto \emph{verbatim},
isto é, exatamente como foi escrito. Para isso, podemos usar (sic) o
ambiente \verb'verbatim'.

\medskip
\begin{center}\footnotesize\hrule\smallskip
\begin{tabular}{c|c}
\begin{minipage}{.465\textwidth}
\verbatiminput{03-verbatim-example-01}
\end{minipage} &
\begin{minipage}{.465\textwidth}
\begin{verbatim}
int main(int argc, char argv) {
  int resposta = 42;
  /* TODO: calcular a pergunta */
  return 0;
}
\end{verbatim}

\end{minipage}
\end{tabular}
\smallskip\hrule
\end{center}
\medskip

Há um outro método para ``cancelar'' a interpretação de caracteres,
para trechos menores, destinados a viver dentro de uma frase
comum. Por exemplo, as várias vezes em que me referi a comandos
\verb'\LaTeX', precisei fazer com que a interpretação do comando fosse
abortada (caso contrário, teria obtido \LaTeX). O comando que faz isso
é o \verb'\verb', que possui uma sintaxe especial. O comando é seguido
por um caractere qualquer (espaço vale!). Esse caractere servirá para
delimitar o fim do argumento de \texttt{\char`\\{}verb}. A esse
caractere se segue o texto a ser ``verbatimizado,'' que é todo o texto
até a próxima ocorrência do delimitador. Exemplo:
`\texttt{\char`\\{}verb!\char`\\LaTeX!}' resulta em `\verb!\LaTeX!',
mas `\verb'\LaTeX'' resulta `\LaTeX'.

Um último comentário. Tanto o comando quanto o ambiente verbatim
possuem uma versão ``estrelada'', que exibe os espaços em branco
\verb*'deste jeito aqui'. O ambiente é chamado \verb'verbatim*' e o comando
\texttt{\char`\\{}verb*}. 

\subsection{Escrevendo com fluência --- rudimentos de tipografia}

Já dissemos que o \LaTeX\ tem um jeito particular de dispor o texto
que escrevemos. Veremos agora que história é essa.

\medskip
\noindent\begin{minipage}{\textwidth}
\begin{center}\footnotesize\hrule\smallskip
\begin{tabular}{c|c}
\begin{minipage}{.465\textwidth}
\verbatiminput{03-verbatim-example-02}
\end{minipage} &
\begin{minipage}{.465\textwidth}
As grandiloquência exibicionista são 
pouco persuasiva para aqueles honestamente 
curioso.
   
Verdade                           isso. 
Para         quem         já tanto 
circunvaga o sentido, cheio de dedos no 
pântano dos significados, um pouco de tento 
com o que passa a ser floreio decorativo é 
no mínimo cortês.

E tudo \LaTeX ado apropriadamente. 
\emph{Muito} apropriadamente.
Usando alguns comandos \LaTeX\ que já foram 
vistos\dots.

\end{minipage}
\end{tabular}
\smallskip\hrule
\end{center}
\end{minipage}
\medskip

Algo que salta à vista de primeira é que as quebras de linha não são
respeitadas. Também parece que os espaços a mais são
desconsiderados\dots e a realidade não está mesmo longe disso: um
espaço ou vários espaços são a mesma coisa para o \LaTeX. Uma (única)
quebra também é equivalente a um espaço. Duas quebras de linha, por
outro lado, fazem com que um novo parágrafo seja iniciado.

Notável também é o fato de que o primeiro
parágrafo\index{paragrafo@parágrafo} não tem recuo, enquanto que os
demais o têm. Isto se deve ao fato de que para algumas culturas (em
particular na tipografia de língua inglesa), não é costumeiro marcar a
primeira linha de um parágrafo com recuo a menos que este seja
precedido imediatamente por outro parágrafo. Afinal, esse recuo tem
por objetivo facilitar a identificação visual do novo parágrafo, o que
não é necessário se o parágrafo é o primeiro de uma seção ou capítulo
do texto, por exemplo.

Encontramos pela primeira vez também os comandos \verb'\LaTeX', que
escreve \LaTeX, e \verb'\emph', que \emph{enfatiza} o texto que lhe é
passado como parâmetro. Note que o que o comando faz é enfatizar: o
jeito como ele faz isso não é a nossa preocupação nesse momento.

\begin{center}
\it O que importa aqui é que o trecho tem que ser destacado.  
\end{center}

E isso é diferente de dizer que o texto deve ser posto em negrito, ser
sublinhado, ser escrito em fúcsia, \reflectbox{ou} \reflectbox{de}
\reflectbox{algum} \reflectbox{jeito} \reflectbox{estranho}. Afinal, o
paradigma aqui é que a aparência do texto refletirá a função, o papel
semântico desempenhado por cada um de seus elementos. Assim,
descreve-se num primeiro momento o que cada um
\emph{significa}\index{marcacao semantica@marcação semântica},
deixando-se para outra etapa (quando pertinente) o ajuste do modo pelo
qual essa função é realçada visualmente.

\LaTeX\ lida com uma granularidade maior de conceitos do que comumente
nos é dado controlar em ambientes usuais de edição de texto; conceitos
que, a princípio, podem surpreender os não iniciados ao universo dos
cuidados tipográficos. A partir de agora, e à medida que adquire
experiência com um sistema tipográfico de alta qualidade como o
\LaTeX, você notará uma série de mudanças na percepção de um texto. Seu
vocabulário vai crescer, seus olhos e atenção serão exercitados em
novas direções, e muito provavelmente você se surpreenderá com a
influência que ``detalhes'' têm no ritmo e facilidade de leitura de um
texto. Mãos à obra!

\subsubsection{Hífens e hifenação}\label{subsec:hifenacao}

Muito embora haja apenas um tipo de hífen em seu teclado, existem
muito mais hífens na tipogravia. Há aquele usado em palavras
compostas, como ``guarda-chuva'' ou ainda ``resguardar-se'', e que também
servem para marcar a quebra de uma palavra no fim de uma linha
(sua \emph{hifenação}); há o traço usado para indicar um intervalo de números,
por exemplo 12--14; há o travessão --- o mais longo entre os hífens; e
há o sinal de menos, usado em equações, como em $20-3=17$. É fácil
produzir cada um desses símbolos em \LaTeX.

\begin{itemize}\footnotesize
\item \verb'guarda-chuva', \verb'resguardar-se'
\item \verb'exercícios das páginas 12--14' 
\item \verb'no dia de hoje --- véspera de amanhã'
\item \verb'diga-me também que $2-2=5$, Winston'
\end{itemize}

O último dos exemplos acima introduz o chamado \emph{modo
  matemático}\index{modo matematico@modo matemático}, assunto da seção
\ref{sec:matematica}.

Mas há ainda o que falar sobre hifenação. Na maior parte dos casos, o
\LaTeX sabe hifenar corretamente as palavras de diversos idiomas (o
portugês entre eles). Para isso basta usar o pacote \pacote{babel},
passando como parâmetro \parametro{brazil}. Algumas vezes, porém,
usamos termos que possuem uma hifenação pouco comum, ou usamos
palavras que o \LaTeX não consegue hifenar a contento. Quando isso
ocorre, podemos dizer explícitamente em que pontos uma palavra pode
ser hifenada. Há dois modos de fazê-lo: pode-se, no preâmbulo,
adicionar um comando \verb'\hyphenation', que leva como parâmetro uma
lista de hifenações, separadas por espaços, como abaixo. Note que não
se podem usar comandos ou caracteres especiais no argumento do comando. 

\begin{center}
  \verb'\hyphenation{FNAC A-bra-cur-six}'
\end{center}

No exemplo acima, FNAC, fnac e Fnac não serão jamais hifenadas, ao
passo que Abracursix e abracursix o serão, segundo os hífens
especificados.

Outro modo é explicar onde uma determinada ocorrência de uma palavra
pode ser hifenada, quando ela ocorre no texto. Nesse caso, a sugestão
de hifenação vale naquele ponto somente. O \LaTeX\ não se lembrará
dela se a palavra for usada novamente.

\medskip
\begin{center}\footnotesize\hrule\smallskip
\begin{tabular}{c|c}
\begin{minipage}{.465\textwidth}
\verbatiminput{03-verbatim-example-04}
\end{minipage} &
\begin{minipage}{.465\textwidth}
É algo assim, como direi?
su\-per\-ca\-li\-frag\-i\-lis%
\-tic\-ex\-pi\-a\-li\-do\-cious

\end{minipage}
\end{tabular}
\smallskip\hrule
\end{center}
\medskip

\subsubsection{Caracteres reservados}

São dez os caracteres reservados pelo \LaTeX\ para funções especiais
(ou seja, é preciso alguma ginástica para obtê-los). Eles são os seguintes.

\begin{center}
\verb'\   _   ^   ~   &   #   {   }   %   $'
\end{center}

A barra marca o início de um comando; o ``underscore'' e o circumflexo
são usados no modo matemático (seção~\ref{sec:matematica}); o ``e
comercial'' é usado em tabelas (seção~\ref{sec:tabelas}); o ``jogo da
velha'' é usado na definição de comandos (seção~\ref{sec:comandos});
as chaves agrupam texto; o caractere de porcentagem marca o início de
comentários; e o cifrão delimita o modo matemático.

Esses caracteres podem ser usados em um documento prefixando-os por
uma barra.

\medskip
\begin{center}\footnotesize\hrule\smallskip
\begin{tabular}{c|c}
\begin{minipage}{.465\textwidth}
\verbatiminput{03-verbatim-example-05}
\end{minipage} &
\begin{minipage}{.465\textwidth}
\centering \input{03-verbatim-example-05}
\end{minipage}
\end{tabular}
\smallskip\hrule
\end{center}
\medskip

A exceção é a barra, que pode ser obtida  por meio do
comando \verb'\textbackslash'.
 
\subsubsection{Apurando os sentidos: ligaduras, kerning}% e história}

As letras por vezes requerem pequenas modificações no espaçamento
entre si, ou mesmo em sua forma, a depender dos símbolos que estão
próximos de si. Por exemplo, alguns pares de letras são aproximados,
enquanto outras vezes, partes de letras se fundem. Observe os exemplos
abaixo.

\medskip
\noindent\begin{center}%
\scalebox{3}[3]{f{i}}\hfil%
\scalebox{3}[3]{fi}\hfil%
\scalebox{3}[3]{T{a}}\hfil%
\scalebox{3}[3]{Ta}
\\[.9cm] 
\scalebox{3}[3]{s{t}}\hfil%
\scalebox{3}[3]{st}\hfil%
\scalebox{3}[3]{f{l}}\hfil%
\scalebox{3}[3]{fl}
\end{center}
\medskip

Ligaduras (do inglês, \emph{ligatures}), ocorrem quando um agrupamento
de letras é substituído por um outro símbolos, quer para melhorar sua
legibilidade, quer para tornar o texto mais belo.

Já o \emph{kerning} é um aumento ou diminuição do espaço entre letras,
que varia de acordo com o entorno de cada caractere.

\medskip
\noindent\begin{center}%
\scalebox{2}[2]{Uma {T}orta {P}ara {J}aiminho}

\scalebox{2}[2]{Uma Torta Para Jaiminho}

\medskip

%\hfil\scalebox{2}[2]{\textsc{{V}á}}%
%\scalebox{2}[2]{\textsc{Vá}}\hfil%
\scalebox{2}[2]{\textsc{{A}v{a}r{o}}}\hfil%
\scalebox{2}[2]{\textsc{Avaro}}%
\hfil\scalebox{2}[2]{\textsc{{P}a{r}a}}
\hfil\scalebox{2}[2]{\textsc{Para}}\hfil
\end{center}

\begin{comment}
  (breve) história do TeX, do LaTeX e irmãos
  Resumo de como o sistema ``monta'' as páginas
\end{comment}




\section{Estruturando o texto}

Textos, assim como animais, possuem uma anatomia. Essa anatomia é o que permite ao leitor se localizar em sua leitura, identificar algo que procura. A estrutura do texto, além disso, carrega uma mensagem em si, ao menos em potencial, ao refletir o encadeamento do texto.

A depender da classe do documento, há uma certa variedade de tipos de segmentações à nossa disposição para organizar o texto.\footnote{E, como tudo o mais quando se trata da família \TeX, esse conjunto pode ser estendido e modificado como melhor nos aprouver.}
Artigos podem ser particionados em seções, subseções, subsubseções, apêndices.
Livros possuem, adicionalmente ao que está disponível em artigos, capítulos (contendo um certo número de seções).
Relatórios possuem adicionalmente (a livros) \emph{partes} (que contém capítulos).
E por aí vai.

Você pode mesmo criar seu próprio nível hierárquico, como parágrafos.%%, como veremos na seção~\ref{sec:contadores}.

Neste capítulo, abordaremos, a título de exemplo, secionamento (segmentação) de um texto em artigos (documentos da classe~\pacote{article}). O comportamento apresentado em livros, relatórios etcétera é análogo, e em caso de dúvida basta recorrer aos manuais da respectiva classe (que, por padrão, vêm juntamente com o pacote quando a sua distribuição \LaTeX\ é instalada).

\subsection{Títulos, autor e data de documentos}

Em muitas classes de documentos, estão disponíveis os comandos para
definir o título, o(s) autor(es) e a data do documento. Cada classe
exibe essa informação de um modo, mas em boa parte delas você define o
título com um comando \verb'\title{Minhas Férias}', o autor usando o
comando \verb'\author{YoMoiIchEu}'. A data é composta automaticamente
com a data em que o documento for processado (no idioma do
documento). Você pode escolher (fixar) a data usando o comando
\verb'\date{Muito, muito tempo atrás}'.

Depois de especificados o título e o autor (mais de um autor pode ser
declarado, separando-se seus nomes por \verb'\and'), você escolhe o
ponto do texto no qual quer que apareçam, e usa o comando
\verb'\maketitle'. Voilà!

\subsection{Marcando a anatomia}\label{sec:comandos-de-secionamento}

O exemplo a seguir ilustra o uso de seções, subseções, subsubseções, seções não numeradas\index{secao@seção} e apêndices.


\medskip
{\footnotesize\verbatiminput{exemplos/05-01-sectioning}}

\medskip

Dizemos que uma seção inicia a partir do comando \verb'\section'. O
argumento que este comando leva é o título da seção. O mesmo acontece
para sub-seções, e as demais divisões do texto.

Seções são numeradas por padrão. Para obter uma seção, sub-seção, etc. não numerada,
use o respectivo comando em sua versão com asterisco, por
exemplo \verb'\section*{Prefácio}'.

A classe de documento e os pacotes que você usa definem quais os
comandos de secionamento disponíveis. Livros, por exemplo, têm
\verb'\chapter', relatórios têm \verb'\part', e por aí vai.

\subsection{Sumários}

Falemos agora do acompanhamento natural de um texto secionado:
sumários (ou índices). Fazer um sumário\index{sumario@sumário}, com o
\LaTeX\ é muito simples. Marque os títulos das partes usando os comandos de
secionamento que acabamos de ver, e, no ponto do texto em que deseja
acrescentar o índice, coloque o comando \verb'\tableofcontents'.

Uma vez marcadas as seções do texto e solicitado o índice, o
\LaTeX\ anota (em um arquivo auxiliar) as páginas em que começam as
seções do texto, à medida que o processa. Essas informações
são usadas para escrever o sumário. A depender da parte do
texto na qual sumário foi posto, pode ser necessário processar o texto duas vezes\footnote{Para sermos estritamente precisos, é possível construir documentos anômalos que ``ludibriam'' o índice e requerem mais processamento, mas não se preocupe: se seu texto tiver essa propriedade, certamente você a terá causado conscientemente.} (na primeira as páginas em que ocorrem as seções são anotadas, e
na segunda as entradas no sumário são atualizadas com os valores
corretos).
Além das informações escritas no arquivo auxiliar (que tem a
extensão~\extensao{aux}), o comando \verb'\tableofcontents' faz ser
gerado um outro arquivo, com a extensão \extensao{toc} (\emph{table of
  contents}), que contém o sumário em si. 

É possível --- e igualmente fácil --- gerar listas de figuras, tabelas
ou quaisquer outros elementos usando \LaTeX. 
%% Veremos como fazê-lo nas
%% seções~\ref{sec:floats} e~\ref{sec:contadores}.\footnote{Para os
%%   curiosos, a receita: basta usar  \texttt{\char`\\{}listoffigures}
%%   e~\texttt{\char`\\{}listoftables} em conjunto com os ambientes
%%   \ambiente{figure} e \ambiente{table}.}


Os comandos de secionamento possuem em geral um parâmetro opcional,
que é uma versão ``mais compacta'' do título, para ser usada em no
sumário (ou, por vezes, no cabeçalho ou rodapé de páginas).
\begin{footnotesize}
\begin{verbatim}
\section[Prova Documental]{%
  Documento provando a corretude do argumento %
  que concebi em uma longa insônia alcoolizada}
\end{verbatim}
\end{footnotesize}


\subsection{Referenciando elementos do texto}\label{sec:ref-e-label}

Assim como sumários são elementos importantes para a orientação do leitor-explorador, existem outros tipos de referências que ocorrem com frequência. Outro modo de remeter o leitor a um trecho, página --- em geral, a um \emph{elemento} qualquer --- do texto é usando \emph{referências}, que são o assunto desta seção.

Há vários tipos de referências, e várias maneiras de se referir a
alguma coisa. Podemos fazer referência a uma~\emph{figura} ou a
um~\emph{capítlo}; assim como podemos identificá-los por um número
próprio, ou pelo número da página em que se iniciam.\footnote{Outros
  tipos de referência incluem: referência a notas de rodapé e a
  elementos ``externos'', como itens de bibliografia, glossário e
  índices remissivos, que são explorados nas seções~\ref{sec:biblio} 
  e~\ref{sec:indice-glossario}.} Em sua essência, porém, existem
apenas dois componentes imprescindíveis em uma referência: um
\emph{indicador} e um \emph{indicado}.

Para referenciar algo em \LaTeX, usamos \emph{rótulos}. Rótulos são nomes que damos a algum elemento do texto. Para criar um rótulo, use o comando \verb'\label{nome do rotulo}'\index{label@\verb'\label'}, e para referenciá-lo use o comando \verb'\ref{nome do rotulo}'\index{ref@\verb'\ref'}. 

Quando usados em um ponto do texto, o label fica automaticamente associado à página, seção (subseção e etcétera) a que pertence aquele ponto no texto. Em enumerações, associa-se ainda ao item correspondente, e assim vale para figuras, tabelas e ambientes em geral.

Note que no exemplo o nome do rótulo não tem acentos. Quando criar seus rótulos, use apenas caracteres simples: mais especificamente, caracteres  \acronimo{ascii}\footnote{Entre os caracteres \acronimo{ascii} estão as letras de `\texttt{a}' a `\texttt{z}' (maiúsculas e minúsculas), os dígitos, o espaço em branco, assim como os caracteres ``\texttt{@\#\$\%\&*`'"!()-\char`\_=+[]\char`\{\char`\}\^{}\~{},.;/\char`\\|<>?}''.}.


\section{Alguns elementos do texto}

\subsection[Listas]{Listas {\it \&} Cia.}

itemise, enumerate, description, quotation

\begin{center}\footnotesize\hrule\smallskip
\begin{tabular}{c|c}
\begin{minipage}{.465\textwidth}
\begin{verbatim}
\begin{itemize}
\item cebola,
\item açafrão, e
\item alho.
\end{itemize}
\end{verbatim}
\end{minipage} &
\begin{minipage}{.465\textwidth}
\begin{itemize}
\item cebola,
\item açafrão, e
\item alho.
\end{itemize}
\end{minipage}
\end{tabular}
\smallskip\hrule
\end{center}


\begin{center}\footnotesize\hrule\smallskip
\begin{tabular}{c|c}
\begin{minipage}{.465\textwidth}
\begin{verbatim}
\begin{description}
\item[cebola] Muito empregada p/ temperar.
\item[açafrão] Também.
\item[alho] Idem.
\end{description}
\end{verbatim}
\end{minipage} &
\begin{minipage}{.465\textwidth}
\begin{description}
\item[cebola] Muito empregada p/ temperar.
\item[açafrão] Também.
\item[alho] Idem.
\end{description}
\end{minipage}
\end{tabular}
\smallskip\hrule
\end{center}

\subsection{Alinhamento}\label{sec:alinhamento}

Boa parte dos textos possui alinhamento justificado, i.e., possui
ambas as margens retas e paralelas. Nem sempre isso é
desejado. Existem muitas maneiras de definir o alinhamento do texto:
falamos de duas delas aqui.

\subsection{Texto não-justificado}

\begin{flushleft}
No ambiente
\ambiente{flushleft}\index{flushleft@\ambiente{flushleft}}, o texto é
``empurrado'' para a esquerda. Os espaços não são nem esticados nem
comprimidos. O efeito resultante são linhas de comprimento variável, 
 por vezes uma opção interessante de diagramação.
\end{flushleft}

\begin{flushright}
Simetricamente,
\ambiente{flushright}\index{flushright@\ambiente{flushright}} tem o
comportamento que seria de se esperar, fazendo  parágrafos  ficarem com a esquerda~\emph{rasgada}, isto é, sejam empurrados para a direita.
\end{flushright}

\subsection{Elementos flutuantes}\label{sec:floats}
% TODO: falar de listoffigures e listtof

Tipógrafos atentam para uma série de características na disposição do
texto que frequentemente passam despercebidas ao nosso consciente. Uma
delas é o equilíbrio entre o texto que se espalha pelas páginas e os
demais elementos, como figuras e tabelas, que pontuam a paisagem aqui
e ali. 

O \LaTeX\ toma várias precauções na disposição desses elementos,
ditos~\emph{flutuantes} (porque sua posição não é fixa no texto como a
de uma palavra em uma frase). É como se os elementos fossem troncos de
árvore à deriva sobre a correnteza de palavras que compõe o texto.

Figuras e tabelas são somente alguns exemplos de elementos
flutuantes. Eles são ambientes (respectivamente~\ambiente{figure}
e~\ambiente{table}) que encapsulam o conteúdo que irá flutuar.

% colocar exemplo usando caption
% enfatizar que o latex sabe onde colocar as imagens
% opções de posicionamento
% todo: como se insere no mecanismo de composição de página

\subsubsection{Figuras}

Figuras são uma ferramenta poderosa na composição de textos, quando
usadas com parcimônia. É possível colocar imagens no documento dizendo
ao \LaTeX\ sua localização (ou apenas seu nome, se estiverem na mesma
pasta que o documento). Também é possível desenhar usando o próprio
\LaTeX, por exemplo, com o pacote \pacote{Tikz}.

Para colocar figuras em um documento \LaTeX, basta usar o comando
\begin{center}
\verb!\includegraphics{nome-do-arquivo}!
\end{center}
em que a extensão do tipo de arquivo não precisa ser incluída. Mas
atenção: nem toda extensão de imagem é conhecida pelo
\LaTeX\ nativamente, embora baste usar um pacote para superar o
problema, na maior parte dos casos. Acrescente
\verb!\usepackage{graphicx}! no preâmbulo de seu documento e você
poderá incluir imagens \extensao{png}, \extensao{jpg} e
\extensao{pdf}, para citar algumas.

\begin{figure}
  \begin{center}
    \input{exemplos-externos/tikz-01}
    \caption{Uma figura gerada com o pacote \pacote{Tikz}.}\label{fig:tikz:piramide-cortada}
  \end{center}
\end{figure}


\begin{figure}
  \begin{center}
    \input{exemplos-externos/tikz-02}
    \caption{Outra figura gerada com o pacote \pacote{Tikz}.}\label{fig:tikz:layers}
  \end{center}
\end{figure}

\subsubsection{Tabelas}\label{sec:tabelas}

Tabelas são um dos maiores desafios de diagramação que têm que ser enfrentados frequentemente por tipógrafos. Cada tabela é única, e uma leve alteração de espaçamento tem o poder de alterar bastante o destaque de uma e outra informação.

\medskip
\begin{center}\hrule\smallskip
\begin{tabular}{c|c}
\begin{minipage}{.405\textwidth}\footnotesize
\verbatiminput{exemplos/05-tabela-01}
\end{minipage} &
\begin{minipage}{.535\textwidth}
\begin{center}
\begin{tabular}{clcr|r|}
  a & b & c & d & e\\
  fg & hi & jk & lm & no\\
  \hline
  p & q & r & s & t
\end{tabular}
\end{center}

\end{minipage}
\end{tabular}
\smallskip\hrule
\end{center}
\medskip


\begin{center}
\begin{tabular}{llr}
\toprule
\multicolumn{2}{c}{Name} \\
\cmidrule(r){1-2}
First name & Last Name & Grade \\
\midrule
John & Doe & $7.5$ \\
Richard & Miles & $2$ \\
\bottomrule
\end{tabular}
\end{center}



\begin{table}\centering
  \caption{Gastos {\it \&} despesas / 1º semestre}

  \begin{tabular}{crrrrrrr}
    mês & jan & fev & mar & abr & mai & jun & total\\
    \hline
  receita & $10$ & $0$ & $5$ & $20$ & $12$ & $13$ & $60$\\
  gastos  & $-3$ & $-4$ & $-3$ & $-3$ & $-5$ & $-3$ & $-21$ \\
  \hline
  balanço & $7$ & $-4$ & $2$ & $17$ & $7$ & $10$ & $39$\\
  \end{tabular}
\end{table}

\section{Expandindo o \LaTeX}

Há uma grande quantidade de comandos disponíveis ao usuário de
\LaTeX\ (e ainda mais são criados em pacotes novos
continuamente). Embora a maior parte das coisas que se pode querer
fazer em \LaTeX\ já exista na forma de algum comando, não raro podemos
nos valer, com proveito, do poder de \emph{estender o \LaTeX} ---
como elástico que é.
 
Há uma série de razões por que usar comandos é uma boa. Falaremos
delas\ldots\ ao fim desta seção.

\subsection{Criando comandos}\label{sec:comandos}

O mecanismo primário de programação do \TeX\ (e logo o \LaTeX) é a
definição de macros. Se isso soa grego, então ignore. Criar, ou
\emph{definir}, comandos é instruir o sistema sobre o que fazer quando
ele encontra determinada sequência de controle.

Definimos comandos usando o comando \verb'\newcommand'. O tipo mais
simples de comando que há é a mera substituição. É como se
definíssemos um apelido para algo que escrevemos com frequência.

Por exemplo, se estou a compor um estudo sobre a genealogia da família
de feiticeiros do castelo \mbox{Rá-tim-bum}, faz sentido definir os
comandos \verb'\strad', e \verb'\nino'
\newcommand{\stradv}{Stradivarius Victorius}
\newcommand{\nino}{Antonino}
\begin{ttsample}
\index{newcommand@{\tt\char`\\newcommand}}
  \defNewCommand{stradv}{Stradivarius Victorius}\\
  \defNewCommand{nino}{Antonino}
\end{ttsample}
que define \verb'\stradv' como sinônimo para ``{\tt Stradivarius
  Victorius}'', e \verb'\nino' analogamente. Então para escrever
``\nino\ \stradv\ II é o mais novo \stradv'', basta escrever
\begin{ttsample}
\macroCall{nino}\barra\ \macroCall{stradv}\barra\ II
 é o mais
  novo \macroCall{stradv}.
\end{ttsample}


Esse tipo de comando é particularmente útil (para definir siglas
grandes, por exemplo, ou nomes comuns). Mas por vezes não nos
basta. Queremos mais. Queremos que o comando seja parametrizável, ou
seja, que parte dele seja alterável. Fazemos isso dizendo ao
\LaTeX\ que o comando leva parâmetros, passando para
\verb'\newcommand' um parêmetro opcional: o número de argumentos
(parâmetros) que o comando usa. Por exemplo,
\begin{ttsample}
  \defNewCommandWithParameter{estrangeiro}{1}{``\#1''}
\end{ttsample}

Este comando coloca aspas em torno de termos estrangeiros. Note o uso
do caractere reservado `{\tt\#}'. Seguido de um número $n$ entre $1$ e
$9$, ele referencia o $n$-ésimo argumento que foi passado para o
comando. Além disso, observe que o número de parâmetros é posto entre
colchetes, \emph{entre} o nome da macro\footnote{Macro e comando são
    sinônimos aqui.}\index{macro@macro} e a sua definição.

Mais um último exemplo. O que o comando \verb'\formal' abaixo faz?
(\emph{Dica: é possível passar argumentos vazios para um comando.})

\begin{ttsample}
  \defNewCommandWithParameter{formal}{2}{Sr\#1 \#2}\\
\end{ttsample}

\subsection{Modificando comandos}

Só é possível definir um comando uma vez. Se após isso queremos mudar
seu significado, o que fazemos é \emph{redefiní-lo}.
\begin{ttsample}
  \redefCommand{emph}{PRESTA ATENÇÃO AQUI!}
\end{ttsample}

A sintaxe é a mesma da criação usual de comandos. Ao redefinir um
comando, não é preciso continuar usando o mesmo número de parâmetros
que originalmente ele usava. É como definir pela primeira vez um
comando, só que você acrescenta ``{\tt re}'' antes de {\tt
  newcommand}.\footnote{A vantagem de ocorrer um erro se você tenta
  definir um comando que já existe é que desse modo o \LaTeX\ garante
  que, se você está redefinindo um comando, está fazendo-o consciente
  de que está sobrescrevendo um comando existente. Isso é importante
  quando você está manipulando macros que serão usadas por outras
  pessoas, ou por você mesmo daqui a algum tempo.}

\subsection{Criando e modificando ambientes}

Você cria um ambiente usando \verb'\newenvironment'. Um ambiente é um
comando com dois argumentos: o texto que você quer que seja inserido
\emph{antes} do início da região,e o que quer que seja inserido
\emph{depois}. É possível (e simples) definir ambientes que recebam
parâmetros. Deixamos por conta do leitor descobrir como fazê-lo.

\bigskip
\begin{ttsampleflushleft}%
\macroCallWithThreeParameters{newenvironment}{com fofoca}{Você não
    acredita o que\ldots}{\ldots e essa agora!}\\
  \ttbegin{com fofoca}\\
    Casaram!\\
  \ttend{com fofoca}
\end{ttsampleflushleft}

\subsection[Estender pra quê?]%
{Não sou mandão. Por que ficar criando comandos?}
\label{sec:porque-estender}

Terminamos esta seção com alguma discussão sobre o que vimos
aqui. Definir comandos é uma parte importante do uso do \LaTeX. É com
eles que você conseguirá marcar a estrutura e o significado de cada um
dos elementos do seu texto. Comandos não precisam fazer nada (por um
bom tempo) para que sejam úteis. Por exemplo

\begin{ttsample}
  \defNewCommandWithParameter{palavraChave}{1}{\#1}
\end{ttsample}

O importante aqui, a princípio, é que o \emph{significado} seja
explicitado. Futuramente, à medida que você ganhar fluência em
\LaTeX, verá que um texto semanticamente marcado transforma-se
facilmente em uma obra de encher os olhos. Por exemplo, se
pudermos confiar que todas as palavras-chave do texto estão marcadas
com o comando \verb'\palavraChave', e toda palavra estrangeira com
\verb'\estrangeira', será fácil colocar em um parágrafo todas as
palavras-chave do texto automaticamente,\negthinspace\footnote{Isto é,
  de modo que novas anotações sejam agregadas a ele toda vez que o
texto for processado, sem necessidade de interferência manual no
processo} ou fazer um glossário dos termos em outro idioma. A
anotação do significado de um elemento é um ponto de referência ao
qual podemos acrescentar formatação e ações.


Comandos estáticos agilizam a digitação, e são ferramentas importantes
para garantir a uniformidade de nomenclatura no texto. 
Conceitos que ainda estamos concebendo ou batizando, termos de cuja
tradução não estamos certos, siglas e  elementos que desejamos
referenciar em glossários ou índices, todos esses são fortes
candidatos a serem parcial ou totalmente encapsulados por um comando
ou ambiente personalisado. Esse é um dos modos pelos quais o
\LaTeX\ apoia o paradigma \acronimowysiwym.\footnote{Além disso, o
  ponto do texto em que ocorre a
  definição do comando ``encapsulador'' é o lugar mais seguro
  (de fato, é \emph{o}~lugar) para se colocar marcações visuais, como
  \textbf{negrito}, \textsf{sem serifa} ou \textsc{versalete}, já que
  assim as mesmas transformações visuais são assim aplicadas
  consistentemente a todos os elementos com igual marcação semântica.}


% situações em que comandos são importantes: abreviações, semântica,
% dry


% encapsulando formatação em semântica (negrito, small capitals,
% centralizando, sans  typewritter)

\section[Múltiplos arquivos]{Projetos com vários arquivos}

input, include, includeonly

\section{Símbolos}

Diagramação é a disposição de símbolos. E há uma infinidade
deles. Citamos nesta apostila alguns deles, mas certamente não o
suficiente para atender à sua necessidade. Recomendamos fortemente que
mantenha uma cópia do excelente trabalho de Scott Pakin, \emph{The
  Comprehensive  \LaTeX\ Symbol
  List}~\cite{article:comprehensive-symbols}, que muito provavelmente
já está em alguma parte da documentação de sua instalação do sistema,
e que exibe uma lista organizada de aproximadamente cinco mil símbolos
que estão a sua disposição. 

Entre os símbolos disponíveis, estão elementos decorativos, símbolos
fonéticos, matemáticos, de linguagens arcaicas, musicais,
genealógicos, enxadrísticos, químicos, diacríticos incomuns ou
compostos, de diagramas de Feynman, de segurança, de legenda em mapas,
etcétera. Nada inesperado para um sistema que permite escrever em
élfico\ldots

%todo escrever em élfico

\section{Matemática}\label{sec:matematica}\index{modo matematico@modo matemático}

Ah, a matemática\ldots Ela é em grande parte a razão pela qual temos o
\LaTeX\ (e os computadores!). Aqui, em particular, o \LaTeX\ brilha.


Existem dois ``modos'' principais nos quais o \LaTeX\ pode operar
quando escreve expressões matemáticas: o \emph{modo matemático inline}
e o \emph{modo matemático ``display''}. Ele está no primeiro, em
geral, quando está escrevendo uma fórmula que deverá ocupar um espaço
limitado (no meio de um parágrafo, por exemplo), mas também em índices
ou em frações, como veremos adiante.

É importante perceber que as regras de espaçamento entre letras são
diferentes quando se está trabalhando no modo matemático. As letras
são postas em um tipo itálico, e os espaços são desconsiderados entre
letras; o espaço entre caracteres como $=$, $+$ e $-$ mudam, e
parágrafos (duas quebras de linha consecutivas) não são
permitidos.

Nesta seção faremos uma pequena incursão na composição de fórmulas
usando \LaTeX. Tenha em mente que há uma série de parâmetros que
afetam a legibilidade de uma expressão matemática --- e
mencionaremos apenas algumas delas. Em todos os casos, lance mão de
seu bom-senso, pergunte a opinião de seus amigos, e você não deve ter
problemas. Mãos à obra!

\subsection{Entrando no modo matemático}

Já mencionamos a existência de dois modos matemáticos. Para escrever
uma fórmula em meio a um parágrafo, basta escrevê-la entre um par de
cifrões \verb'$'. A fórmula será então tratada peloa \LaTeX\ como
qualquer outra palavra no parágrafo: \verb'$a + b$' resulta em
$a+b$. A presença de espaços é indiferente no modo
matemático. `\verb'a+b'', `\verb'a +b'', e `\verb'a + b'' são todas formas
equivalentes. Uma grande vantagem disso é que o autor pode formatar a
expressão como melhor lhe convier em termos de legibilidade quando
está a escrever o texto, e o resultado não será ``estragado'' por
isso. Esse é o modo de escrever matemática no meio da linha
(\emph{inline}\index{modo matematico!inline@inline}) . Outro modo é
colocar a fórmula em um banner, com destaque: é o modo de exibição
(\emph{display})\index{modo matematico!display@display}. Para usá-lo,
coloque a fórmula entre `\verb'\['' e `\verb'\]'': a expressão
\verb'\[a\times b = c.\]' faz o \LaTeX\ produzir
\[
a\times b = c.
\]

(O ponto final foi colcado só para terminar a frase, poderia ser
omitido sem maiores consequências que uma frase interminada.)

\subsection[Índices e expoentes]{Para cima e para baixo}

É simples: para gerar uma expressão ``superscrita'' a outra, ($a^b$)
usa-se o comando `\verb'^''. Para ``subscritos'', usamos
'\verb'_''. Observe os exemplos, e note que a necessidade de agrupar
alguns conjuntos de símbolos para obter certos resultados.

\medskip
\begin{center}\hrule\smallskip\footnotesize
\begin{tabular}{c|c}
\begin{minipage}{.405\textwidth}\footnotesize
\verbatiminput{exemplos/09-sub-e-superscripts-math-01}
\end{minipage} &
\begin{minipage}{.535\textwidth}\setlength{\parindent}{1pc}
\[  a^b = c^de  \]

\[  a^b = c^{de}  \]

\[  a_b = \log c \approx f(b)  \]

\[  a^{(c + d)} 
    = \lim_{a \to 0} \gamma^{a\tau}  
\]
\[  \sum_{i=1}^n i =  n(n+1)/2 \]

\end{minipage}
\end{tabular}
\smallskip\hrule
\end{center}
\medskip

Repare que para obter as expressões ``lim'' e ``log'' têm texto
escrito de modo diferente. Para  escrever texto na fonte romana
(i.e., fonte to texto corrente), é preciso sair temporariamente do
modo matemático, para que os espaços voltem a valer. Isso pode ser
feito usando-se uma~\emph{caixa}%%
%%(sobre caixas, vide a seção~\ref{sec:caixas})
. (Preste atenção aos espaços!)

\medskip
\begin{center}\hrule\smallskip
\begin{tabular}{c|c}
\begin{minipage}{.405\textwidth}\footnotesize
\verbatiminput*{exemplos/09-sub-e-superscripts-math-02}
\end{minipage} &
\begin{minipage}{.535\textwidth}\setlength{\parindent}{1pc}
\[ 
\mbox{para todo } a \in A 
\mbox{ há um único } b \in B
\]

\end{minipage}
\end{tabular}
\smallskip\hrule
\end{center}
\medskip

Alguns comandos para digitar expressões matemáticas, como o
\verb'\frac' levam mais de um parâmetro. Outros, como o \verb'\choose'
operam sobre todo o conteúdo do grupo que o contém, usando como
operandos o texto que se encontra à sua esquerda e à sua direita no
grupo.

\medskip
\begin{center}\hrule\smallskip
\begin{tabular}{c|c}
\begin{minipage}{.405\textwidth}\footnotesize
\verbatiminput{exemplos/09-sub-e-superscripts-math-03}
\end{minipage} &
\begin{minipage}{.535\textwidth}\setlength{\parindent}{1pc}
\[ \frac{a}{b} = \sqrt cD = \sqrt{cD}
  = \sqrt[n] p \]

\[ {a \choose b} 
   = {a \choose {b +c \choose c +d}} \]

\end{minipage}
\end{tabular}
\smallskip\hrule
\end{center}
\medskip

Nos exemplos anteriores vimos situações em que parênteses esticaram ou
encolheram, junto com o conteúdo que envolvem. Para obter esse efeito,
usamos os comandos \verb'\left' e~\verb'\right' (sempre aos pares),
seguidos do caractere que se deseja expandir (colchetes, chaves, barra
vertical ou parênteses). O exemplo a seguir já exibe um ou outro
requinte a mais. 

\medskip
\begin{center}\hrule\smallskip
\begin{tabular}{c|c}
\begin{minipage}{.405\textwidth}\footnotesize
\verbatiminput{exemplos/09-sub-e-superscripts-math-04}
\end{minipage} &
\begin{minipage}{.535\textwidth}\setlength{\parindent}{1pc}
\[ 
\left(\int f\right)
\stackrel{\textrm{def}}{=}
\left( \int_0^{+\infty} 
\!\!\!f(x)\,\textrm{d}x \right)
\]


\end{minipage}
\end{tabular}
\smallskip\hrule
\end{center}
\medskip

Primeiro, usamos \verb'\textrm' para que o `d' em d$x$ (e o ``def''
em~$\stackrel{\textrm{def}}=$) seja escrito 
em texto romano. O comando \macroCall{stackrel} \emph{empilha}
pequenos textos, e trata o símbolo resultante como um operador
relacional (maior, menor, menor ou igual e etc.). Usamos ainda os
comandos `\verb'\!' e~`\verb'\,'', correspondentes aos comandos
'\verb'\negthinspace'' e '\verb'\thinspace'' em modo texto,
respectivamente, usados (também respectivamente) para aproximar ou
afastar elementos do texto horizontalmente\footnote{Comumente dizemos
  que esses comandos   \emph{produzem espaço horizontal}:
  \macroCall{negthinspace} produz   espaço fino negativo,
  e~\macroCall{thinspace} produz um espaço fino.} %%
%%(mais sobre espaços na seção~\ref{sec:medidas})
.

Você pode obter letras gregas no modo matemátio facilmente. Vários
outros símbolos estão disponíveis, e a internet é sua amiga para
encontrá-los.

\medskip
\begin{center}\hrule\smallskip
\begin{tabular}{c|c}
\begin{minipage}{.405\textwidth}\footnotesize
\verbatiminput{exemplos/09-sub-e-superscripts-math-05}
\end{minipage} &
\begin{minipage}{.535\textwidth}\setlength{\parindent}{1pc}
\[
\partial\delta\alpha\beta\gamma
 > \Gamma
 < \epsilon
 \geq\varepsilon
\]

\[ \leq\psi=\sim\neq\leq\geq\in\notin
\cap\cup\oplus\cdot\times\div/\equiv \]

\[ \forall\exists\mapsto\Rightarrow
\longleftrightarrow\nu \]

\end{minipage}
\end{tabular}
\smallskip\hrule
\end{center}
\medskip

E isto é apenas parte do que se pode fazer com o \LaTeX, apenas
tocamos a superfície. Uma boa
referência é~\cite{graetzer00}.

% $$\vec{\nu}+f\overbrace{(a_1a_2\ldots a_n)}^{\hbox{$n$ primo}}$$
% ensuremath

\section{Aspectos estruturais}

Chegou a hora de aproximar-mo-nos um pouco mais do \LaTeX,
conhecer-lhe melhor os trejeitos e manhas, falar mais a sua
língua. Esse conhecimento é valioso quando quisermos convencê-lo a
fazer algo diferente para nós.

%% \subsection{A construção da página}

%% Knuth descreve o processo pelo qual o \TeX\ compõe o texto fazendo uma
%% analogia entre as etapas do algoritmo e a digestão. O sistema teria
%% olhos, boca, estômago, e intestinos (que soltam ao final a nossa obra).
%% Não nos cabe aqui abordar os detalhes sórdidos do que se passa nas
%% entranhas do sistema. Trataremos do grande projeto da página, e de
%% como o \LaTeX\ move informação para cá e para lá. Mas é bom ter em
%% mente de onde viemos, nem que seja para levar o sistema ao médico
%% certo quando alguma coisa entalar.

%% clearpage, newpage

%% \subsubsection{Anatomia da página}

%% notas de rodapé, notas marginais, cabeçalho, número de página

\subsection{Parágrafos marginais}
\newcommand{\amounttorotate}{0}\newlength{\recuo}
Usar \marginpar{\ifthenelse{\isodd{\thepage}}{\raggedright}{\raggedleft}\footnotesize notas marginais} notas
marginais no texto pode ser uma maneira interessante de destacar algum
conceito. O comando \verb'\marginpar{parágrafo}' acrescenta um
parágrafo à margem do parágrafo atual. 

É possível mudar drasticamente
a aparência de um parágrafo lateral (assim como de qualquer outro tipo
de parágrafo): diminuir a fonte em que é escrito, deixá-lo rasgado à
direita ou esquerda (seção~\ref{sec:alinhamento}),
envolvê-lo em uma caixa,
rotacioná-lo%
\marginpar{%
  \ifthenelse{\isodd{\thepage}}%
             {\raggedright\renewcommand{\amounttorotate}{-90}\setlength{\recuo}{-1em}}%
             {\raggedleft\renewcommand{\amounttorotate}{90}\setlength{\recuo}{-1em}}%
  \rotatebox{\amounttorotate}{\hspace{\recuo}\footnotesize\it $\mathcal{A}$ssim.}},
etc. --- em suma, qualquer transformação. Por exemplo, parágrafos de
páginas pares e ímpares são por padrão colocados de modo a que estejam
na lateral da folha que ficaria ``para fora'' caso o texto seja
encadernado. Esse comportamento, para ser mais preciso, depende de
algumas definições na classe do documento.\footnote{Por exemplo, se
  você está usando alguma  classe de documento padrão,
  como~\classedoc{article} ou~\classedoc{book}, a opção
  \parametro{twoside} implica que o documento será impresso
  frente-e-verso, o que geralmente implica que parágrafos marginais
  serão colocados à direita ou esquerda dependendo de a página a que
  pertencem ser par ou ímpar (a opção~\parametro{oneside} faz todo
  paragrafo marginal aparecer no mesmo lado da página).}


\begin{center}\footnotesize\hrule\nopagebreak\smallskip
\begin{tabular}{c|c}
\begin{minipage}{.47\textwidth}
\begin{verbatim}
Houve um tempo\footnote{Por volta de 1920.} 
em que as pessoas viviam como se estivessem 
na década de 20.
\end{verbatim}
\vfill
\end{minipage} &
\begin{minipage}{.47\textwidth}
Houve um tempo\footnote{Por volta de 1920.} em
que as pessoas viviam como se estivessem na 
década de 20.
\vspace*{1cm}
\end{minipage}\nobreak
\end{tabular}%
\nobreak\smallskip\nobreak\hrule
\end{center}


%% \subsubsection{Caixas}\label{sec:caixas}

%% \subsubsection{Medidas}\label{sec:medidas}

%% comprimentos  (medidas, medindo coisas, criando seus próprios
%% comprimentos)

%% espaços

%% \subsubsection{Encaixotando}

\subsection{Arquivos auxiliares}

O \LaTeX\ se vale de um bom número de arquivos auxiliares para
realizar seu trabalho. Tomemos um tempo para observar como funciona o
processo de uso de um arquivo auxiliar.

Algus desses arquivos são produzidos pelo próprio \LaTeX, durante a
compilação do documento. O índice do documento, suas listas de tabelas
e figuras, assim como vários outros, são criados para armazenar as
linhas de índices de elementos de certos tipos. Ao processar um
documento com índice, por exemplo, os números das páginas em que as
seções se iniciam são armazenados para posterior uso. Cada vez que o
texto passa pelo \LaTeX, os números de página mais recentemente
armazenados nos arquivos auxiliares são colocados nos índices.

Um processo parecido acontece com arquivos externos que são gerados
por programas como Bib\TeX\ (seção~\ref{sec:biblio}),
\programa{makeindex} e~\programa{makeglossaries}
(seção~\ref{sec:indice-glossario}). A diferença, então, é que
\emph{após} ser processado pelo \LaTeX, o
texto-fonte\index{texto-fonte} passa ainda por um dos demais, ou
ambos. E, depois, ainda deve ser \LaTeX ado mais duas vezes ao menos,
para que as referências sejam atualizadas.

%% \subsection{Contadores}\label{sec:contadores}

\subsection{Comandos frágeis}

Alguns comandos, como todos nós, precisam por vezes de atenção
especial, é preciso protegê-los. Você pode fazê-lo com
\macroCall{\protect}, que tem como argumento algum texto que precise
ser protegido.

Essa necessidade advém do fato que alguns fazem sentido apenas
se presentes em determinada parte do texto, e, se não protegidos,
podem ser inadvertidamente removidos de seu contexto-natal por outros
comandos.

Exemplos de comandos ``transportadores'' são \macroCall{section}
e~\macroCall{caption}, por exemplo. O texto que é passado como
parâmetro para essas sequências de controle não aparece apenas \emph{in loco},
mas são carregados para índices, listas de figuras, ou mesmo para o
cabeçalho da página.

Mas não entremos em detalhes ainda. O importante por agora é que,
havendo perigo à vista, pode ser necessária proteção.

\section{Bibliografia {\it \&} Cia.: Bib\TeX}\label{sec:biblio}

Veremos nesta seção duas abordagens para a composição de
bibliografias. Em uma delas, escrevemos a bibliografia linha por
linha, assim como escrevemos o texto. O \LaTeX\ automaticamente 
numera as entradas da bibliografia, e você pode referenciá-los com o
comando \macroCall{cite}.

Outro jeito, muito popular a propósito, de trabalhar com bibliografias
é usando o programa Bib\TeX. Nessa abordagem, as entradas
bibliográficas são escritas em um arquivo \extensao{bib}, seus campos
(autor, edição, editora, etc.) são marcados semanticamente, e a
formataçao é deixada a encargo do programa Bib\TeX\footnote{E pacotes
  que você porventura acrescente para configurar esse conportamento.}.

\subsection{Fazendo no muque}

O mecanismo original de composição de bibliografias pressupõe que elas
estejam postas em um ambiente próprio,
o~\ambiente{thebibliography}. Cada entrada possui opcionalmente um
rótulo público, que aparecerá entre colchetes quando for citada, e
ainda um rótulo interno, que funciona como os rótulos definidos com
\macroCall{ref}. Se nenhum rótulo público é fornecido, o
\LaTeX\ numera as entradas, e coloca ali o respectivo número.

Outra característica importante deste método é que as referências
aparecem exatamente na ordem em que foram declaradas, como seria de se
esperar. Isso não acontece, veremos, quando se usa o Bib\TeX, que
automatiza a ordenação dos itens da referência.

O processo de compilação do documento se altera quando se acrescenta
bibliografias em um documento, do mesmo modo como acontece quando
usam-se referências internas: o processamento do arquivo gera alguns
arquivos auxiliares, que são usados para escrever as citações.

\begin{verbatim}
\begin{thebibliography}{longuissimo}
\bibitem[Tahan83]{malba-tahan} TAHAN, Malba. \emph{O Homem que
Calculava}. Ed. Círculo do Livro. Edição integral. 1983.

\bibitem[Calvino03]{se-um-viajante} CALVINO, Ítalo. 
\emph{Se um Viajante numa Noite de Inverno}. Ed. Schwarcz. 2003.
\end{thebibliography}
\end{verbatim}

\begin{thebibliography}{longuissimo}
\bibitem[Tahan83]{malba-tahan} TAHAN, Malba. \emph{O Homem que
Calculava}. Ed. Círculo do Livro. Edição integral. 1983.

\bibitem[Calvino03]{se-um-viajante} CALVINO, Ítalo. 
\emph{Se um Viajante numa Noite de Inverno}. Ed. Schwarcz. 2003.
\end{thebibliography}

Vejamos o papel de cada um dos elementos no exemplo. {\tt
  longuissimo} é qualquer texto que tenha tamanho maior (ou igual) ao
rótulo mais longa entrada da bibliografia. Ele é usado pelo
\LaTeX\ para reservar espaço para os rótulos quando ele compõe os
itens da bibliografia.

Tanto {\tt Tahan83} quanto {\tt Calvino03} são rótulos que aparecerão,
por exemplo, quando usar o comando~\macroCallWithParameter{cite}{malba-tahan}.

\subsection{Bib\TeX}


bibliografia

citando

bibtex

\section{Índices remissivos {\it \&} Cia.:
  Makeindex}\label{sec:indice-glossario}

Glossários e índices remissivos.

\input{13-e-agora}
\section{Utilidades}

A partir de agora estás outorgado o título de \LaTeX nico! O que vem
adiante são apenas adendos ao teu cinto de utilidades, mas havendo
dominado o material até aqui, deves estar apto a enfrentar a maior
parte das quiméras tipográficas que o aventureiro compositor
encontrará em uma jornada habitual. No mais, não hesite em convocar a
comunidade, que não se fará surda a qualquer pedido de auxílio.

A lista abaixo contém algumas (poucas!) sugestões de pagotes que você pode achar
interessante investigar. Existem vários pacotes que possuem
finalidades parecidas, quando não idênticas --- fica à sua escolha. A
ideia é que vocÊ conheça um pouco do que dá para fazer com o \LaTeX, a
nova ferramenta no seu cinto de utilidades. Sem mais delongas, a
lista.

\begin{description}
  \item[hyperref] Cria hiperlinks dentro do
    próprio documento, além de controlar seu aspecto. Tem forte
    integração com a estrutura de documentos \extensao{pdf},
    permitindo controlar propriedades como \emph{autor},
    \emph{língua}, etc.
  \item[url] Cria o comando \verb'url', que encapsula páginas na
    internete faz uma quebra ``inteligente''.
  \item[xcolor] Deixa seu texto mais colorido!
  \item[fancyhdr] Personaliza o cabeçalho e rodapé de páginas,
    exibindo, por exemplo, a seção atual, nome do autor ou qualquer
    texto.
  \item[tikz] Desenhe figuras com texto! Visite
    \url{http://www.texample.net/} para ver do que ele é capaz.
  \item[beamer] uma classe de documento para compor slides.
  \item[amsmath] Pacote da \emph{American Mathematical Society} com
    vários comandos para facilitar a composição de expressões
    matemáticas.
  \item[a0poster] Pôsteres em a0!
  \item[microtype] Microtipografia.
  \item[memoir] Uma classe de documento que estende e aprimora
    grandemente as classes documento tradicionais, acrescentando uma
    série de outras categorias.
  \item[multicolumn] Permite usar um número variável de colunas no
    texto.
  \item[indentfirst] Recua a primeira linha do primeiro parágrafo de
    seções.
  \itema[belbib] Traduz palavras da bibliografia, como ``edição'',
  ``ano'', etc.
  \item[helvet] Permite usar a fonte Helvética no texto. 
\end{description}



\newpage 

\appendix

\input{fdl-1.3}
\newpage

\printindex%
\addcontentsline{toc}{section}{Índice Remissivo}
\newpage 

\bibliographystyle{babalpha}
\cleardoublepage \phantomsection
\addcontentsline{toc}{section}{Referências}
\bibliography{thebib}
\end{document}
