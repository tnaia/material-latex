\documentclass[%draft, 
a4paper,11pt,twoside]{article}

\usepackage[brazil,english]{babel}
\usepackage[utf8]{inputenc}
\usepackage[pdfpagelabels,colorlinks]{hyperref}
\hypersetup{colorlinks, 
           bookmarksopen=true,
           pdftex}
\usepackage{a4wide}
\usepackage{indentfirst}
%\usepackage[xindy]{glossaries}
%\makeglossaries

\usepackage{tikz}
%\usepackage{showidx}
\usepackage{makeidx}
\makeindex
\newcommand{\extensao}[1]{\texttt{#1}}
\newcommand{\pacote}[1]{\textsc{#1}}
\usepackage{verbatim}
\newsavebox{\mybox}
\newlength{\mydepth}
\newlength{\myheight}
%\newenvironment{detalhe}{\hfill\begin{minipage}{.8\textwidth}}{\end{minipage}}



\newenvironment{detalhe}%
{\medskip\begin{lrbox}{\mybox}\footnotesize\begin{minipage}{\textwidth}}%
{\end{minipage}\end{lrbox}%
\settodepth{\mydepth}{\usebox{\mybox}}%
\settoheight{\myheight}{\usebox{\mybox}}%
\addtolength{\myheight}{\mydepth}%
\noindent\makebox[0pt]{\hspace{-20pt}\rule[-\mydepth]{1pt}{\myheight}}%
\usebox{\mybox}\medskip}


\title{Mais uma apostila de \LaTeX}
\author{Tássio Naia dos Santos} % por enquanto


\title{Mais uma apostila de \LaTeX}
\author{Tássio Naia dos Santos} % por enquanto

% pdf meta-informação
\hypersetup{baseurl={http://www.ime.usp.br/~tassio/apostila.pdf},
  pdftitle={Mais uma apostila de LaTeX},
  pdfauthor={Tássio Naia dos Santos},
  pdfkeywords={LaTeX,TeX,tipografia,apostila,CCSL,PoliGNU},
  pdflang={pt-BR (Portuguese)},
  unicode=true}




\begin{document}
\maketitle
\thispagestyle{empty}
\clearpage
\section*{Sobre}

Este documento nasce como um material de apoio a oficinas de
\LaTeX. As oficinas são oferecidas pelo Grupo de Estudos de Software
Livre da Escola Politécnica da Universidade de São Paulo, o PoliGNU.
Contamos com o apoio do Instituto de Matemática e Estatística da
Universidade de São Paulo (IME).

\subsection*{Como usar esta apostila}

Como você bem entender. Este texto está licenciado sob a \emph{GNU Free
Documentation License} --- uma cópia está anexa ao fim deste documento).
Resumidamente, tens o direito de distribuir cópias deste documento,
com ou sem modificações, sob a única condição de mantê-lo licenciado
pela mesma licença.

\begin{detalhe}
Parágrafos que estejam com esta marcação contêm detalhes que talvez
sejam prescindíveis em uma primeira leitura. Falam de assuntos
marginais ao uso do \LaTeX, ou de tópicos que requerem alguma
\TeX nica (i.e., podem empregar conceitos que não são abordados até
um ponto mais adiantado do texto).
\end{detalhe}

\clearpage


\clearpage
\vspace*{.33\textheight}
\thispagestyle{empty}
\begin{flushright}
a José Augusto
\end{flushright}
\clearpage
%\begin{footnotesize}
\tableofcontents
%\end{footnotesize}

\clearpage
\section{Introdução}

\subsection{A metáfora}

\LaTeX\ se apoia fortemente numa certa relação entre apresentação e
conteúdo do texto, cujo
conhecimento pode poupar-nos (ou melhor, auxiliar-nos a lidar com)
algumas dores de cabeça: é o princípio de que o \emph{ritmo visual} de
um texto deve enfatizar sua estrutura. Por exemplo: a formatação
consistente de títulos de seções, destacando em que ponto se iniciam,
realça a coordenação entre os trechos que compõem o documento.

Esse pressuposto é válido para a vasta maioria dos escritos, em
particular livros convencionais, publicações de caráter técnico,
tais como relatórios, monografias,  relatórios, cartas, etc. A fatia
deixada de fora abarca produções caracterizadas por alguma
inconstância, defasagem intencional ou arritmia entre o conteúdo e a
formatação --- como trabalhos artísticos.


\subsection{Sinopse da Ópera}
% TeX e LaTeX (Knuth, Lamport, comunidade, uso)

Donald E.~Knuth criou \TeX, um sistema de tipografia digital muito~(!!)
poderoso, e extremamente flexível.

\begin{center}
\begin{minipage}{.75\textwidth}
  [\TeX\ is] \textit{a new typesetting system intended for the creation of
  beautiful books---and especially for books that contain a lot of
  mathematics.}

  \hfill Knuth---The \TeX book
\end{minipage}
\end{center}

  
Leslie Lamport criou o \LaTeX, que, a grosso modo, é uma interface
mais simplista para o uso do \TeX. Uma preocupação do \LaTeX\ é que,
ao usá-lo, tenhamos foco no conteúdo, na estrutura do que estamos a
compor. Busca separar as etapas de composição conceitual e visual do
texto.

Em contraposição ao modelo de edição de texto dos programas mais
populares hoje, em que 
\emph{o que você vê é o que você obtém}%
\footnote{Conhecido pela sigla em inglês \acronimo{wysiwyg}\index{wysiwyg@\acronimo{wysiwyg}},
  \emph{what you see is what you get}.} (ao menos deveria ser), ao usar
\LaTeX\ \emph{o que você vê é o que você quis dizer}%
\footnote{Do inglês: \emph{what you see is what you mean} (\acronimo{wyhiwym})\index{wysiwym@\acronimo{wysiwym}}.}.



\subsection{O que dá pra fazer}

Compor textos belíssimos. (E, por que não? Compor textos horrorosos.) Na
prática, veremos em breve, é simples produzir
documentos~\extensao{pdf}, \extensao{ps}, e~\extensao{dvi}; documentos
com diagramas (que podem ser desenhados usando o próprio sistema, ou
importando imagens~\extensao{jpg}, \extensao{eps}, \extensao{pdf},
etc.), tabelas, versos, referências bibliográficas, índices,
hiperlinks, e muitas outras coisas.

% colocar uma tabela com células (uma imagem importada, uma desenhada com o tikx, e texto)

\section{Rotina de trabalho}

Escrever um documento usando \LaTeX, não é muito diferente de escrever
um documento numa máquina de escrever, embora o resultado seja
bastante diverso. Em geral, você irá abrir um programa para edição
de texto%
\footnote{%
  Existem mesmo alguns programas sofisticadíssimos
  para a edição de documentos \LaTeX, mas este não é nosso foco
  aqui.}% todo: citar exemplos de TeXmakers da vida etc..
, escreverá o texto, e pedirá ao \LaTeX\ que gere o
documento \extensao{pdf} (ou~\extensao{ps}, ou~\extensao{dvi})que
desejar. Simples assim.

Não abordaremos aqui o processo de instalação do \LaTeX, ou como
preparar o seu computador para processar os arquivos \extensao{tex}. A
boa notícia é que essa é uma tarefa simples. Há várias páginas na
internet que explicam detalhadamente como instalar o programa,
independentemente de qual seja o sistema do seu computador. Abaixo seguem
alguns links de páginas que vale a pena visitar.

Certamente, vale a pena ler 
\begin{itemize}
\item \emph{\LaTeX, A Document Preparation System}, de Leslie Lamport
  (criador do \LaTeX),
  e
\item \emph{The \TeX book}, de Donald E.~Knuth (criador do \TeX, que é
  a base sobre o qual se assenta o \LaTeX).
\item Wiki brasileiro de \TeX: \url{www.tex-br.org}
\item Getting to Grips with \LaTeX, de Andrew Roberts: \url{http://www.andy-roberts.net/misc/latex/}
\item Apostila de \LaTeX\ da Universidade Federal Fluminense: \url{www.telecom.uff.br/pet/petws/downloads/apostilas/LaTeX.pdf}
\item \TeX\ Users Group: \url{www.tug.org}
\item Comprehensive \TeX\ Archive Network: \url{www.ctan.org}
\end{itemize}

\section{Primeiro documento}


\subsection{Texto e sequências de controle}\label{sec:seq-controle}

Quando você escreve um texto~\LaTeX, a maior parte do tempo você está
escrevendo como se usasse uma máquina de escrever comum (talvez você nunca tenha usado uma, mas provavelmente pode imaginar como é). Mas não todo o tempo. 

Uma primeira diferença das máquinas de escrever é o espaçamento. Muitos textos são feitos de modo que formem retângulos na página (o afamado \emph{alinhamento justificado}. Mas para que as linhas tenham o mesmo comprimento, é preciso hifenar\footnote{Vide seção~\ref{sec:hifenacao}.} algumas palavras (ou seja quebrá-las), e ainda alargar ou comprimir ligeiramente o espaço entre as palavras (essa é uma tarefa complicada, que o \LaTeX\ desempenha exemplarmente). Não é surpresa que o sistema tenha um modo diferente de lidar com o espaçamento que colocamos no texto do que outros sistemas. Por exemplo, colocar um espaço entre palavras faz com que elas fiquem separadas por um espaço (até aí, nenhuma surpresa). Mas colocar dois, três, ou cinquenta espaços entre um par de palavras tem o mesmo efeito que colocar apenas um. E mais: quebrar a linha no texto não causa uma quebra de linha no texto final. Observe atentamente o exemplo abaixo. O caractere `\texttt{\textvisiblespace}' indica um espaço em branco.

\medskip
\begin{center}\hrule\smallskip
\begin{tabular}{c|c}
\begin{minipage}{.405\textwidth}\footnotesize
\verbatiminput*{exemplos/03-espacos-01}
\end{minipage} &
\begin{minipage}{.535\textwidth}\setlength{\parindent}{1pc}
Separado por
um espaço

Separado por um espaço

Separado por  dois espaços

Separado por        vários espaços!

Separado por 
                    ai caramba!

Duas quebras de linha seguidas
(criando uma linha em branco)
iniciam um novo parágrafo, o que não 
acontece quando há apenas uma.
Se quiser forçar uma quebra de linha\\
Existe o comando barra-barra.

\end{minipage}
\end{tabular}
\smallskip\hrule
\end{center}
\medskip

Até o momento, falamos de texto puro e simples. Caracteres e espaços. Eventualmente, no entanto, você desejará acrescentar algo ao
texto além de palavras. Pode ser que queira \emph{enfatizar alguma
  passagem}, ou

\begin{quote}
  ``\ldots \textsl{citar algo que, alguma vez, muito apropriadamente, foi
    dito ou escrito, e que ilustra bem o que quer que seja.''}

  \hfill\textsl{Autor Conhecido}
\end{quote}

Em situações como essas, empregam-se \emph{sequências de controle},
que especificam o papel de alguma palavra, região ou ponto do texto.

Por exemplo, empreguei uma palavra de controle (\emph{control word\/})
pouco acima, para dizer ao \LaTeX\ que ``Texto e sequências de
controle'' é um título de seção. Sabendo disso,  o sistema pode fazer
várias coisas, como
\begin{enumerate}
\item descobrir o número da seção,
\item alterar o tamanho e peso da fonte empregada para escrever as
  palavras do título (com o número da seção ao lado), e
\item acrescentar uma linha ao sumário do texto com o número da página
  em que a seção começa.
\end{enumerate}

Sequências de controle iniciam por uma barra `\verb|\|'. A maior parte
delas, que chamamos \emph{palavras de controle}, são formadas pela
barra seguida por letras\index{letras} (consideramos aqui letras os
caracteres `\texttt{A}' a `\texttt{Z}', e `\texttt{a}' a
`\texttt{z}'). Há um outro tipo
de sequência de controle, que chamaremos aqui de \emph{caractere de controle}
(control character), que consiste de uma barra seguida de um caractere
não-letra, por exemplo `\verb|\-|', e `\verb|\{|' (a função dessas
sequências será explicada nas seções \ref{sec:hifenacao} e
\ref{sec:matematica}).

Naturalmente, surge a pergunta: mas e se eu quiser usar uma
\textbackslash\ no meu texto? De fato, se você digitar
``\verb|amigo\inimigo|'' para obter amigo\textbackslash inimigo, terá
uma surpresa: muito provavelmente o \LaTeX\ reclamará de uma
\verb!undefined control sequence \inimigo!. Veremos mais adiante que
alguns caracteres são ``reservados'' pelo \LaTeX\ para algumas funções
especiais. Alguns exemplos são os caracteres `\%', `\$' e `\_', além,
claro, do nosso amigo `\verb|\|'. Se você deseja usá-los no seu texto,
será preciso usar alguma sequência de controle que os coloque lá. A
propósito, as sequências de controle necessárias para esses caracteres
em particular são

\begin{center}
  `\verb|\%|' \ para \ `\%'%
  \qquad`\verb|\$|' \ para  \ `\$'%
  \qquad`\verb|\_|\negthinspace' \ para \ `\_'%
  \qquad`\verb|\textbackslash|' \ para \ `\textbackslash'\qquad
\end{center}

\subsection{Um documento simples}

Um texto preparado para o \LaTeX\ em geral é precedido por um
\emph{preâmbulo}, em que geralmente são descritas características do
texto (por exemplo, se ele é uma carta, um livro, um relatório; quem é
o seu autor; se o documento será impresso frente e verso, ou se apenas
uma página por folha.

O trecho abaixo tem três sequências de controle. Vejamos o que
significam. 

\begin{footnotesize}
\begin{verbatim}
\documentclass{article}
\begin{document}

Olá mundo! % Colocar um conteúdo de verdade.

\end{document}
\end{verbatim}
\end{footnotesize}

Primeiro definimos a \emph{classe} do documento, com a
sequência de controle \verb|\documentclass|. Essa sequência requer um
parâmetro, (qual a classe do documento, no caso \verb!article!) que é
posto entre chaves. Teremos mais a falar sobre parâmetros, ou
\emph{argumentos} daqui a pouco.

A classe \verb!article!, define uma série
de coisas, como o tamanho das margens e a formatação de muitos
elementos do texto, p.~ex., a formatação dos números das páginas. Outras
classes comumente usadas incluem \verb!letter!, para cartas,
\verb!beamer! para apresentações de slides, \verb!report! para
relatórios, \verb!book! para livros, \verb!a0poster! para pôsteres em
A0, etc. Há vários outros, como p.~ex., modelos para teses
disponibilizados por universidades, muitos dos quais se pode obter
gratuitamente na internet.

A seguir, demarca-se o início do documento propriamente dito. O par de
sequências de controle \verb!\begin! e \verb!\end! delimita uma
\emph{região} (falaremos mais delas em breve). Aqui, a região é o
próprio documento, seu conteúdo visível. Assim,
\verb!\begin{document}! delimita o início de uma região do tipo
\emph{document}, que é encerrada por \verb!\end{document}!.

Finalmente, o conteúdo do documento: a frase ``Olá mundo!'', seguida
de um \emph{comentário}. Se você é um programador, a noção de
comentário (como aliás muitas outras que abordaremos aqui)  deve
ser-lhe bem familiar. Em nosso exemplo, o comentário é 

\begin{center}
  \textit{Colocar um conteúdo de verdade.}
\end{center}

Comentários iniciam-se por um caractere `\verb.%.', e vão até o fim da
linha. Eles são ignorados pelo \LaTeX: são anotações no texto que o autor pode fazer para lembrar-se de algo ou temporariamente remover um trecho do texto, por exemplo.

Um detalhe importante: todo o texto que faz parte do comentário é como se não existisse para o \LaTeX\ quando ele processa o texto: tudo o que está entre o caractere `\verb`%`' e a primeira quebra de linha é ignorado, incluindo o caractere de porcentagem e a quebra de linha!

\medskip
\begin{center}\hrule\smallskip
\begin{tabular}{c|c}
\begin{minipage}{.405\textwidth}\footnotesize
\verbatiminput{exemplos/03-comentario-e-quebra-de-linha-01}
\end{minipage} &
\begin{minipage}{.535\textwidth}
Um comentário pode para inibir o%

início de um parágrafo novo, já q%
ue engole a quebra de linha também.

\end{minipage}
\end{tabular}
\smallskip\hrule
\end{center}
\medskip


\subsection{Parâmetros}

As sequências de controle (também chamadas aqui de
\emph{comandos}\index{comando}) encontradas,  até agora foram
sempre seguidas de algum texto entre chaves. Em \LaTeX, as chaves
servem para agrupar coisas, para que sejam vistas como uma unidade
só. 

De modo geral, sequências de controle operam de acordo com os
parâmetros, ou argumentos, que passamos para elas. Se uma sequência
emprega um certo número de parâmetros (digamos, 2), ela considera que
eles são os (dois) agrupamentos imediatamente depois dela. Mas
atenção: o \LaTeX\ sempre considera \emph{agrupamento} a menor unidade indivisível
que encontra ao ler um texto! Letras que não estejam  envolvidas em
chaves são, cada uma, um elemento diferente, assim como sequências de
controle o são. Por outro lado, um texto envolvido entre chaves conta
como um único agrupamento, um único elemento.

Por exemplo, suponhamos que haja um comando \verb!\importante! para
destacar texto, que opere sobre um único parâmetro (o texto
importante). O que cada uma das linhas a seguir destaca?

\begin{footnotesize}
\begin{verbatim}
\importante Lembre-se de usar chaves!
\importante{fazer as compras}
\importante{Destacar textos {importantes}}
\end{verbatim}
\end{footnotesize}

Respostas: (Você tentou os exercícios? Vá lá, mais uma chance!)
Respectivamente: ``L''; ``fazer as compras'', e ``Destacar textos {importantes}''.

Comandos nem sempre precisam de argumentos. Por exemplo,
\verb!\newpage! termina a página atual e continua o texto na página
seguinte, e \verb!\maketitle! mostra o título, autor e data do texto.

\subsection{Regiões}

Você já deve ter reparado que há uma certa ``anatomia'' no
texto. Por exemplo, há imagens, citações, tabelas, poemas, listas,
enumerações, e descrições, só para citar alguns. Todos são 
elementos de natureza diferente do texto, tanto visual como
conceitualmente.

Essas regiões, também chamadas de \emph{ambientes}, são trechos do
texto que têm um papel diferente, e, assim, provavelmente demandam um
tratamento diferente.

Já usamos regiões uma vez nesta apostila: o corpo do texto, o
\emph{document}, onde vivem seus elementos visíveis. Neste ponto, você
já deve imaginar como fazer para delimitar um ambiente. Digamos, que
uma parte de nosso relatório seja pura magia. Para que isso seja de
fato incorporado ao texto, basta fazer:

\begin{footnotesize}
\begin{verbatim}
\begin{pura-magia}
Chirrin-chirrion!
\end{pura-magia}
\end{verbatim}
\end{footnotesize}


\subsection{Acentuação: para além do ASCII}\label{sec:ascii}

Ao experimentar os exemplos dados até agora (se não fez, esta é uma
boa oportunidade! Tente gerar documentos a partir dos exemplos, eu
fico aqui esperando) você deve ter reparado que os caracteres
acentuados não aparecem no documento final. Mas experimente o seguinte
\begin{footnotesize}
\begin{verbatim}
\documentclass{article}
\begin{document}
Ol\'a mundo! Voc\^e come\c cou a notar algo?
\end{document}
\end{verbatim}
\end{footnotesize}

Não desespere. Acentuar é muito mais fácil do que isso. Tentemos outra
coisa

\begin{footnotesize}
\begin{verbatim}
\documentclass{article}
\usepackage[utf8]{inputenc}
\begin{document}
Olá mundo! Você começou a notar algo?
\end{document}
\end{verbatim}
\end{footnotesize}

Qual o resultado? E se você tentar o seguinte?
\begin{footnotesize}
\begin{verbatim}
\documentclass{article}
\usepackage[T1]{fontenc}
\begin{document}
Olá mundo! Você começou a notar algo?
\end{document}
\end{verbatim}
\end{footnotesize}

Uma das alternativas acima deve solucionar a questão dos acentos em
seu computador, a depender de como estão armazendas as letras no
seu texto. Mais precisamente, cada uma das linhas novas, que começam
por \verb'\usepackage', tenta dizer ao \LaTeX\ como interpretar a
\emph{codificação} do arquivo que ele irá processar.

\begin{detalhe}
O leitor atento poderá se perguntar: mas o texto que salvei é
\emph{puro}\footnote{Usamos aqui \emph{texto puro} como tradução da
  expressão em inglês \emph{plain text}: texto sem formatação.}, sem
formatação alguma --- como ele pode ser armazenado de mais de um modo?
quem determina que codificação o arquivo tem? 
A resposta direta a essa pergunta é a seguinte: arquivos são
armazenados como sequências de zeros e uns no computador (ao menos até
este momento, em 2010). A \emph{codificação} de um arquivo é o
conjunto de regras que associa a determinadas sequências de zeros e
uns a cada uma das letras de um texto.
\end{detalhe}

Apesar de os comandos para acentuação serem dispensáveis na maioria
dos casos, há situações em que pode ser útil saber um truque ou
outro. Principalmente quando o que se deseja é escrever algum nome
estrangeiro em algum ponto particular do texto, e não se sabe como
obter o caractere a partir do seu teclado.

O trecho a seguir é um excerto do \TeX book.

\medskip
\begin{center}\hrule\smallskip
\begin{tabular}{c|c}
\begin{minipage}{.405\textwidth}\footnotesize
\verbatiminput{exemplos/03-verbatim-example-03}
\end{minipage} &
\begin{minipage}{.535\textwidth}
Erd\"os, B\=askara, Gabor Szeg\"o.

`\`o' (grave accent)
`\'o' (acute accent)
`\^o' (circumflex or “hat”)
`\"o' (umlaut or dieresis)
`\~o' (tilde or “squiggle”)
`\=o' (macron or “bar”)
`\.o' (dot accent)
`\u o' (breve accent)
`\v o' (há\v cek or “check”)
`\H o' (long Hungarian umlaut)
`\t oo' (tie-after accent)
`\c o' (cedilla)
`\d o' (dot-under accent)
`\b o' (bar-under accent)
`\oe',`\OE' (French ligature OE)
`\ae',`\AE' (Latin and 
             Scandinavian ligature AE)
`\aa,\AA' (Scandinavian A-with-circle)
`\o',`\O' (Scandinavian O-with-slash)
`\l',`\L' (Polish suppressed-L)
`\ss' (German “es-zet” or sharp S)

\end{minipage}
\end{tabular}
\smallskip\hrule
\end{center}
\medskip


Mas o que faz o comando `\verb'\usepackage''? Veremos a seguir.

\subsection{Pacotes}

Uma característica importantíssima do \LaTeX\  é sua
expansibilidade, que permite que ele se adapte às necessidades
dos mais variados usuários. Assim como é possível estender as
capacidades de um programa acrescentando-lhe `plugins', `add-ons', ou,
em mais baixo-nível, bibliotecas, é possível dotar o \LaTeX\ de mais
comandos, pela inclusão de \emph{pacotes}.

Pacotes são documentos de texto (como os que você escreve ao seguir
esta apostila). Certo, eles não são \emph{exatamente} documentos de
texto como os que você escreve agora: os pacotes possuem diversas
definições de comandos, macros e ambientes, que agregam funcionalidade
ao \LaTeX. Pacotes têm muitas vezes a extensão \extensao{sty}, embora
você não precise se preocupar com esse detalhe (ao menos enquanto você
não estiver escrevendo seus próprios pacotes, ou investigando as
fascinantes entranhas do sistema).

Para usar um pacote, basta usar o comando \verb'\usepackage'. Esse
comando faz com que o \LaTeX\ procure pelo arquivo do pacote e torne
sua funcionalidade disponível para que você dela disponha como quiser.
O argumento do comando é o nome do pacote. Pouco atrás usamos um
comando para poder usar acentos em arquivos codificados em \extensao{utf8}.
\begin{center}\footnotesize
\begin{verbatim}
\usepackage[utf8]{inputenc}
\end{verbatim}
\end{center}

Este comando tem um \emph{parâmetro opcional},
\texttt{utf8}. Parâmetros opcionais estão presentes em vários
comandos. Um parâmetro opcional pode ser omitido; ele geralmente
representa alguma configuração ou pequena alteração no modo de
funcionamento do comando.

Assim, é comum que pacotes possam ser configurados por meio de
parâmetros opcionais passados a eles.

Da mesma maneira, classes de documento também podem ser configuradas
por meio da passagem de parâmetros opcionais. Alguns exemplos: pode-se
passar os parâmetros opcionais \texttt{11pt}, \texttt{twocolumn},
\texttt{twoside}, \texttt{draft} para a declaração da classe
\texttt{article}. Assim, para um documento a ser impresso
frente-e-verso, em duas colunas, podemos escrever
\begin{footnotesize}
\begin{verbatim}
\documentclass[twocolumn,twoside]{article}
\begin{document}
...
\end{document}
\end{verbatim}
\end{footnotesize}

É importante notar que separamos os parâmetros opcionais por
vírgulas. Isso acontece para comandos como \texttt{documentclass} e
\texttt{usepackage}, mas não é válido para outros comandos (vide
seção~\ref{sec:comandos}).

Há pacotes para as mais diversas coisas: acrescentar cor ao texto,
usar capitulares (letras grandes, muitas vezes cheias de adornos, no
início de parágrafos), para descrever palavras-cruzadas, jogos de
xadrez, para desenhar, para fazer tabelas grandes, colocar trechos de
texto em números variáveis de colunas, acrescentar marcas d'água,
personalizar cabeçalhos e rodapés, e mesmo
``meta-pacotes.''\footnote{Pacotes que auxiliam a escrita de outros
  pacotes. Esses pacotes geralmente são de um gênero mais técnico,
  parecendo às vezes ``coisa de programador.''}

\subsection{Caracteres reservados}

São dez os caracteres reservados pelo \LaTeX\ para funções especiais
(ou seja, é preciso alguma ginástica para obtê-los). Eles são os seguintes.

\begin{center}
\verb'\   _   ^   ~   &   #   {   }   %   $'
\end{center}

A barra marca o início de um comando; o ``underscore'' e o circumflexo
são usados no modo matemático (seção~\ref{sec:matematica}); o ``e
comercial'' é usado em tabelas (seção~\ref{sec:tabelas}); o ``jogo da
velha'' é usado na definição de comandos (seção~\ref{sec:comandos});
as chaves agrupam texto; o caractere de porcentagem marca o início de
comentários; e o cifrão delimita o modo matemático.

Esses caracteres podem ser usados em um documento prefixando-os por
uma barra.

\medskip
\begin{center}\footnotesize\hrule\smallskip
\begin{tabular}{c|c}
\begin{minipage}{.465\textwidth}
\verbatiminput{exemplos/03-verbatim-example-05}
\end{minipage} &
\begin{minipage}{.465\textwidth}
\centering \_  \^{}  \~{}   \&  \#   \{   \}   %   $

\end{minipage}
\end{tabular}
\smallskip\hrule
\end{center}
\medskip

A exceção é a barra, que pode ser obtida  por meio do
comando \verb'\textbackslash'.

\subsection{Palavras de controle e texto}\label{sec:palavras-de-controle}

Para construir um texto usamos aqui nada mais que palavras e comandos,
simplesmente. Nesta seção veremos como eles se coordenam.

Durante o processamento de seu texto, a maior parte do tempo o
\LaTeX\ apenas encontra letras comuns, que prepara para colocar em um
parágrafo. Algumas vezes, no entanto, ele encontra uma barra --- o que
significa que uma sequência de controle foi encontrada. Se o caractere
seguinte não for uma letra, trata-se de um 
\emph{caractere de controle} (vide seção \ref{sec:seq-controle}), e
o \LaTeX\ continua processando o texto, levando em conta, claro, o
significado do comando que encontrou. Já se após a barra há uma
letra, o sistema se prepara para ler uma palavra de controle: continua
a ler caracteres do texto até encontrar o primeiro caractere que não
seja uma letra\footnote{Lembre-se: letras são os caracteres de
  \texttt{a} a \texttt{z} e de \texttt{A} a \texttt{Z}.}. Se a palavra
de controle é seguida de espaços em branco, \emph{eles são
  ignorados}; e se ela é seguida de \emph{uma} quebra de linha, ela é
ignorada também. O que acontece com mais quebras de linha?
Experimente! Exercício: como você faria para escrever \TeX emplo?

Se os espaços em branco são ignorados, como faço para que uma palavra
de controle como \LaTeX\ seja seguida por um espaço (como foi aqui)?
Os espaços são necessários após uma palavra de controle para definir
seu fim --- caso contrário, o \LaTeX\ consideraria que \verb'\TeXemplo' é
uma palavra de controle só. Mas qualquer coisa que permita ao sistema
identificar que a palavra de controle terminou serve para o mesmo
propósito. Assim, se você colocar um grupo vazio seguindo o comando
(`\verb'\TeX{} emplo''), ou colocar um grupo envolvendo o comando
(`\verb'{\TeX} emplo''), o espaço que segue o fim do grupo será
preservado. Há ainda um outro modo, mais simples, de colocar um espaço
logo depois de uma palavra de controle: basta usar o comando
`\verb'\ '', que é uma barra seguida de um espaço. Esse comando
simplesmente produz um espaço em branco, e podemos escrever
`\verb'\TeX\ emplo'' para obter o \TeX\ emplo.

\subsection{Texto sem formatação}

Por vezes o que queremos é que o texto digitado apareça exatamente
como o escrevemos. Veremos a seguir que o \LaTeX\ toma algumas
decisões por conta própria na hora de compor o texto, e os importantes
benefícios que esse comportamento traz consigo. Por hora, mencionemos
um importante exemplo: nem todo espaço no arquivo-fonte corresponderá
a um espaço na formatação final. Calma, as palavras não serão
coladas. Mas experimente usar dois espaços entre um par de palavras. O
que acontece\footnote{Não há resposta aqui \texttt{=)}.}?

Em algumas situações, como por exemplo em listagens de programas, pode
ser útil usar o \LaTeX\ como se ele não fosse mais do que uma máquina
de escrever digital. Queremos que o texto seja posto
\emph{verbatim}\index{verbatim}, isto é, exatamente como foi
escrito. Para isso, podemos usar (sic) o ambiente \verb'verbatim'.

\medskip
\begin{center}\footnotesize\hrule\smallskip
\begin{tabular}{c|c}
\begin{minipage}{.465\textwidth}
\verbatiminput{exemplos/03-verbatim-example-01}
\end{minipage} &
\begin{minipage}{.465\textwidth}
\begin{verbatim}
int main(int argc, char argv) {
  int resposta = 42;
  /* TODO: calcular a pergunta */
  return 0;
}
\end{verbatim}

\end{minipage}
\end{tabular}
\smallskip\hrule
\end{center}
\medskip

Há um outro método para ``cancelar'' a interpretação de caracteres,
para trechos menores, destinados a viver dentro de uma frase
comum. Por exemplo, as várias vezes em que me referi a comandos
\verb'\LaTeX', precisei fazer com que a interpretação do comando fosse
abortada (caso contrário, teria obtido \LaTeX). O comando que faz isso
é o \verb'\verb', que possui uma sintaxe especial. O comando é seguido
por um caractere qualquer (espaço vale!). Esse caractere servirá para
delimitar o fim do argumento de \texttt{\char`\\{}verb}. A esse
caractere se segue o texto a ser ``verbatimizado,'' que é todo o texto
até a próxima ocorrência do delimitador. Exemplo:
`\texttt{\char`\\{}verb!\char`\\LaTeX!}' resulta em `\verb!\LaTeX!',
mas `\verb'\LaTeX'' resulta `\LaTeX'.

Um último comentário. Tanto o comando quanto o ambiente verbatim
possuem uma versão ``estrelada'', que exibe os espaços em branco
\verb*'deste jeito aqui'. O ambiente é chamado \verb'verbatim*' e o comando
\texttt{\char`\\{}verb*}. 

\subsection{Alguma tipografia}

Já dissemos que o \LaTeX\ tem um jeito particular de dispor o texto
que escrevemos. Veremos agora que história é essa.

\medskip
\noindent\begin{minipage}{\textwidth}
\begin{center}\footnotesize\hrule\smallskip
\begin{tabular}{c|c}
\begin{minipage}{.465\textwidth}
\verbatiminput{exemplos/03-verbatim-example-02}
\end{minipage} &
\begin{minipage}{.465\textwidth}
As grandiloquência exibicionista são 
pouco persuasiva para aqueles honestamente 
curioso.
   
Verdade                           isso. 
Para         quem         já tanto 
circunvaga o sentido, cheio de dedos no 
pântano dos significados, um pouco de tento 
com o que passa a ser floreio decorativo é 
no mínimo cortês.

E tudo \LaTeX ado apropriadamente. 
\emph{Muito} apropriadamente.
Usando alguns comandos \LaTeX\ que já foram 
vistos\dots.

\end{minipage}
\end{tabular}
\smallskip\hrule
\end{center}
\end{minipage}
\medskip

Algo que salta à vista de primeira é que as quebras de linha não são
respeitadas. Também parece que os espaços a mais são
desconsiderados\dots e a realidade não está mesmo longe disso: um
espaço ou vários espaços são a mesma coisa para o \LaTeX. Uma (única)
quebra também é equivalente a um espaço. Duas quebras de linha, por
outro lado, fazem com que um novo parágrafo seja iniciado.

Notável também é o fato de que o primeiro
parágrafo\index{paragrafo@parágrafo} não tem recuo, enquanto que os
demais o têm. Isto se deve ao fato de que para algumas culturas (em
particular na tipografia de língua inglesa), não é costumeiro marcar a
primeira linha de um parágrafo com recuo a menos que este seja
precedido imediatamente por outro parágrafo. Afinal, esse recuo tem
por objetivo facilitar a identificação visual do novo parágrafo, o que
não é necessário se o parágrafo é o primeiro de uma seção ou capítulo
do texto, por exemplo.

Encontramos também os comandos \verb'\LaTeX', que
escreve \LaTeX, e \verb'\emph', que \emph{enfatiza} o texto que lhe é
passado como parâmetro. Note que o que o comando faz é enfatizar: o
jeito como ele faz isso não é a nossa preocupação nesse momento.

\begin{center}
\it O que importa aqui é que o trecho tem que ser destacado.  
\end{center}

E isso é diferente de dizer que o texto deve ser posto em negrito, ser
sublinhado, ser escrito em fúcsia, \reflectbox{ou} \reflectbox{de}
\reflectbox{algum} \reflectbox{jeito} \reflectbox{estranho}. Afinal, o
paradigma aqui é que a aparência do texto refletirá a função, o papel
semântico desempenhado por cada um de seus elementos. Assim,
descreve-se num primeiro momento o que cada um
\emph{significa}\index{marcacao semantica@marcação semântica},
deixando-se para outra etapa (quando pertinente) o ajuste do modo pelo
qual essa função é realçada visualmente.

\LaTeX\ lida com uma granularidade maior de conceitos do que comumente
nos é dado controlar em ambientes usuais de edição de texto; conceitos
que, a princípio, podem surpreender os não iniciados ao universo dos
cuidados tipográficos. A partir de agora, e à medida que adquire
experiência com um sistema tipográfico de alta qualidade como o
\LaTeX, você notará uma série de mudanças na percepção de um texto. Seu
vocabulário vai crescer, seus olhos e atenção serão exercitados em
novas direções, e muito provavelmente você se surpreenderá com a
influência que ``detalhes'' têm no ritmo e facilidade de leitura de um
texto. Mãos à obra!

\subsubsection{Hífens e hifenação}\label{sec:hifenacao}

Muito embora haja apenas um tipo de hífen em seu teclado, existem
muito mais hífens na tipografia. Há aquele usado em palavras
compostas, como ``guarda-chuva'' ou ainda ``resguardar-se'', e que também
servem para marcar a quebra de uma palavra no fim de uma linha
(sua \emph{hifenação}); há o traço usado para indicar um intervalo de números,
por exemplo 12--14; há o travessão --- o mais longo entre os hífens; e
há o sinal de menos, usado em equações, como em $20-3=17$. É fácil
produzir cada um desses símbolos em \LaTeX.

\begin{itemize}\footnotesize
\item \verb'guarda-chuva', \verb'resguardar-se'
\item \verb'exercícios das páginas 12--14' 
\item \verb'no dia de hoje --- véspera de amanhã'
\item \verb'diga-me também que $2-2=5$, Winston'
\end{itemize}

O último dos exemplos acima introduz o chamado \emph{modo
  matemático}\index{modo matematico@modo matemático}, assunto da seção
\ref{sec:matematica}.

Mas há ainda o que falar sobre hifenação. Na maior parte dos casos, o
\LaTeX sabe hifenar corretamente as palavras de diversos idiomas (o
portugês entre eles). Para isso basta usar o pacote \pacote{babel},
passando como parâmetro \parametro{brazil}. Algumas vezes, porém,
usamos termos que possuem uma hifenação pouco comum, ou usamos
palavras que o \LaTeX não consegue hifenar a contento. Quando isso
ocorre, podemos dizer explícitamente em que pontos uma palavra pode
ser hifenada. Há dois modos de fazê-lo: pode-se, no preâmbulo,
adicionar um comando \verb'\hyphenation', que leva como parâmetro uma
lista de hifenações, separadas por espaços, como abaixo. Note que não
se podem usar comandos ou caracteres especiais no argumento do comando. 

\begin{center}
  \verb'\hyphenation{FNAC A-bra-cur-six}'
\end{center}

No exemplo acima, FNAC, fnac e Fnac não serão jamais hifenadas, ao
passo que Abracursix e abracursix o serão, segundo os hífens
especificados.

Outro modo é explicar onde uma determinada ocorrência de uma palavra
pode ser hifenada, quando ela ocorre no texto. Nesse caso, a sugestão
de hifenação vale naquele ponto somente. O \LaTeX\ não se lembrará
dela se a palavra for usada novamente.

\medskip
\begin{center}\footnotesize\hrule\smallskip
\begin{tabular}{c|c}
\begin{minipage}{.465\textwidth}
\verbatiminput{exemplos/03-verbatim-example-04}
\end{minipage} &
\begin{minipage}{.465\textwidth}
É algo assim, como direi?
su\-per\-ca\-li\-frag\-i\-lis%
\-tic\-ex\-pi\-a\-li\-do\-cious

\end{minipage}
\end{tabular}
\smallskip\hrule
\end{center}
\medskip
 
\subsubsection[Ligaduras e Kerning]{Apurando os sentidos: ligaduras, kerning}% e história}

As letras por vezes requerem pequenas modificações no espaçamento
entre si, ou mesmo em sua forma, a depender dos símbolos que estão
próximos de si. Por exemplo, alguns pares de letras são aproximados,
enquanto outras vezes, partes de letras se fundem. Observe os exemplos
abaixo.

\medskip
\noindent\begin{center}%
\scalebox{3}[3]{f{i}}\hfil%
\scalebox{3}[3]{fi}\hfil%
\scalebox{3}[3]{T{a}}\hfil%
\scalebox{3}[3]{Ta}
\\[.9cm] 
\scalebox{3}[3]{s{t}}\hfil%
\scalebox{3}[3]{st}\hfil%
\scalebox{3}[3]{f{l}}\hfil%
\scalebox{3}[3]{fl}
\end{center}
\medskip

Ligaduras (do inglês, \emph{ligatures}), ocorrem quando um agrupamento
de letras é substituído por algum outro símbolo, quer para melhorar sua legibilidade, quer para tornar o texto mais belo.

Já o \emph{kerning} é um aumento ou diminuição do espaço entre letras,
que varia de acordo com o entorno de cada caractere.

\medskip
\noindent\begin{center}%
\scalebox{2}[2]{Uma {T}orta {P}ara {J}aiminho}

\scalebox{2}[2]{Uma Torta Para Jaiminho}

\medskip

%\hfil\scalebox{2}[2]{\textsc{{V}á}}%
%\scalebox{2}[2]{\textsc{Vá}}\hfil%
\scalebox{2}[2]{\textsc{{A}v{a}r{o}}}\hfil%
\scalebox{2}[2]{\textsc{Avaro}}%
\hfil\scalebox{2}[2]{\textsc{{P}a{r}a}}
\hfil\scalebox{2}[2]{\textsc{Para}}\hfil
\end{center}

\begin{comment}
  (breve) história do TeX, do LaTeX e irmãos
  Resumo de como o sistema ``monta'' as páginas
\end{comment}


\subsubsection{Sobre espaçamento horizontal}\label{sec:espacos}

Nem todos os espaços são iguais. Não só variam em tamanho, mas possuem
comportamentos distintos. Falaremos a seguir dos 
\emph{espaços duros}\index{espacos duros@espaços duros}
e de espaços um pouco mais largos, embora isso esteja longe de esgotar
o assunto.\footnote{Espaço preenchido e espaço em branco estão em
  constante interação em qualquer peça de composição visual.} Falaremos dos espaços mais comuns no texto, como os que separam palavras. Algumas vezes (principalmente quando abordarmos a escrita de expressões matemáticas), outros tipos de espaçamento serão necessários. 

Todo parágrafo justificado, isto é, que tem as margens direita e
esquerda alinhadas verticalmente, exige que o espaçamento entre palavras seja ``elástico'',
aumentando ou diminuindo conforme a necessidade. O \LaTeX\ possui um
mecanismo interno elaborado para o gerenciamento desses espaços (que não descreveremos aqui). Ainda
assim, é importante saber que alguns espaços são mais elásticos do que
outros, e que os espaços comuns possuem limites de compressão e
expansão.

Por exemplo, o espaço que segue o ponto final (ou a interrogação, ou a exclamação) em uma frase é mais
elástico que o espaço que une as demais palavras. Mas como o fim de
uma frase é identificado?

Por padrão, o \LaTeX\ assume que um ponto final --- ou outra pontuação
como `?{}', ou `!{}', ou `\ldots'
(reticências\index{reticencias@reticências} são produzidas usando o
comando \verb'\ldots') --- marca o fim de uma
frase sempre que, e somente quando, for precedida por uma letra
minúscula. Na maior parte dos casos esse comportamento é exatamente o
que precisamos, mas nem sempre.

Títulos como Dr.\ não terminam uma frase, na maior parte dos
casos. Isso é resolvido usando `\verb*'Dr.\ ''.\footnote{Como vimos na seção \ref{sec:palavras-de-controle}.}
Por outro lado, existem casos em que uma letra maiúscula seguida de
pontuação \emph{termina} uma frase: URSS\@. Para indicar que o ponto
final após uma letra maiúscula termina a frase, existe o comando `\verb'\@'\thinspace', correto?
\begin{center}\footnotesize
\verb'Entendi, OK\@. A frase terminou no último ponto. E nesse também. E nesses.'
\end{center}

Falaremos agora dos espaços \emph{duros}. Existem palavras que estão
naturalmente ligadas, e não toleram quebras de linha entre si. Isto
acontece em expressões como ``seção~\ref{sec:espacos}'',
``Dr.\\ House'' (viu?). Frequentemente é preciso prestar atenção a expressões como ``Teorema de~Kuratowski'', ``Associação Contra os~Maus-tratos a~Espécies'', em que nem todos os espaços são duros, mas alguns são.
Para produzir um espaço duro em \LaTeX, usa-se o til `\verb'~''. Por exemplo, ``\verb'5~cm'''. Com um pouco de prática se torna natural o a introdução desses espaços quando apropriado.



\section{Estruturando o texto}

Textos, assim como animais, possuem uma anatomia. Essa anatomia é o que permite ao leitor se localizar em sua leitura, identificar algo que procura. A estrutura do texto, além disso, carrega uma mensagem em si, ao menos em potencial, ao refletir o encadeamento do texto.

A depender da classe do documento, há uma certa variedade de tipos de segmentações à nossa disposição para organizar o texto.\footnote{E, como tudo o mais quando se trata da família \TeX, esse conjunto pode ser estendido e modificado como melhor nos aprouver.}
Artigos podem ser particionados em seções, subseções, subsubseções, apêndices.
Livros possuem, adicionalmente ao que está disponível em artigos, capítulos (contendo um certo número de seções).
Relatórios possuem adicionalmente (a livros) \emph{partes} (que contém capítulos).
E por aí vai.

Você pode mesmo criar seu próprio nível hierárquico, como parágrafos, como veremos na seção~\ref{sec:contadores}.

Neste capítulo, abordaremos, a título de exemplo, secionamento (segmentação) de um texto em artigos (documentos da classe~\pacote{article}). O comportamento apresentado em livros, relatórios etcétera é análogo, e em caso de dúvida basta recorrer aos manuais da respectiva classe (que, por padrão, vêm juntamente com o pacote quando a sua distribuição \LaTeX\ é instalada).

\subsection{Títulos, autor e data de documentos}

Em muitas classes de documentos, estão disponíveis os comandos para
definir o título, o(s) autor(es) e a data do documento. Cada classe
exibe essa informação de um modo, mas em boa parte delas você define o
título com um comando \verb'\title{Minhas Férias}', o autor usando o
comando \verb'\author{YoMoiIchEu}'. A data é composta automaticamente
com a data em que o documento for processado (no idioma do
documento). Você pode escolher (fixar) a data usando o comando
\verb'\date{Muito, muito tempo atrás}'.

Depois de especificados o título e o autor (mais de um autor pode ser
declarado, separando-se seus nomes por \verb'\and'), você escolhe o
ponto do texto no qual quer que apareçam, e usa o comando
\verb'\maketitle'. Voilà!

\subsection{Marcando a anatomia}\label{sec:comandos-de-secionamento}

O exemplo a seguir ilustra o uso de seções, subseções, subsubseções, seções não numeradas\index{secao@seção} e apêndices.


\medskip
{\footnotesize\verbatiminput{exemplos/05-01-sectioning}}

\medskip

Dizemos que uma seção inicia a partir do comando \verb'\section'. O
argumento que este comando leva é o título da seção. O mesmo acontece
para sub-seções, e as demais divisões do texto.

Seções são numeradas por padrão. Para obter uma seção, sub-seção, etc. não numerada,
use o respectivo comando em sua versão com asterisco, por
exemplo \verb'\section*{Prefácio}'.

A classe de documento e os pacotes que você usa definem quais os
comandos de secionamento disponíveis. Livros, por exemplo, têm
\verb'\chapter', relatórios têm \verb'\part', e por aí vai.

\subsection{Sumários}

Falemos agora do acompanhamento natural de um texto secionado:
sumários (ou índices). Fazer um sumário\index{sumario@sumário}, com o
\LaTeX\ é muito simples. Marque os títulos das partes usando os comandos de
secionamento que acabamos de ver, e, no ponto do texto em que deseja
acrescentar o índice, coloque o comando \verb'\tableofcontents'.

Uma vez marcadas as seções do texto e solicitado o índice, o
\LaTeX\ anota (em um arquivo auxiliar) as páginas em que começam as
seções do texto, à medida que o processa. Essas informações
são usadas para escrever o sumário. A depender da parte do
texto na qual sumário foi posto, pode ser necessário processar o texto duas vezes\footnote{Para sermos estritamente precisos, é possível construir documentos anômalos que ``ludibriam'' o índice e requerem mais processamento, mas não se preocupe: se seu texto tiver essa propriedade, certamente você a terá causado conscientemente.} (na primeira as páginas em que ocorrem as seções são anotadas, e
na segunda as entradas no sumário são atualizadas com os valores
corretos).
Além das informações escritas no arquivo auxiliar (que tem a
extensão~\extensao{aux}), o comando \verb'\tableofcontents' faz ser
gerado um outro arquivo, com a extensão \extensao{toc} (\emph{table of
  contents}), que contém o sumário em si. 

É possível --- e igualmente fácil --- gerar listas de figuras, tabelas
ou quaisquer outros elementos usando \LaTeX. Veremos como fazê-lo nas
seções~\ref{sec:floats} e~\ref{sec:contadores}.\footnote{Para os
  curiosos, a receita: basta usar  \texttt{\char`\\{}listoffigures}
  e~\texttt{\char`\\{}listoftables} em conjunto com os ambientes
  \ambiente{figure} e \ambiente{table}.}


Os comandos de secionamento possuem em geral um parâmetro opcional,
que é uma versão ``mais compacta'' do título, para ser usada em no
sumário (ou, por vezes, no cabeçalho ou rodapé de páginas).
\begin{footnotesize}
\begin{verbatim}
\section[Prova Documental]{%
  Documento provando a corretude do argumento %
  que concebi em uma longa insônia alcoolizada}
\end{verbatim}
\end{footnotesize}


\subsection{Referenciando elementos do texto}\label{sec:ref-e-label}

Assim como sumários são elementos importantes para a orientação do leitor-explorador, existem outros tipos de referências que ocorrem com frequência. Outro modo de remeter o leitor a um trecho, página --- em geral, a um \emph{elemento} qualquer --- do texto é usando \emph{referências}, que são o assunto desta seção.

Há vários tipos de referências, e várias maneiras de se referir a
alguma coisa. Podemos fazer referência a uma~\emph{figura} ou a
um~\emph{capítlo}; assim como podemos identificá-los por um número
próprio, ou pelo número da página em que se iniciam.\footnote{Outros
  tipos de referência incluem: referência a notas de rodapé e a
  elementos ``externos'', como itens de bibliografia, glossário e
  índices remissivos, que são explorados nas seções~\ref{sec:biblio} 
  e~\ref{sec:indice-glossario}.} Em sua essência, porém, existem
apenas dois componentes imprescindíveis em uma referência: um
\emph{indicador} e um \emph{indicado}.

Para referenciar algo em \LaTeX, usamos \emph{rótulos}. Rótulos são nomes que damos a algum elemento do texto. Para criar um rótulo, use o comando \verb'\label{nome do rotulo}'\index{label@\verb'\label'}, e para referenciá-lo use o comando \verb'\ref{nome do rotulo}'\index{ref@\verb'\ref'}. 

Quando usados em um ponto do texto, o label fica automaticamente associado à página, seção (subseção e etcétera) a que pertence aquele ponto no texto. Em enumerações, associa-se ainda ao item correspondente, e assim vale para figuras, tabelas e ambientes em geral.

Note que no exemplo o nome do rótulo não tem acentos. Quando criar seus rótulos, use apenas caracteres simples: mais especificamente, caracteres  \acronimo{ascii}\footnote{Entre os caracteres \acronimo{ascii} estão as letras de `\texttt{a}' a `\texttt{z}' (maiúsculas e minúsculas), os dígitos, o espaço em branco, assim como os caracteres ``\texttt{@\#\$\%\&*`'"!()-\char`\_=+[]\char`\{\char`\}\^{}\~{},.;/\char`\\|<>?}''.}.


\section{Alguns elementos do texto}

\subsection[Listas]{Listas {\it \&} Cia.}

itemise, enumerate, description, quotation

\begin{center}\footnotesize\hrule\smallskip
\begin{tabular}{c|c}
\begin{minipage}{.465\textwidth}
\begin{verbatim}
\begin{itemize}
\item cebola,
\item açafrão, e
\item alho.
\end{itemize}
\end{verbatim}
\end{minipage} &
\begin{minipage}{.465\textwidth}
\begin{itemize}
\item cebola,
\item açafrão, e
\item alho.
\end{itemize}
\end{minipage}
\end{tabular}
\smallskip\hrule
\end{center}


\begin{center}\footnotesize\hrule\smallskip
\begin{tabular}{c|c}
\begin{minipage}{.465\textwidth}
\begin{verbatim}
\begin{description}
\item[cebola] Muito empregada p/ temperar.
\item[açafrão] Também.
\item[alho] Idem.
\end{description}
\end{verbatim}
\end{minipage} &
\begin{minipage}{.465\textwidth}
\begin{description}
\item[cebola] Muito empregada p/ temperar.
\item[açafrão] Também.
\item[alho] Idem.
\end{description}
\end{minipage}
\end{tabular}
\smallskip\hrule
\end{center}

\subsection{Alinhamento}\label{sec:alinhamento}

Boa parte dos textos possui alinhamento justificado, i.e., possui
ambas as margens retas e paralelas. Nem sempre isso é
desejado. Existem muitas maneiras de definir o alinhamento do texto:
falamos de duas delas aqui.

\subsection{Texto não-justificado}

\begin{flushleft}
No ambiente
\ambiente{flushleft}\index{flushleft@\ambiente{flushleft}}, o texto é
``empurrado'' para a esquerda. Os espaços não são nem esticados nem
comprimidos. O efeito resultante são linhas de comprimento variável, o
que por vezes é uma opção interessante de diagramação.
\end{flushleft}

\begin{flushright}
Simetricamente,
\ambiente{flushright}\index{flushright@\ambiente{flushright}} tem o
comportamento esperado, fazendo com que o texto no ambiente em
questão, a partir do parágrafo em que aparece, fique com a
esquerda~\emph{rasgada} --- ou seja, para a direita.
\end{flushright}

\subsection{Elementos flutuantes}\label{sec:floats}

Tipógrafos atentam para uma série de características na disposição do
texto que frequentemente passam despercebidas ao nosso consciente. Uma
delas é o equilíbrio entre o texto que se espalha pelas páginas e os
demais elementos, como figuras e tabelas, que pontuam a paisagem aqui
e ali. 

O \LaTeX\ toma várias precauções na disposição desses elementos,
ditos~\emph{flutuantes} (porque sua posição não é fixa no texto como a
de uma palavra em uma frase). É como se os elementos fossem troncos de
árvore à deriva sobre a correnteza de palavras que compõe o texto.

Figuras e tabelas são somente alguns exemplos de elementos
flutuantes. Eles são ambientes (respectivamente~\ambiente{figure}
e~\ambiente{table}) que encapsulam o conteúdo que irá flutuar.

% colocar exemplo usando caption
% enfatizar que o latex sabe onde colocar as imagens
% opções de posicionamento
% todo: como se insere no mecanismo de composição de página

\subsubsection{Figuras}

Figuras são uma ferramenta poderosa na composição de textos, quando
usadas com parcimônia. É possível colocar imagens no documento dizendo
ao \LaTeX\ sua localização (ou apenas seu nome, se estiverem na mesma
pasta que o documento). Também é possível desenhar usando o próprio
\LaTeX, por exemplo, com o pacote \pacote{Tikz}.

Para colocar figuras em um documento \LaTeX, basta usar o comando
\begin{center}
\verb!\includegraphics{nome-do-arquivo}!
\end{center}
em que a extensão do tipo de arquivo não precisa ser incluída. Mas
atenção: nem toda extensão de imagem é conhecida pelo
\LaTeX\ nativamente, embora basta usar um pacote para superar o
problema, na maior parte dos casos. Acrescente
\verb!\usepackage{graphicx}! no preâmbulo de seu documento e você
poderá incluir imagens \extensao{png}, \extensao{jpg} e
\extensao{pdf}, para citar algumas.

\begin{figure}
  \begin{center}
    % From: TeXsample
% The following code is generated by Sketch. I have edited it a bit
% to make it easier to read.
\begin{tikzpicture}[join=round]
    \tikzstyle{conefill} = [fill=blue!20,fill opacity=0.8]
    \tikzstyle{ann} = [fill=white,font=\footnotesize,inner sep=1pt]
    \tikzstyle{ghostfill} = [fill=white]
         \tikzstyle{ghostdraw} = [draw=black!50]
    \filldraw[conefill](-.775,1.922)--(-1.162,.283)--(-.274,.5)
                        --(-.183,2.067)--cycle;
    \filldraw[conefill](-.183,2.067)--(-.274,.5)--(.775,.424)
                        --(.516,2.016)--cycle;
    \filldraw[conefill](.516,2.016)--(.775,.424)--(1.369,.1)
                        --(.913,1.8)--cycle;
    \filldraw[conefill](-.913,1.667)--(-1.369,-.1)--(-1.162,.283)
                        --(-.775,1.922)--cycle;
    \draw(1.461,.107)--(1.734,.127);
    \draw[arrows=<->](1.643,1.853)--(1.643,.12);
    \filldraw[conefill](.913,1.8)--(1.369,.1)--(1.162,-.283)
                        --(.775,1.545)--cycle;
    \draw[arrows=->,line width=.4pt](.274,-.5)--(0,0)--(0,2.86);
    \draw[arrows=-,line width=.4pt](0,0)--(-1.369,-.1);
    \draw[arrows=->,line width=.4pt](-1.369,-.1)--(-2.1,-.153);
    \filldraw[conefill](-.516,1.45)--(-.775,-.424)--(-1.369,-.1)
                        --(-.913,1.667)--cycle;
    \draw(-1.369,.073)--(-1.369,2.76);
    \draw(1.004,1.807)--(1.734,1.86);
    \filldraw[conefill](.775,1.545)--(1.162,-.283)--(.274,-.5)
                        --(.183,1.4)--cycle;
    \draw[arrows=<->](0,2.34)--(-.913,2.273);
    \draw(-.913,1.84)--(-.913,2.447);
    \draw[arrows=<->](0,2.687)--(-1.369,2.587);
    \filldraw[conefill](.183,1.4)--(.274,-.5)--(-.775,-.424)
                        --(-.516,1.45)--cycle;
    \draw[arrows=<-,line width=.4pt](.42,-.767)--(.274,-.5);
    \node[ann] at (-.456,2.307) {$r_0$};
    \node[ann] at (-.685,2.637) {$r_1$};
    \node[ann] at (1.643,.987) {$h$};
    \path (.42,-.767) node[below] {$x$}
        (0,2.86) node[above] {$y$}
        (-2.1,-.153) node[left] {$z$};
    % Second version of the cone
    \begin{scope}[xshift=3.5cm]
    \filldraw[ghostdraw,ghostfill](-.775,1.922)--(-1.162,.283)--(-.274,.5)
                                   --(-.183,2.067)--cycle;
    \filldraw[ghostdraw,ghostfill](-.183,2.067)--(-.274,.5)--(.775,.424) 
                                   --(.516,2.016)--cycle;
    \filldraw[ghostdraw,ghostfill](.516,2.016)--(.775,.424)--(1.369,.1)
                                   --(.913,1.8)--cycle;
    \filldraw[ghostdraw,ghostfill](-.913,1.667)--(-1.369,-.1)--(-1.162,.283)
                                   --(-.775,1.922)--cycle;
    \filldraw[ghostdraw,ghostfill](.913,1.8)--(1.369,.1)--(1.162,-.283)
                                   --(.775,1.545)--cycle;
    \filldraw[ghostdraw,ghostfill](-.516,1.45)--(-.775,-.424)--(-1.369,-.1)
                                   --(-.913,1.667)--cycle;
    \filldraw[ghostdraw,ghostfill](.775,1.545)--(1.162,-.283)--(.274,-.5)
                                   --(.183,1.4)--cycle;
    \filldraw[fill=red,fill
      opacity=0.5](-.516,1.45)--(-.775,-.424)--(.274,-.5)
                                         --(.183,1.4)--cycle;
    \fill(-.775,-.424) circle (2pt);
    \fill(.274,-.5) circle (2pt);
    \fill(-.516,1.45) circle (2pt);
    \fill(.183,1.4) circle (2pt);
    \path[font=\footnotesize]
            (.913,1.8) node[right] {$i\hbox{$=$}0$}
            (1.369,.1) node[right] {$i\hbox{$=$}1$};
    \path[font=\footnotesize]
            (-.645,.513) node[left] {$j$}
            (.228,.45) node[right] {$j\hbox{$+$}1$};
    \draw (-.209,.482)+(-60:.25) [yscale=1.3,->] arc(-60:240:.25);
    \fill[black,font=\footnotesize]
                    (-.516,1.45) node [above] {$P_{00}$}
                    (-.775,-.424) node [below] {$P_{10}$}
                    (.183,1.4) node [above] {$P_{01}$}
                    (.274,-.5) node [below] {$P_{11}$};
    \end{scope}
\end{tikzpicture}

    \caption{Uma figura gerada com o pacote \pacote{Tikz}.}\label{fig:tikz:piramide-cortada}
  \end{center}
\end{figure}


\begin{figure}
  \begin{center}
    \begin{tikzpicture}[scale=.9,every node/.style={minimum size=1cm},on
  grid]
  
    %slanting: production of a set of n 'laminae' to be piled
    %up. N=number of grids.
    \begin{scope}[
            yshift=-83,every node/.append style={
            yslant=0.5,xslant=-1},yslant=0.5,xslant=-1
            ]
        % opacity to prevent graphical interference
        \fill[white,fill opacity=0.9] (0,0) rectangle (5,5);
        \draw[step=4mm, black] (0,0) grid (5,5); %defining grids
        \draw[step=1mm, red!50,thin] (3,1) grid (4,2);  %Nested Grid
        \draw[black,very thick] (0,0) rectangle (5,5);%marking borders
        \fill[red] (0.05,0.05) rectangle (0.35,0.35);
        %Idem as above, for the n-th grid:
    \end{scope}
    
    \begin{scope}[
        yshift=0,every node/.append style={
              yslant=0.5,xslant=-1},yslant=0.5,xslant=-1
                     ]
        \fill[white,fill opacity=.9] (0,0) rectangle (5,5);
        \draw[black,very thick] (0,0) rectangle (5,5);
        \draw[step=5mm, black] (0,0) grid (5,5);
    \end{scope}
    
    \begin{scope}[
        yshift=90,every node/.append style={
          yslant=0.5,xslant=-1},yslant=0.5,xslant=-1
                     ]
      \fill[white,fill opacity=.9] (0,0) rectangle (5,5);
      \draw[step=10mm, black] (1,1) grid (4,4);
      \draw[black,very thick] (1,1) rectangle (4,4);
      \draw[black,dashed] (0,0) rectangle (5,5);
    \end{scope}
    
    \begin{scope}[
        yshift=170,every node/.append style={
              yslant=0.5,xslant=-1},yslant=0.5,xslant=-1
          ]
        \fill[white,fill opacity=0.6] (0,0) rectangle (5,5);
        \draw[step=10mm, black] (2,2) grid (5,5);
        \draw[step=2mm, green] (2,2) grid (3,3);
        \draw[black,very thick] (2,2) rectangle (5,5);
        \draw[black,dashed] (0,0) rectangle (5,5);
    \end{scope}
    
    \begin{scope}[
        yshift=-170,every node/.append style={
        yslant=0.5,xslant=-1},yslant=0.5,xslant=-1
                  ]
        %marking border
        \draw[black,very thick] (0,0) rectangle (5,5);

        %drawing corners (P1,P2, P3): only 3 points needed to define a
        %plane.
        \draw [fill=lime](0,0) circle (.1) ;
        \draw [fill=lime](0,5) circle (.1);
        \draw [fill=lime](5,0) circle (.1);
        \draw [fill=lime](5,5) circle (.1);

        %drawing bathymetric hypotetic countours on the bottom grid:    
        \draw [ultra thick](0,1) parabola bend (2,2) (5,1)  ;
        \draw [dashed] (0,1.5) parabola bend (2.5,2.5) (5,1.5) ;
        \draw [dashed] (0,2) parabola bend (2.7,2.7) (5,2)  ;
        \draw [dashed] (0,2.5) parabola bend (3.5,3.5) (5,2.5)  ;
        \draw [dashed] (0,3.5)  parabola bend (2.75,4.5) (5,3.5);
        \draw [dashed] (0,4)  parabola bend (2.75,4.8) (5,4);
        \draw [dashed] (0,3)  parabola bend (2.75,3.8) (5,3);
        \draw[-latex,thick](2.8,1)node[right]{$\mathsf{Shoreline}$}
                 to[out=180,in=270] (2,1.99);
    \end{scope} %end of drawing grids

    %putting arrows and labels:
    \draw[-latex,thick] (6.2,2) node[right]{$\mathsf{Bathymetry}$}
         to[out=180,in=90] (4,2);

    \draw[-latex,thick](5.8,-.3)node[right]{$\mathsf{Comp.\ G.}$}
        to[out=180,in=90] (3.9,-1);

    \draw[-latex,thick](5.9,5)node[right]{$\mathsf{Wind\ G.}$}
        to[out=180,in=90] (3.6,5);

    \draw[-latex,thick](5.9,8.4)node[right]{$\mathsf{Friction\ G.}$}
        to[out=180,in=90] (3.2,8);

    \draw[-latex,thick,red](5.3,-4.2)node[right]{$\mathsf{G. Cell}$}
        to[out=180,in=90] (0,-2.5);

    \draw[-latex,thick,red](4.3,-1.9)node[right]{$\mathsf{Nested\ G.}$}
        to[out=180,in=90] (2,-.5);

    \draw[-latex,thick](4,-6)node[right]{$\mathsf{Batymetry}$}
    to[out=180,in=90] (2,-5);
    %drawing points on grid's conrners.
    \fill[black,font=\footnotesize]
        (-5,-4.3) node [above] {$P_{1}$}
        (-.3,-5.6) node [below] {$P_{2}$}
    (5.5,-4) node [above] {$P_{3}$};
\end{tikzpicture}

    \caption{Outra figura gerada com o pacote \pacote{Tikz}.}\label{fig:tikz:layers}
  \end{center}
\end{figure}

\subsubsection{Tabelas}\label{sec:tabelas}

\begin{center}
\begin{tabular}{clcr|r|}
  a & b & c & d & e\\
  f & g & h & i & j\\
  \hline
  k & l & m & n & o
\end{tabular}
\end{center}


\begin{table}\centering
  \caption{Gastos {\it \&} despesas / 1º semestre}

  \begin{tabular}{crrrrrrr}
    mês & jan & fev & mar & abr & mai & jun & total\\
    \hline
  receita & $10$ & $0$ & $5$ & $20$ & $12$ & $13$ & $60$\\
  gastos  & $-3$ & $-4$ & $-3$ & $-3$ & $-5$ & $-3$ & $-21$ \\
  \hline
  balanço & $7$ & $-4$ & $2$ & $17$ & $7$ & $10$ & $39$\\
  \end{tabular}
\end{table}

\section{Expandindo o \LaTeX}

\subsection{Criando comandos}

\label{sec:comandos}

Há uma grande quantidade de comandos disponíveis ao usuário de
\LaTeX\ (e ainda mais são criados em pacotes novos
continuamente). Embora a maior parte das coisas que se pode querer
fazer em \LaTeX\ já exista na forma de algum comando, não raro podemos
nos valer, com proveito, do poder de \emph{definir nossos
  próprios comandos}. 

% situações em que comandos são importantes: abreviações, semântica,
% dry


% encapsulando formatação em semântica (negrito, small capitals,
% centralizando, sans  typewritter)

\subsection{Modificando comandos}

\subsection{Criando ambientes}

newenvironment, 

\subsection{Modificando ambientes}
renewenvironment

\section[Múltiplos arquivos]{Projetos com vários arquivos}

Nem todas as pessoas já tiveram a experiência de trabalhar em projetos em que vários arquivos de texto são necessários --- donde o título desta seção pode soar estranho. Aqui discutiremos como (e por que) separar um documento em arquivos diferentes, que geram ainda um único arquivo~\extensao{pdf} ou~\extensao{dvi}.

Há diversas situções em que é vantajoso ter um texto em vários pedaços. Uma bastante comum é o reuso. Dependendo do tipo de documento que você costuma escrever, determinados conjuntos de pacotes serão imprescindíveis, e você se verá acrescentando sempre os mesmos e definindo os mesmos comandos de novo e de novo e de novo\ldots\ Mantendo um arquivo com o seu preâmbulo, você só precisa dizer ao \LaTeX\ (uma vez) onde encontrá-lo.

Arquivos menores são mais fáceis e rápidos de transmitir, imprimir, e de editar (é rápido encontrar o lugar no texto que se quer modificar). Ganha-se ainda em organização: em trabalhos de médio e grande porte, não se pode menosprezar o benefício de ter arquivos relacionados agrupados em uma mesma pasta. Essa vantagem é crucial se há mais de uma pessoa participando do projeto.

Uma possibilidade que a quebra em arquivos traz é processar apenas parte do documento por vez: somente o capítulo que se está editando, por exemplo. (A ``compilação'' de um projeto complexo pode levar alguns minutos --- e podem ser necessárias várias iterações  seguidas durante revisões e restruturações.)

Alguma separação, é inevitável. Os pacotes, e classes de documento, por exemplo, são arquivos de texto que são incluídos no seu texto de dissimuladamente. Listas de figuras e o sumário são outros exemplos.\footnote{Estes são arquivos auxiliares criados pelo sistema durante o processamento de seu texto, e que são incluídos no documento quando ao processá-lo o \LaTeX\ percebe sua existência.} Muitas vezes você irá acrescentar imagens, que, sendo ou não  arquivos de texto, são externos ao documento.

\subsection{\texttt{\char`\\input}}

O modo mais simples e ``puro'' de acrescentar um arquivo, digamos, \arquivo{agradecimentos.tex} ao texto usar o comando \verb'\input{agradecimentos}'. Quando o texto é processado, esse comando tem o efeito de fazer com que o conteúdo do arquivo seja enxertado no texto, na posição exata em que ele ocorre: \emph{para o \LaTeX, é como se o conteúdo sempre houvesse estado ali}.


Se chamado via linha de comando, o \LaTeX\ procura pelo arquivo no diretório (pasta) a partir do qual foi invocado. Ambientes mais elaborados para a edição de documentos têm provavelmente alguma opção de configuração do diretório de ``referência''. Num projeto em que todos os arquivos estão na mesma pasta, isso é indiferente.\footnote{Claro, há casos em que o diretório de referência \emph{faz} diferença. Se esse é o seu caso, sugiro que procure alguém que use o mesmo ambiente que você, ou mesmo peça ajuda na internet. É provável que a solução do problema seja bem simples.}
O diretório de referência passa a ser importante quando o projeto usa arquivos que estão em pastas diferentes. Isso porque o argumento do comando \verb'\input' é mais do que o nome do arquivo. É o \emph{caminho} até o arquivo.

Digamos que arquivo ``principal'' do texto (i.e., aquele que o \LaTeX\  irá processar), chama-se \arquivo{carta-a-dulcineia.tex}. Ele será uma pequena narrativa das aventuras e desventuras que esteve a enfrentar em honra de sua amada. Cada trecho dessa narração está em um arquivo, e digamos que já estão escritos os arquivos \arquivo{o-gigante.tex} e~\arquivo{terrivel-feitico.tex}, guardados na pasta \arquivo{capitulos}; há também um prólogo etílicamente enamorado. O projeto como um todo está numa pasta chamada \arquivo{carta-a-dulcineia}, que está organizado conforme mostra a tabela~\ref{tab:projeto-carta-a-dulcineia}.

\begin{table}\centering
  \begin{tabular}{rl}
%-------------------------------------------------
    \hline
%-------------------------------------------------
    \arquivo{carta-a-dulcineia}: 
    & \\
    & \arquivo{carta-a-dulcineia.tex},\\
    & \arquivo{capitulos}\\
    & \arquivo{prologo.tex}\\
%-------------------------------------------------
%    \hline
%-------------------------------------------------
    \arquivo{capitulos}:
    & \\
    & \arquivo{o-gigante.tex},\\
    & \arquivo{terrivel-feitico.tex}\\
%-----------------------------------------------
    \hline
  \end{tabular}
  \caption{Arquivos do projeto \emph{Carta a Dulcinéia}}%
  \label{tab:projeto-carta-a-dulcineia}
\end{table}


Para que todos esses arquivos apareçam no documento final, eles precisam ser incluídos na \arquivo{carta-a-dulcineia.tex}, que poderia ser escrito como segue.

\begin{ttsampleflushleft}%
\macroCallWithParameter{documentclass}{letter}\\
\ttbegin{document}\\
\macroCallWithParameter{input}{prologo}\\
\macroCallWithParameter{input}{capitulos/o-gigante}\\
\macroCallWithParameter{input}{capitulos/terrivel-feitico}\\
\ttend{document}
\end{ttsampleflushleft}

\subsection{\texttt{\char`\\include\textrm{ e }\char`\\includeonly}}

Outro modo de incluir arquivos é com o comando \verb'\include'. Ele se comporta de maneira idêntica ao \verb'\input', só que cada arquivo enxertado começa em uma nova página. Outra diferença é que você pode usar o comando \verb'\includeonly' no preâmbulo para dizer exatamente quais dos arquivos incluídos (por um \verb'\include') devem ser processados e aparecer no arquivo final. Considerando o exemplo do projeto de carta para a doce Dulcinéia, pode-se processar apenas o capítulo sobre o gigante e o prólogo enquanto se está trabalhando neles, bastando formatar o arquivo como mostrado abaixo. 

\bigskip
\begin{ttsampleflushleft}%
\macroCallWithParameter{documentclass}{letter}\\
\macroCallWithParameter{includeonly}{o-gigante,prologo}\\
\ttbegin{document}\\
\macroCallWithParameter{include}{prologo}\\
\macroCallWithParameter{include}{capitulos/o-gigante}\\
\macroCallWithParameter{include}{capitulos/terrivel-feitico}\\
\ttend{document}
\end{ttsampleflushleft}
\bigskip

Note a lista de documentos que se quer processar, e que os nomes são separados por uma vírgula, sem nenhum espaço entre eles.
\section{Símbolos}

Diagramação é a disposição de símbolos. E há uma infinidade
deles. Citamos nesta apostila alguns deles, mas certamente não o
suficiente para atender à sua necessidade. Recomendamos fortemente que
mantenha uma cópia do excelente trabalho de Scott Pakin, \emph{The
  Comprehensive  \LaTeX\ Symbol
  List}~\cite{Pakin2008}, que muito provavelmente
já está em alguma parte de sua instalação do sistema\footnote{Em
  algumas instalações o arquivo é chamado \arquivo{symbols-a4.pdf}.},
e que exibe uma lista organizada de aproximadamente cinco mil símbolos
que estão a sua disposição. 

Entre os símbolos disponíveis, estão elementos decorativos, símbolos
fonéticos, matemáticos, de linguagens arcaicas, musicais,
genealógicos, enxadrísticos, químicos, diacríticos incomuns ou
compostos, de diagramas de Feynman, de segurança, de legenda em mapas,
etcétera. Nada inesperado para um sistema que permite escrever em
élfico\ldots

%todo escrever em élfico

\section{Matemática}\label{sec:matematica}

Ah, a matemática\ldots Ela é em grande parte a razão pela qual temos o
\LaTeX\ (e os computadores!). Aqui, em particular, o \LaTeX\ brilha.

Existem dois ``modos'' principais nos quais o \LaTeX\ pode operar
quando escreve expressões matemáticas: o \emph{modo matemático inline}
e o \emph{modo matemático ``display''}. Ele está no primeiro, em
geral, quando está escrevendo uma fórmula que deverá ocupar um espaço
limitado. No meio de um parágrafo, por exemplo; mas também em índices
ou em frações, como veremos adiante.

% como o texto normal não é mais texto normal em modo matemático

amsmath


$$x^2$$

$$x_2$$

$$2x^2x$$

$$2x^{2x}$$

$$\sqrt{a}$$

$$\sqrt[a]b$$

Com versão inline.

$$\frac ab$$

$$\sum \sum_a^b$$

$$\int\!\!\!\int_{-\infty}^{+\infty}$$

Frações contínuas e displaystyle
$$(\frac ab) \hbox{ versus } \left(\frac ab\right)$$

$$\partial\delta\alpha\beta\gamma\Gamma\epsilon\varepsilon$$

$$a \choose b$$

$$=\sim\neq\leq\geq\in\notin\cap\cup\oplus\cdot\times\div/\equiv\forall\exists\mapsto\Rightarrow\longleftrightarrow
\lim_{x\to b}$$


$$\vec{\nu}+f(\overbrace{a_1a_2\ldots a_n}^{\hbox{$n$ primo}})$$

\section{Aspectos estruturais}

\subsection{A construção da página}
\subsubsection{Anatomia da página}

notas de rodapé, notas marginais, cabeçalho, número de página

\subsection{Parágrafos marginais}

Usar \marginpar{\raggedleft\footnotesize notas marginais} notas
marginais no texto pode ser uma maneira interessante de destacar algum
conceito. O comando \verb'\marginpar{parágrafo}' acrescenta um
parágrafo à margem do parágrafo atual. É possível mudar drasticamente
a aparência de um parágrafo lateral (assim como de qualquer outro tipo
de parágrafo): diminuir a fonte em que é escrito, deixá-lo rasgado à
direita ou esquerda (seção~\ref{sec:alinhamento}) ou à direita,
envolvê-lo em uma caixa,
rotacioná-lo\marginpar{\raggedleft\rotatebox{90}{\footnotesize\it
    $\mathcal{A}$ssim.}}, etc. --- em suma, qualquer
transformação. Por exemplo, parágrafos de páginas pares e ímpares são
por padrão colocados de modo a que estejam na lateral da folha que
ficaria ``para fora'' caso o texto seja encadernado. Esse
comportamento, para ser mais preciso, depende de algumas definições na
classe do documento.\footnote{Por exemplo, se   você está usando
  alguma  classe de documento padrão, como~\classedoc{article}
  ou~\classedoc{book}, a opção \parametro{twoside} implica que o
  documento será impresso frente-e-verso, o que geralmente implica que
  parágrafos marginais serão colocados à direita ou esquerda
  dependendo de a página a que pertencem ser par ou ímpar (a
  opção~\parametro{oneside} faz todo paragrafo marginal aparecer no
  mesmo lado da página).}

\begin{center}\footnotesize\hrule\smallskip
\begin{tabular}{c|c}
\begin{minipage}{.47\textwidth}
\begin{verbatim}
Houve um tempo\footnote{Por volta de 1920.} 
em que as pessoas viviam como se estivessem 
na década de 20.
\end{verbatim}
\vfill
\end{minipage} &
\begin{minipage}{.47\textwidth}
Houve um tempo\footnote{Por volta de 1920.} em
que as pessoas viviam como se estivessem na 
década de 20.
\vspace*{1cm}
\end{minipage}
\end{tabular}
\smallskip\hrule
\end{center}

clearpage, newpage

\subsubsection{Caixas}

\subsubsection{Medidas}

comprimentos  (medidas, medindo coisas, criando seus próprios
comprimentos)

espaços

pacote \verb!geometry!, \verb!multicolumn!

\subsubsection{Encaixotando}

\subsection{Arquivos auxiliares}

\subsection{Contadores}\label{sec:contadores}

\subsection{Comandos frágeis}

\section{Bibliografia {\it \&} Cia.: Bib\TeX}\label{sec:biblio}

Veremos nesta seção duas abordagens para a composição de
bibliografias. Em uma delas, escrevemos a bibliografia linha por
linha, assim como escrevemos o texto. O \LaTeX\ automaticamente 
numera as entradas da bibliografia, e você pode referenciá-los com o
comando \macroCallWithParameter{cite}{rotulo}.

Outro jeito, muito popular a propósito, de trabalhar com bibliografias,
é usando o programa Bib\TeX. Nessa abordagem, as entradas
bibliográficas são escritas em um arquivo de extensão~\extensao{bib},
seus campos (autor, edição, editora, etc.) são marcados
semanticamente, e a formataçao é deixada a encargo do programa
Bib\TeX\footnote{E pacotes que você porventura acrescente para
  configurar esse conportamento.}.

\subsection{Fazendo no muque}

O mecanismo original de composição de bibliografias pressupõe que elas
estejam postas em um ambiente próprio,
o~\ambiente{thebibliography}. Cada entrada possui opcionalmente um
rótulo público, que aparecerá entre colchetes quando for citada, e
ainda um rótulo interno, que funciona como os rótulos definidos com
\macroCall{label}, podendo ser referenciado usando o
comando~\macroCallWithParameter{cite}{rotulo}. Se nenhum rótulo
público é fornecido, o \LaTeX\ numera as entradas, e coloca ali o
respectivo número.

Outra característica importante deste método é que as referências
aparecem exatamente na ordem em que foram declaradas, como seria de se
esperar. Isso não acontece, veremos, quando se usa o Bib\TeX, que
automatiza a ordenação dos itens da referência.

O processo de compilação do documento se altera quando se acrescenta
bibliografias em um documento, do mesmo modo como acontece quando
usam-se referências internas: o processamento do arquivo gera alguns
arquivos auxiliares, que são usados para escrever as citações.

\begin{verbatim}
\begin{thebibliography}{longuissimo}
\bibitem[Tahan83]{malba-tahan} TAHAN, Malba. \emph{O Homem que
Calculava}. Ed. Círculo do Livro. Edição integral. 1983.

\bibitem[Calvino03]{se-um-viajante} CALVINO, Ítalo. 
\emph{Se um Viajante numa Noite de Inverno}. Ed. Schwarcz. 2003.
\end{thebibliography}
\end{verbatim}

\begin{thebibliography}{longuissimo}
\bibitem[Tahan83]{malba-tahan} TAHAN, Malba. \emph{O Homem que
Calculava}. Ed. Círculo do Livro. Edição integral. 1983.

\bibitem[Calvino03]{se-um-viajante} CALVINO, Ítalo. 
\emph{Se um Viajante numa Noite de Inverno}. Ed. Schwarcz. 2003.
\end{thebibliography}

Vejamos o papel de cada um dos elementos no exemplo. {\tt
  longuissimo} é qualquer texto que tenha tamanho maior (ou igual) ao
rótulo mais longa entrada da bibliografia. Ele é usado pelo
\LaTeX\ para reservar espaço para os rótulos quando ele compõe os
itens da bibliografia.

Tanto {\tt Tahan83} quanto {\tt Calvino03} são rótulos que aparecerão,
por exemplo, quando usar o comando~\macroCallWithParameter{cite}{malba-tahan}.

\subsection{Bib\TeX}

O programa Bib\TeX\ foi criado por Oren Patashnik há mais de vinte
anos. A ideia é preparar um arquivo (geralmente com a extensão
\extensao{bib}), em que estão listadas diversas entradas. Cada entrada
tem um tipo, e uma série de campos com seus respectivos valores. Os
tipos de entradas padrão incluem livros, revistas, propostas de
palestras em conferências (\emph{inproceedings}), artigos
(\emph{article}), parte de livros (\emph{inbook} --- capítulo, seção
etc.), manuais (\emph{manual}), dissertações de mestrado
(\emph{masterthesis}), teses de doutorado (\emph{phdthesis}),
relatórios (\emph{report}), textos que não foram publicados
(\emph{unpublished}) e alguns outros (entre eles miscelâneo ---
\emph{misc} --- quando nada mais servir). Para uma lista de todos os
tipos, nada como olhar a documentação do Bib\TeX\ em seu
sistema.\negthinspace\footnote{Geralmente o arquivo se chama
  \arquivo{btxdoc}.}$^\textrm,$\thinspace\footnote{Também recomendamos
  a páginas~\url{http://www.bibtex.org}
  e~\url{http://data.bibbase.org/}. O formato Bib\TeX\ é bastante
  difundido, e diversos programas possuem expansões para facilitar a
  busca e conversão de entradas de bibliografia usando esse formato.}

Para acrescentar a bibliografia ao documento, podemos usar 
\macroCallWithParameter{bibliography}{arquivo} (\extensao{.bib} não é
necessário), onde se quer que ela seja incluída. 

O comum é que a bibliografia cresça juntamente com o texto. Assim,
eventualmente você escreverá um texto em que citará um determinado
texto. Você acrescenta o \macroCall{cite} correspondente, e depois
acrescenta ao arquivo com as entradas bibliográficas a nova
entrada. Para que ela apareça no documento, será preciso que o
Bib\TeX\ processe o documento, após o que o documento ainda deverá ser
processado como de costume, para que as referências sejam
atualizadas. Para dar conta do recado, processe o documento antes e
ao menos duas vezes depois de processá-lo com o Bib\TeX:
\begin{enumerate}
\item Processe o documento uma vez (\LaTeX e o documento),
\item Bib\TeX e o documento, e
\item processe o documento mais duas vezes (a primeira produz as
  referências corretas e a seguinda as coloca no documento produzido).
\end{enumerate}

O pacote \pacote{babelbib}, usado em conjunto com~\pacote{babel}
facilita a composição de entradas bibliográficas quando as referências
contém itens em diversos idiomas. Problemas com acentuação podem ser
solucionados amiúde com o uso de comandos para acentuação como os
vistos na seção~\ref{sec:ascii}. 

Terminamos a seção com um exemplo simplório de um arquivo com as
entradas bibliográficas para o Bib\TeX. (O comando~\macroCall{LaTeXe}
produz \LaTeXe.)

\begin{verbatim}
@BOOK{malba-tahan,
 author={Malba Tahan},
 title={O Homem que Calculava},
 publusher={Círculo do Livro}
 year={1983},
 note={edição integral}
}

@MANUAL{{lshort,
  author = "Tobias Oetiker and others",
  title = "The Not So Short Introduction to {\LaTeXe}:
           Or {\LaTeXe} in 157 minutes",
  month = june,
  note = "on ctan \url{/info/lshort}",
  year = 2010
}
\end{verbatim}


\section{Índices remissivos {\it \&} Cia.}\label{sec:indice-glossario}

Glossários e índices remissivos são elementos delicados, por sua
natureza distribuída. São como que amarrações de palavras, uma teia
que se espalha por toda a tessitura do documento, associando
conceitos. Veremos agora como usar esses elementos no \LaTeX.

Aqui arquivos auxiliares continuam desempenhando um papel
importante. Mantemos um fluxo de trabalho análogo ao da
preparação de bibliografias (seção~\ref{sec:biblio}). Acrescentamos ao
documento diretrizes declarando a existência do elemento em questão, e
marcamos ainda a as suas manifestações (expressões, indexadas ou
definidas num glossário). Glossários demandam uma preparação a mais: é
preciso escrever as definições dos termos, ou declarar o arquivo
externo em que estão contidas. Tendo feito isso, o procedimento é
padrão:
\begin{enumerate}
\item \LaTeX ar o documento,
\item processar o documento para gerar glossário ou índice remissivo,
\item \LaTeX ar o documento mais duas vezes, para atualizar os índices
  e referências de páginas.
\end{enumerate}

\subsection{Índices remissivos}

O primeiro passo é acrescentar um pacote que permita gerar índices,
como por exemplo~\pacote{makeidx} (que usamos aqui). Ele funciona em
conjunto com o programa~\programa{makeindex}, que processa o documento
em busca dos índices, anotando as páginas em que ocorrem. Alem disso,
é preciso colocar o comando \macroCall{makeindex}, caso contrário o
índice não será produzido (é um jeito fácil de ``desligar'' o índice,
quando oportuno).

O segundo passo é criar as entradas do índice. O que é uma entrada do
índice? É uma referência para um ponto (em geral sua página) do texto,
associado a uma palavra. Como índices podem ser longos, é comum que
sejam ordenados alfabeticamente.
Para associar a um termo uma entrada no índice existe o comando
\macroCallWithParameter{index}{nome no
  índice}\index{index@\macroCall{index}}, que faz com que, no índice
remissivo, seja criada uma entrada apontando para a página corrente,
rotulado com ``{\tt nome no índice}''. 

Acontece que \programa{makeindex} nem sempre tem pela frente uma
tarefa fácil quando se propõe a organizar o índice em ordem
alfabética. Por exemplo, como você ordenaria ``néctar'', ``nervo'',
``\macroCall{newcommand}'', e ``$\pi$''? Para isso existe um modo
alternativo de declarar uma entrada no índice, usando a
seguinte sintaxe.

\begin{center}
\macroCallWithParameter{index}{versao ascii para ordenacao@versão
  \barra com\wrapinbraces{firulas}}\index{versao ascii para
  ordenacao@versão \textbackslash com\{firulas\}}
\end{center}

(Não, \macroCall{com} não é um comando.)

Uma última característica: algumas vezes queremos expressar uma
hierarquia de conceitos no índice. Isso também é possível (os
conceitos-subordinados aparecem todos organizados ``dentro'' da
entrada do elemento hierarquicamente superior, no índice). Como chegar
a esse efeito almejadíssimo é algo a respeito de que nos preservamos o
direito de calar. Tire a caneta da orelha e tome cá, esta é a tua
pulga. Poderá tirá-la daí com uma olhadela na documentação do
pacote~\pacote{makeidx}.

A terceira e última etapa antes que se passe ao processamento do texto
é definir em que ponto daquele estará o índice. É nesse exato lugar
que deve deixar o comando

\begin{center}
  \macroCall{printindex}\index{printindex@\macroCall{printindex}}
\end{center}

Agora é a dança de processamento de sempre, mudando os parezinhos: uma
com~\LaTeX, outra com~\programa, e mais duas com~\LaTeX\ para terminar
bem a farra.

\subsection{Glossários}

A maior parte dos passos envolvidos na composição de um glossário já
foi abordada em alguma parte deste documento. Um pacote muito usado
para compor glossários é~\pacote{glossaries}. Como acontece para
índices remissivos, é preciso haver \macroCall{makeglossaries} em
algum ponto do preâmbulo do documento para que o o glossário seja de
fato processado. Para compor um glossário, definem-se as entradas
(dotando-lhes de um rótulo para futura referência); esses rótulos são
então usados pelo texto para referir aos termos que representam; por
fim, o documento é processado, pelo \LaTeX\ e por algum programa que
permita extrair do documento as ocorrências de termos do glossário.

Definimos rótulos com
\macroCallWithTwoParameters{newglossaryentry}{rotulo}{configuracao}
(há outros meios). O rótulo é apenas um nome pelo qual nos referiremos
ao conceito, como na discussão sobre referências internas na
seção~\ref{sec:ref-e-label}. A configuração é um texto que define
várias propriedades do item do glossário (por exemplo, sua forma plural, sua
descrição, seu nome, caso texto do item seja diferente de seu
rótulo). Há ainda a possibilidade de fornecer um texto opcional que é
usado para a ordenação das entradas no glossário. Observe o
exemplo. (\macroCall{ensuremath} é um comando que garante que seu
argumento esteja em modo matemático.)

\begin{verbatim}
\newglossaryentry{pi}
{
  name={\ensuremath{\pi}},
  description={razão entre o comprimento de uma circunferência 
               e seu diâmetro},
  sort=pi
}
\end{verbatim}

É possível colocar acrônimos (e siglas) em uma lista separada no
glossário. Como fazê-lo é um mistério revelado apenas aos interessados
que se dão ao trabalho de investigar o assunto.

Os itens do glossário são  assessíveis pelas macros
\macroCall{gls}, \macroCall{Gls}, \macroCall{glspl}, \macroCall{Glspl},
entre outras, que recebem como parâmetro um rótulo. Os comandos
iniciando com letra minúscula apresentam a entrada tal como ela foi
declarada (ou o rótulo, se nenhum \parametro{nome} foi
configurado). Com letra maiúscula, deixam a inicial do item maiúscula
(para uso em início de frases, por exemplo). A variante desses
comandos que termina por ``{\tt pl}'' representa uma versão
pluralizada do item.

Definidos e empregados os termos do glossário, determinamos a
localização do glossário no texto com \macroCall{printglossaries}. A
partir daí, é só processar o documento:
\begin{enumerate}
\item \LaTeX ar,
\item processar com o programa~\programa{makeglossaries} (ou
  equivalente), e
\item \LaTeX ar mais duas vezes.
\end{enumerate}

\section{E agora José?}

Uma parte importante de qualquer processo de aprendizado é saber como
dar-lhe continuidade, sempre. As pessoas que usam \LaTeX\ fazem-no com
os objetivos mais diversos, e tem interesses dos mais
variados. Constituem uma comunidade ativa e prestativa, e a rede
pulula de fóruns, blogs e pessoas interessadas em ensinar e aprender
mais sobre a composição de documentos bem-feitos, seja usando \LaTeX,
\TeX\ ou mesmo outra tecnologia qualquer.

Algo frequente, ainda quando ainda se está a habituar-se ao \LaTeX, é
precisar fazer algo e não saber como fazê-lo. Além de excelentes
livros sobre o assunto (veja as referências bibliográficas!), não
deixe de integrar ativamente a comunidade de \LaTeX istas. Pergunte,
investigue, colabore. Queremos conhecê-lo!

% dar alguns ponteiros...

\section{Utilidades}

A partir de agora estás outorgado o título de \LaTeX nico! O que vem
adiante são apenas adendos ao teu cinto de utilidades, mas havendo
dominado o material até aqui, deves estar apto a enfrentar a maior
parte das quiméras tipográficas que o aventureiro compositor
encontrará em uma jornada habitual. No mais, não hesite em convocar a
comunidade, que não se fará surda a qualquer pedido de auxílio.

A lista abaixo contém algumas (poucas!) sugestões de pagotes que você pode achar
interessante investigar. Existem vários pacotes que possuem
finalidades parecidas, quando não idênticas --- fica à sua escolha. A
ideia é que vocÊ conheça um pouco do que dá para fazer com o \LaTeX, a
nova ferramenta no seu cinto de utilidades. Sem mais delongas, a
lista.

\begin{description}
  \item[hyperref] Cria hiperlinks dentro do
    próprio documento, além de controlar seu aspecto. Tem forte
    integração com a estrutura de documentos \extensao{pdf},
    permitindo controlar propriedades como \emph{autor},
    \emph{língua}, etc.
  \item[url] Cria o comando \verb'url', que encapsula páginas na
    internete faz uma quebra ``inteligente''.
  \item[xcolor] Deixa seu texto mais colorido!
  \item[fancyhdr] Personaliza o cabeçalho e rodapé de páginas,
    exibindo, por exemplo, a seção atual, nome do autor ou qualquer
    texto.
  \item[tikz] Desenhe figuras com texto! Visite
    \url{http://www.texample.net/} para ver do que ele é capaz.
  \item[beamer] uma classe de documento para compor slides.
  \item[amsmath] Pacote da \emph{American Mathematical Society} com
    vários comandos para facilitar a composição de expressões
    matemáticas.
  \item[a0poster] Pôsteres em a0!
  \item[microtype] Microtipografia.
  \item[memoir] Uma classe de documento que estende e aprimora
    grandemente as classes documento tradicionais, acrescentando uma
    série de outras categorias.
  \item[multicolumn] Permite usar um número variável de colunas no
    texto.
  \item[indentfirst] Recua a primeira linha do primeiro parágrafo de
    seções.
  \item[belbib] Traduz palavras da bibliografia, como ``edição'',
  ``ano'', etc.
  \item[helvet] Permite usar a fonte Helvética no texto. 
\end{description}



\newpage 

\appendix

%---------The file header---------------------------------------------
%\documentclass[a4paper,12pt]{book}
%
%\usepackage[english]{babel} %language selection
%\selectlanguage{english}
%
%\pagenumbering{arabic}
%
%\usepackage{hyperref}
%\hypersetup{colorlinks, 
%           citecolor=black,
%           filecolor=black,
%           linkcolor=black,
%           urlcolor=black,
%           bookmarksopen=true,
%           pdftex}%%%%%
%
%\hfuzz = .6pt % avoid black boxes
%           
%\begin{document}
%---------------------------------------------------------------------
\phantomsection  % so hyperref creates bookmarks
\addcontentsline{toc}{section}{GNU Free Documentation License}
\section*{\rlap{GNU Free Documentation License}}
%\label{label_fdl}

 \begin{center}

       Version 1.3, 3 November 2008


 Copyright \copyright{} 2000, 2001, 2002, 2007, 2008  Free Software Foundation, Inc.
 
 \bigskip
 
     \url{<http://fsf.org/>}
  
 \bigskip
 
 Everyone is permitted to copy and distribute verbatim copies
 of this license document, but changing it is not allowed.
\end{center}


\begin{center}
{\bf\large Preamble}
\end{center}

The purpose of this License is to make a manual, textbook, or other
functional and useful document ``free'' in the sense of freedom: to
assure everyone the effective freedom to copy and redistribute it,
with or without modifying it, either commercially or noncommercially.
Secondarily, this License preserves for the author and publisher a way
to get credit for their work, while not being considered responsible
for modifications made by others.

This License is a kind of ``copyleft'', which means that derivative
works of the document must themselves be free in the same sense.  It
complements the GNU General Public License, which is a copyleft
license designed for free software.

We have designed this License in order to use it for manuals for free
software, because free software needs free documentation: a free
program should come with manuals providing the same freedoms that the
software does.  But this License is not limited to software manuals;
it can be used for any textual work, regardless of subject matter or
whether it is published as a printed book.  We recommend this License
principally for works whose purpose is instruction or reference.


\begin{center}
{\Large\bf 1. APPLICABILITY AND DEFINITIONS\par}
\phantomsection
%\addcontentsline{toc}{section}{1. APPLICABILITY AND DEFINITIONS}
\end{center}

This License applies to any manual or other work, in any medium, that
contains a notice placed by the copyright holder saying it can be
distributed under the terms of this License.  Such a notice grants a
world-wide, royalty-free license, unlimited in duration, to use that
work under the conditions stated herein.  The ``\textbf{Document}'', below,
refers to any such manual or work.  Any member of the public is a
licensee, and is addressed as ``\textbf{you}''.  You accept the license if you
copy, modify or distribute the work in a way requiring permission
under copyright law.

A ``\textbf{Modified Version}'' of the Document means any work containing the
Document or a portion of it, either copied verbatim, or with
modifications and/or translated into another language.

A ``\textbf{Secondary Section}'' is a named appendix or a front-matter section of
the Document that deals exclusively with the relationship of the
publishers or authors of the Document to the Document's overall subject
(or to related matters) and contains nothing that could fall directly
within that overall subject.  (Thus, if the Document is in part a
textbook of mathematics, a Secondary Section may not explain any
mathematics.)  The relationship could be a matter of historical
connection with the subject or with related matters, or of legal,
commercial, philosophical, ethical or political position regarding
them.

The ``\textbf{Invariant Sections}'' are certain Secondary Sections whose titles
are designated, as being those of Invariant Sections, in the notice
that says that the Document is released under this License.  If a
section does not fit the above definition of Secondary then it is not
allowed to be designated as Invariant.  The Document may contain zero
Invariant Sections.  If the Document does not identify any Invariant
Sections then there are none.

The ``\textbf{Cover Texts}'' are certain short passages of text that are listed,
as Front-Cover Texts or Back-Cover Texts, in the notice that says that
the Document is released under this License.  A Front-Cover Text may
be at most 5 words, and a Back-Cover Text may be at most 25 words.

A ``\textbf{Transparent}'' copy of the Document means a machine-readable copy,
represented in a format whose specification is available to the
general public, that is suitable for revising the document
straightforwardly with generic text editors or (for images composed of
pixels) generic paint programs or (for drawings) some widely available
drawing editor, and that is suitable for input to text formatters or
for automatic translation to a variety of formats suitable for input
to text formatters.  A copy made in an otherwise Transparent file
format whose markup, or absence of markup, has been arranged to thwart
or discourage subsequent modification by readers is not Transparent.
An image format is not Transparent if used for any substantial amount
of text.  A copy that is not ``Transparent'' is called ``\textbf{Opaque}''.

Examples of suitable formats for Transparent copies include plain
ASCII without markup, Texinfo input format, LaTeX input format, SGML
or XML using a publicly available DTD, and standard-conforming simple
HTML, PostScript or PDF designed for human modification.  Examples of
transparent image formats include PNG, XCF and JPG.  Opaque formats
include proprietary formats that can be read and edited only by
proprietary word processors, SGML or XML for which the DTD and/or
processing tools are not generally available, and the
machine-generated HTML, PostScript or PDF produced by some word
processors for output purposes only.

The ``\textbf{Title Page}'' means, for a printed book, the title page itself,
plus such following pages as are needed to hold, legibly, the material
this License requires to appear in the title page.  For works in
formats which do not have any title page as such, ``Title Page'' means
the text near the most prominent appearance of the work's title,
preceding the beginning of the body of the text.

The ``\textbf{publisher}'' means any person or entity that distributes
copies of the Document to the public.

A section ``\textbf{Entitled XYZ}'' means a named subunit of the Document whose
title either is precisely XYZ or contains XYZ in parentheses following
text that translates XYZ in another language.  (Here XYZ stands for a
specific section name mentioned below, such as ``\textbf{Acknowledgements}'',
``\textbf{Dedications}'', ``\textbf{Endorsements}'', or ``\textbf{History}''.)  
To ``\textbf{Preserve the Title}''
of such a section when you modify the Document means that it remains a
section ``Entitled XYZ'' according to this definition.

The Document may include Warranty Disclaimers next to the notice which
states that this License applies to the Document.  These Warranty
Disclaimers are considered to be included by reference in this
License, but only as regards disclaiming warranties: any other
implication that these Warranty Disclaimers may have is void and has
no effect on the meaning of this License.


\begin{center}
{\Large\bf 2. VERBATIM COPYING\par}
\phantomsection
%\addcontentsline{toc}{section}{2. VERBATIM COPYING}
\end{center}

You may copy and distribute the Document in any medium, either
commercially or noncommercially, provided that this License, the
copyright notices, and the license notice saying this License applies
to the Document are reproduced in all copies, and that you add no other
conditions whatsoever to those of this License.  You may not use
technical measures to obstruct or control the reading or further
copying of the copies you make or distribute.  However, you may accept
compensation in exchange for copies.  If you distribute a large enough
number of copies you must also follow the conditions in section~3.

You may also lend copies, under the same conditions stated above, and
you may publicly display copies.


\begin{center}
{\Large\bf 3. COPYING IN QUANTITY\par}
\phantomsection
%\addcontentsline{toc}{section}{3. COPYING IN QUANTITY}
\end{center}


If you publish printed copies (or copies in media that commonly have
printed covers) of the Document, numbering more than 100, and the
Document's license notice requires Cover Texts, you must enclose the
copies in covers that carry, clearly and legibly, all these Cover
Texts: Front-Cover Texts on the front cover, and Back-Cover Texts on
the back cover.  Both covers must also clearly and legibly identify
you as the publisher of these copies.  The front cover must present
the full title with all words of the title equally prominent and
visible.  You may add other material on the covers in addition.
Copying with changes limited to the covers, as long as they preserve
the title of the Document and satisfy these conditions, can be treated
as verbatim copying in other respects.

If the required texts for either cover are too voluminous to fit
legibly, you should put the first ones listed (as many as fit
reasonably) on the actual cover, and continue the rest onto adjacent
pages.

If you publish or distribute Opaque copies of the Document numbering
more than 100, you must either include a machine-readable Transparent
copy along with each Opaque copy, or state in or with each Opaque copy
a computer-network location from which the general network-using
public has access to download using public-standard network protocols
a complete Transparent copy of the Document, free of added material.
If you use the latter option, you must take reasonably prudent steps,
when you begin distribution of Opaque copies in quantity, to ensure
that this Transparent copy will remain thus accessible at the stated
location until at least one year after the last time you distribute an
Opaque copy (directly or through your agents or retailers) of that
edition to the public.

It is requested, but not required, that you contact the authors of the
Document well before redistributing any large number of copies, to give
them a chance to provide you with an updated version of the Document.


\begin{center}
{\Large\bf 4. MODIFICATIONS\par}
\phantomsection
%\addcontentsline{toc}{section}{4. MODIFICATIONS}
\end{center}

You may copy and distribute a Modified Version of the Document under
the conditions of sections 2 and 3 above, provided that you release
the Modified Version under precisely this License, with the Modified
Version filling the role of the Document, thus licensing distribution
and modification of the Modified Version to whoever possesses a copy
of it.  In addition, you must do these things in the Modified Version:

\begin{itemize}
\item[A.] 
   Use in the Title Page (and on the covers, if any) a title distinct
   from that of the Document, and from those of previous versions
   (which should, if there were any, be listed in the History section
   of the Document).  You may use the same title as a previous version
   if the original publisher of that version gives permission.
   
\item[B.]
   List on the Title Page, as authors, one or more persons or entities
   responsible for authorship of the modifications in the Modified
   Version, together with at least five of the principal authors of the
   Document (all of its principal authors, if it has fewer than five),
   unless they release you from this requirement.
   
\item[C.]
   State on the Title page the name of the publisher of the
   Modified Version, as the publisher.
   
\item[D.]
   Preserve all the copyright notices of the Document.
   
\item[E.]
   Add an appropriate copyright notice for your modifications
   adjacent to the other copyright notices.
   
\item[F.]
   Include, immediately after the copyright notices, a license notice
   giving the public permission to use the Modified Version under the
   terms of this License, in the form shown in the Addendum below.
   
\item[G.]
   Preserve in that license notice the full lists of Invariant Sections
   and required Cover Texts given in the Document's license notice.
   
\item[H.]
   Include an unaltered copy of this License.
   
\item[I.]
   Preserve the section Entitled ``History'', Preserve its Title, and add
   to it an item stating at least the title, year, new authors, and
   publisher of the Modified Version as given on the Title Page.  If
   there is no section Entitled ``History'' in the Document, create one
   stating the title, year, authors, and publisher of the Document as
   given on its Title Page, then add an item describing the Modified
   Version as stated in the previous sentence.
   
\item[J.]
   Preserve the network location, if any, given in the Document for
   public access to a Transparent copy of the Document, and likewise
   the network locations given in the Document for previous versions
   it was based on.  These may be placed in the ``History'' section.
   You may omit a network location for a work that was published at
   least four years before the Document itself, or if the original
   publisher of the version it refers to gives permission.
   
\item[K.]
   For any section Entitled ``Acknowledgements'' or ``Dedications'',
   Preserve the Title of the section, and preserve in the section all
   the substance and tone of each of the contributor acknowledgements
   and/or dedications given therein.
   
\item[L.]
   Preserve all the Invariant Sections of the Document,
   unaltered in their text and in their titles.  Section numbers
   or the equivalent are not considered part of the section titles.
   
\item[M.]
   Delete any section Entitled ``Endorsements''.  Such a section
   may not be included in the Modified Version.
   
\item[N.]
   Do not retitle any existing section to be Entitled ``Endorsements''
   or to conflict in title with any Invariant Section.
   
\item[O.]
   Preserve any Warranty Disclaimers.
\end{itemize}

If the Modified Version includes new front-matter sections or
appendices that qualify as Secondary Sections and contain no material
copied from the Document, you may at your option designate some or all
of these sections as invariant.  To do this, add their titles to the
list of Invariant Sections in the Modified Version's license notice.
These titles must be distinct from any other section titles.

You may add a section Entitled ``Endorsements'', provided it contains
nothing but endorsements of your Modified Version by various
parties---for example, statements of peer review or that the text has
been approved by an organization as the authoritative definition of a
standard.

You may add a passage of up to five words as a Front-Cover Text, and a
passage of up to 25 words as a Back-Cover Text, to the end of the list
of Cover Texts in the Modified Version.  Only one passage of
Front-Cover Text and one of Back-Cover Text may be added by (or
through arrangements made by) any one entity.  If the Document already
includes a cover text for the same cover, previously added by you or
by arrangement made by the same entity you are acting on behalf of,
you may not add another; but you may replace the old one, on explicit
permission from the previous publisher that added the old one.

The author(s) and publisher(s) of the Document do not by this License
give permission to use their names for publicity for or to assert or
imply endorsement of any Modified Version.


\begin{center}
{\Large\bf 5. COMBINING DOCUMENTS\par}
\phantomsection
%\addcontentsline{toc}{section}{5. COMBINING DOCUMENTS}
\end{center}


You may combine the Document with other documents released under this
License, under the terms defined in section~4 above for modified
versions, provided that you include in the combination all of the
Invariant Sections of all of the original documents, unmodified, and
list them all as Invariant Sections of your combined work in its
license notice, and that you preserve all their Warranty Disclaimers.

The combined work need only contain one copy of this License, and
multiple identical Invariant Sections may be replaced with a single
copy.  If there are multiple Invariant Sections with the same name but
different contents, make the title of each such section unique by
adding at the end of it, in parentheses, the name of the original
author or publisher of that section if known, or else a unique number.
Make the same adjustment to the section titles in the list of
Invariant Sections in the license notice of the combined work.

In the combination, you must combine any sections Entitled ``History''
in the various original documents, forming one section Entitled
``History''; likewise combine any sections Entitled ``Acknowledgements'',
and any sections Entitled ``Dedications''.  You must delete all sections
Entitled ``Endorsements''.

\begin{center}
{\Large\bf 6. COLLECTIONS OF DOCUMENTS\par}
\phantomsection
%\addcontentsline{toc}{section}{6. COLLECTIONS OF DOCUMENTS}
\end{center}

You may make a collection consisting of the Document and other documents
released under this License, and replace the individual copies of this
License in the various documents with a single copy that is included in
the collection, provided that you follow the rules of this License for
verbatim copying of each of the documents in all other respects.

You may extract a single document from such a collection, and distribute
it individually under this License, provided you insert a copy of this
License into the extracted document, and follow this License in all
other respects regarding verbatim copying of that document.


\begin{center}
{\Large\bf 7. AGGREGATION WITH INDEPENDENT WORKS\par}
\phantomsection
%\addcontentsline{toc}{section}{7. AGGREGATION WITH INDEPENDENT WORKS}
\end{center}


A compilation of the Document or its derivatives with other separate
and independent documents or works, in or on a volume of a storage or
distribution medium, is called an ``aggregate'' if the copyright
resulting from the compilation is not used to limit the legal rights
of the compilation's users beyond what the individual works permit.
When the Document is included in an aggregate, this License does not
apply to the other works in the aggregate which are not themselves
derivative works of the Document.

If the Cover Text requirement of section~3 is applicable to these
copies of the Document, then if the Document is less than one half of
the entire aggregate, the Document's Cover Texts may be placed on
covers that bracket the Document within the aggregate, or the
electronic equivalent of covers if the Document is in electronic form.
Otherwise they must appear on printed covers that bracket the whole
aggregate.


\begin{center}
{\Large\bf 8. TRANSLATION\par}
\phantomsection
%\addcontentsline{toc}{section}{8. TRANSLATION}
\end{center}


Translation is considered a kind of modification, so you may
distribute translations of the Document under the terms of section~4.
Replacing Invariant Sections with translations requires special
permission from their copyright holders, but you may include
translations of some or all Invariant Sections in addition to the
original versions of these Invariant Sections.  You may include a
translation of this License, and all the license notices in the
Document, and any Warranty Disclaimers, provided that you also include
the original English version of this License and the original versions
of those notices and disclaimers.  In case of a disagreement between
the translation and the original version of this License or a notice
or disclaimer, the original version will prevail.

If a section in the Document is Entitled ``Acknowledgements'',
``Dedications'', or ``History'', the requirement (section~4) to Preserve
its Title (section~1) will typically require changing the actual
title.


\begin{center}
{\Large\bf 9. TERMINATION\par}
\phantomsection
%\addcontentsline{toc}{section}{9. TERMINATION}
\end{center}


You may not copy, modify, sublicense, or distribute the Document
except as expressly provided under this License.  Any attempt
otherwise to copy, modify, sublicense, or distribute it is void, and
will automatically terminate your rights under this License.

However, if you cease all violation of this License, then your license
from a particular copyright holder is reinstated (a) provisionally,
unless and until the copyright holder explicitly and finally
terminates your license, and (b) permanently, if the copyright holder
fails to notify you of the violation by some reasonable means prior to
60 days after the cessation.

Moreover, your license from a particular copyright holder is
reinstated permanently if the copyright holder notifies you of the
violation by some reasonable means, this is the first time you have
received notice of violation of this License (for any work) from that
copyright holder, and you cure the violation prior to 30 days after
your receipt of the notice.

Termination of your rights under this section does not terminate the
licenses of parties who have received copies or rights from you under
this License.  If your rights have been terminated and not permanently
reinstated, receipt of a copy of some or all of the same material does
not give you any rights to use it.


\begin{center}
{\Large\bf 10. FUTURE REVISIONS OF THIS LICENSE\par}
\phantomsection
%\addcontentsline{toc}{section}{10. FUTURE REVISIONS OF THIS LICENSE}
\end{center}


The Free Software Foundation may publish new, revised versions
of the GNU Free Documentation License from time to time.  Such new
versions will be similar in spirit to the present version, but may
differ in detail to address new problems or concerns.  See
http://www.gnu.org/copyleft/.

Each version of the License is given a distinguishing version number.
If the Document specifies that a particular numbered version of this
License ``or any later version'' applies to it, you have the option of
following the terms and conditions either of that specified version or
of any later version that has been published (not as a draft) by the
Free Software Foundation.  If the Document does not specify a version
number of this License, you may choose any version ever published (not
as a draft) by the Free Software Foundation.  If the Document
specifies that a proxy can decide which future versions of this
License can be used, that proxy's public statement of acceptance of a
version permanently authorizes you to choose that version for the
Document.


\begin{center}
{\Large\bf 11. RELICENSING\par}
\phantomsection
%\addcontentsline{toc}{section}{11. RELICENSING}
\end{center}


``Massive Multiauthor Collaboration Site'' (or ``MMC Site'') means any
World Wide Web server that publishes copyrightable works and also
provides prominent facilities for anybody to edit those works.  A
public wiki that anybody can edit is an example of such a server.  A
``Massive Multiauthor Collaboration'' (or ``MMC'') contained in the
site means any set of copyrightable works thus published on the MMC
site.

``CC-BY-SA'' means the Creative Commons Attribution-Share Alike 3.0
license published by Creative Commons Corporation, a not-for-profit
corporation with a principal place of business in San Francisco,
California, as well as future copyleft versions of that license
published by that same organization.

``Incorporate'' means to publish or republish a Document, in whole or
in part, as part of another Document.

An MMC is ``eligible for relicensing'' if it is licensed under this
License, and if all works that were first published under this License
somewhere other than this MMC, and subsequently incorporated in whole
or in part into the MMC, (1) had no cover texts or invariant sections,
and (2) were thus incorporated prior to November 1, 2008.

The operator of an MMC Site may republish an MMC contained in the site
under CC-BY-SA on the same site at any time before August 1, 2009,
provided the MMC is eligible for relicensing.


\begin{center}
{\Large\bf ADDENDUM: How to use this License for your documents\par}
\phantomsection
%\addcontentsline{toc}{section}{ADDENDUM: How to use this License for your documents}
\end{center}

To use this License in a document you have written, include a copy of
the License in the document and put the following copyright and
license notices just after the title page:

\bigskip
\begin{quote}
    Copyright \copyright{}  YEAR  YOUR NAME.
    Permission is granted to copy, distribute and/or modify this document
    under the terms of the GNU Free Documentation License, Version 1.3
    or any later version published by the Free Software Foundation;
    with no Invariant Sections, no Front-Cover Texts, and no Back-Cover Texts.
    A copy of the license is included in the section entitled ``GNU
    Free Documentation License''.
\end{quote}
\bigskip
    
If you have Invariant Sections, Front-Cover Texts and Back-Cover Texts,
replace the ``with \dots\ Texts.'' line with this:

\bigskip
\begin{quote}
    with the Invariant Sections being LIST THEIR TITLES, with the
    Front-Cover Texts being LIST, and with the Back-Cover Texts being LIST.
\end{quote}
\bigskip
    
If you have Invariant Sections without Cover Texts, or some other
combination of the three, merge those two alternatives to suit the
situation.

If your document contains nontrivial examples of program code, we
recommend releasing these examples in parallel under your choice of
free software license, such as the GNU General Public License,
to permit their use in free software.

%---------------------------------------------------------------------
%\end{document}

\newpage

\printindex%
\addcontentsline{toc}{section}{Índice Remissivo}
\newpage 

\bibliographystyle{babalpha}
\cleardoublepage \phantomsection
\addcontentsline{toc}{section}{Referências}
\bibliography{thebib.bib}
\end{document}
