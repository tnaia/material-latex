\usepackage[english,brazil,german,portuguese]{babel}
\usepackage[utf8]{inputenc}
\usepackage[pdfpagelabels,colorlinks]{hyperref}
\hypersetup{colorlinks, 
           bookmarksopen=true,
           pdftex}
\usepackage{a4wide}
\usepackage{indentfirst}
%\usepackage[xindy]{glossaries}
%\makeglossaries
\usepackage[light,oldstyle,nosf,nomath,nott]{kpfonts}
\usepackage{microtype}
\usepackage{tikz}
\usetikzlibrary{positioning}
%\usepackage{showidx}
\usepackage{makeidx}
\usepackage[fixlanguage,portuguese]{babelbib}
\makeindex

\usepackage{booktabs}
\usepackage{ifthen}

\newcommand{\extensao}[1]{\texttt{#1}}
\newcommand{\arquivo}[1]{\texttt{#1}}
\newcommand{\programa}[1]{\texttt{#1}}
\newcommand{\pacote}[1]{\textsf{#1}}
\newcommand{\classedoc}[1]{\textsf{#1}}
\newcommand{\parametro}[1]{\texttt{#1}}
\newcommand{\ambiente}[1]{\textsf{#1}}
\newcommand{\acronimo}[1]{\textsc{#1}}
\usepackage{verbatim}\makeatletter\g@addto@macro\verbatim{\microtypesetup{activate=false}}\makeatother
\newsavebox{\mybox}
\newlength{\mydepth}
\newlength{\myheight}
%\newenvironment{detalhe}{\hfill\begin{minipage}{.8\textwidth}}{\end{minipage}}

\newenvironment{detalhe}%
{\medskip\begin{lrbox}{\mybox}\footnotesize\begin{minipage}{\textwidth}}%
{\end{minipage}\end{lrbox}%
\settodepth{\mydepth}{\usebox{\mybox}}%
\settoheight{\myheight}{\usebox{\mybox}}%
\addtolength{\myheight}{\mydepth}%
\noindent\makebox[0pt]{\hspace{-20pt}\rule[-\mydepth]{1pt}{\myheight}}%
\usebox{\mybox}\medskip}

% abreviação para left & right braces
\newcommand{\lch}{\char`\{}
\newcommand{\rch}{\char`\}}
\newcommand{\lrch}{\lch\rch}
\newcommand{\rlch}{\rch\lch}

% barra `\' em tt
\newcommand{\barra}{{\tt\char`\\}}

% underscore '_' em tt
\newcommand{\underscore}{{\tt\char`\_}}

% envolve parâmetro em chaves em tt
\newcommand{\wrapinbraces}[1]{{\tt \lch#1\rch}}

% chamada de macro
\newcommand{\macroCall}[1]{{\tt\barra#1}}
\newcommand{\macroCallWithParameter}[2]{{\tt\barra#1\wrapinbraces{#2}}}
\newcommand{\macroCallWithTwoParameters}[3]{{\tt\barra#1%
        \wrapinbraces{#2}%
        \wrapinbraces{#3}}}
\newcommand{\macroCallWithThreeParameters}[4]{{\tt\barra#1%
        \wrapinbraces{#2}%
        \wrapinbraces{#3}%
        \wrapinbraces{#4}}}


% definição de comando
\newcommand{\defNewCommand}[2]{\macroCallWithTwoParameters{newcommand}{#1}{#2}}
\newcommand{\defNewCommandWithParameter}[3]{{%
        \tt\barra newcommand\wrapinbraces{\barra#1}[#2]\wrapinbraces{#3}}}
\newcommand{\redefCommand}[2]{\macroCallWithTwoParameters{renewcommand}{#1}{#2}}
\newcommand{\ttbegin}[1]{\macroCallWithParameter{begin}{#1}}
\newcommand{\ttend}[1]{\macroCallWithParameter{end}{#1}}


% ambiente em que vou aninhar tt em footnote...
\newenvironment{ttsampleflushleft}{\noindent\footnotesize\tt\microtypesetup{protrusion=false}}{\microtypesetup{protrusion=true}}
\newenvironment{ttsample}%
        {\begin{center}\begin{ttsampleflushleft}}%
        {\end{ttsampleflushleft}\end{center}}

% nomes
\newcommand{\acronimowysiwyg}{wysiwyg}
\newcommand{\wysiwyg}{what you see is what you get}
\newcommand{\acronimowysiwym}{wysiwym}
\newcommand{\wysiwym}{what you see is what you mean}
