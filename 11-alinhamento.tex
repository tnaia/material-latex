\section{Alinhamento}\label{sec:alinhamento}

Boa parte dos textos possui alinhamento justificado, i.e., possui
ambas as margens retas e paralelas. Nem sempre isso é
desejado. Existem muitas maneiras de definir o alinhamento do texto:
falamos de duas delas aqui.

\subsection{Texto não-justificado}

\begin{flushleft}
No ambiente
\ambiente{flushleft}\index{flushleft@\ambiente{flushleft}}, o texto é
``empurrado'' para a esquerda. Os espaços não são nem esticados nem
comprimidos. O efeito resultante são linhas de comprimento variável, o
que por vezes é uma opção interessante de diagramação.
\end{flushleft}

\begin{flushright}
Simetricamente,
\ambiente{flushright}\index{flushright@\ambiente{flushright}} tem o
comportamento esperado, fazendo com que o texto no ambiente em
questão, a partir do parágrafo em que aparece, fique com a
esquerda~\emph{rasgada} --- ou seja, para a direita.
\end{flushright}

\subsection{Parágrafos marginais}

Usar \marginpar{\raggedleft\footnotesize notas marginais} notas
marginais no texto pode ser uma maneira interessante de destacar algum
conceito. O comando \verb'\marginpar{parágrafo}' acrescenta um
parágrafo à margem do parágrafo atual. É possível mudar drasticamente
a aparência de um parágrafo lateral (assim como de qualquer outro tipo
de parágrafo): diminuir a fonte em que é escrito, deixá-lo rasgado à
direita ou esquerda (seção~\ref{sec:alinhamento}) ou à direita,
envolvê-lo em uma caixa,
rotacioná-lo\marginpar{\raggedleft\rotatebox{90}{\footnotesize\it $\mathcal{A}$ssim.}}, etc. ---
em suma, qualquer transformação. Por exemplo, parágrafos de páginas
pares e ímpares são por padrão colocados de modo a que estejam na
lateral da folha que ficaria ``para fora'' caso o texto seja
encadernado. Esse comportamento, para ser mais preciso, depende de
algumas definições na classe do documento.\footnote{Por exemplo, se
  você está usando alguma  classe de documento padrão,
  como~\classedoc{article} ou~\classedoc{book}, a opção
  \parametro{twoside} implica que o documento será impresso
  frente-e-verso, o que geralmente implica que parágrafos marginais
  serão colocados à direita ou esquerda dependendo de a página a que
  pertencem ser par ou ímpar (a opção~\parametro{oneside} faz todo
  paragrafo marginal aparecer no mesmo lado da página).}
