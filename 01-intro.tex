\section{Introdução}

\subsection{A metáfora}

\LaTeX\ se apoia fortemente uma certa metáfora, cujo conhecimento pode
nos poupar (ou melhor, auxiliar-nos a lidar com) algumas dores de
cabeça. É a hipótese de que o \emph{ritmo visual} de um texto deve
enfatizar sua estrutura. Por exemplo: a formatação consistente de
títulos de seções, destacando em que ponto se iniciam, qual o seu
realça a coordenação entre os trechos que compõe o documento.


\subsection{Sinopse da Ópera}
% TeX e LaTeX (Knuth, Lamport, comunidade, uso)

Donald E.~Knuth criou \TeX, um sistema de tipografia digital muito~(!!)
poderoso, e extremamente flexível.

\begin{center}
\begin{minipage}{.75\textwidth}
  [\TeX\ is] \textit{a new typesetting system intended for the creation of
  beautiful books---and especially for books that contain a lot of
  mathematics.}

  \hfill Knuth---The \TeX book
\end{minipage}
\end{center}

  
Leslie Lamport criou o \LaTeX, que, a grosso modo, é uma interface
mais simplista para o uso do \TeX. Uma preocupação do \LaTeX\ é que,
ao usá-lo, tenhamos foco no conteúdo, na estrutura do que estamos a
compor. Busca separar as etapas de composição conceitual e visual do
texto (note que essa é exatamente a premissa do \LaTeX).  

\subsection{O que dá pra fazer}

Compor textos belíssimos. (Por que não?) Compor textos horrorosos. Na
prática, veremos em breve, é simples produzir
documentos~\extensao{pdf}, \extensao{ps}, e~\extensao{dvi}; documentos
com diagramas (que podem ser desenhados usando o próprio sistema, ou
importando imagens~\extensao{jpg}, \extensao{eps}, \extensao{pdf},
etc.), tabelas, versos, referências bibliográficas, índices,
hiperlinks, e muitas outras coisas.

% colocar uma tabela com células (uma imagem importada, uma desenhada com o tikx, e texto)
