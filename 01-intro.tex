\section{Introdução}



\subsection{A metáfora}

\LaTeX\ se apoia fortemente numa certa relação entre apresentação e
conteúdo do texto, cujo
conhecimento pode poupar-nos (ou melhor, auxiliar-nos a lidar com)
algumas dores de cabeça: é o princípio de que o \emph{ritmo visual} de
um texto deve enfatizar sua estrutura. Por exemplo: a formatação
consistente de títulos de seções, destacando em que ponto se iniciam,
realça a coordenação entre os trechos que compõem o documento.

Esse pressuposto é válido para a vasta maioria dos escritos, em
particular livros convencionais, publicações de caráter técnico,
tais como relatórios, monografias,  relatórios, cartas, etc. A fatia
deixada de fora abarca produções caracterizadas por alguma
inconstância, defasagem intencional ou arritmia entre o conteúdo e a
formatação --- como trabalhos artísticos.

\subsection{Sinopse da Ópera}
% TeX e LaTeX (Knuth, Lamport, comunidade, uso)

Donald E.~Knuth criou \TeX, um sistema de tipografia digital muito~(!!)
poderoso, e extremamente flexível.

\begin{center}
\begin{minipage}{.75\textwidth}
  [\TeX\ is] \textit{a new typesetting system intended for the creation of
  beautiful books---and especially for books that contain a lot of
  mathematics.}

  \hfill Knuth---The \TeX book
\end{minipage}
\end{center}

  
Leslie Lamport criou o \LaTeX, que, a grosso modo, é uma interface
mais simplista para o uso do \TeX. Uma preocupação do \LaTeX\ é que,
ao usá-lo, tenhamos foco no conteúdo, na estrutura do que estamos a
compor. Busca separar as etapas de composição conceitual e visual do
texto.

Em contraposição ao modelo de edição de texto dos programas mais
populares hoje, em que 
\emph{o que você vê é o que você obtém}%
\footnote{Conhecido pela sigla em inglês \acronimo{wysiwyg}\index{wysiwyg@\acronimo{wysiwyg}},
  \emph{what you see is what you get}.} (ao menos deveria ser), ao usar
\LaTeX\ \emph{o que você vê é o que você quis dizer}%
\footnote{Do inglês: \emph{what you see is what you mean} (\acronimo{wyhiwym})\index{wysiwym@\acronimo{wysiwym}}.}.



\subsection{O que dá pra fazer}

Compor textos belíssimos. (E, por que não? Compor textos horrorosos.) Na
prática, veremos em breve, é simples produzir
documentos~\extensao{pdf}, \extensao{ps}, e~\extensao{dvi}; documentos
com diagramas (que podem ser desenhados usando o próprio sistema, ou
importando imagens~\extensao{jpg}, \extensao{eps}, \extensao{pdf},
etc.), tabelas, versos, referências bibliográficas, índices,
hiperlinks, e muitas outras coisas.

% colocar uma tabela com células (uma imagem importada, uma desenhada com o tikx, e texto)

\subsection{Antes de por a mão na massa\ldots}

\begin{detalhe}
Parágrafos que estejam com esta marcação contêm detalhes que talvez
sejam prescindíveis em uma primeira leitura. Falam de assuntos
marginais ao uso do \LaTeX, ou de tópicos que requerem alguma
\TeX nica (i.e., podem empregar conceitos que não são abordados até
um ponto mais adiantado do texto).
\end{detalhe}
